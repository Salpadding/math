\section{Numerical Sequences and Series}

\subsection{Convergent Sequences}

\begin{thm}
    \label{thm:3-1-1}
    \begin{enumerate}
        \item suppose $\vx_n \in \R^k(n=1,2,3,...)$ and 

        \[
            \vx_n = (\alpha_{1,n},...,\alpha_{k,n})
        \]

        Then $\{ \vx_n \} \to \vx = (\alpha_1,...,\alpha_k)$ iff 

        \[
            \lim_{n \to \infty}\alpha_{j,n} = \alpha_j \quad (1 \le j \le k)
        \]

        \item suppose $\{ \vx_n \}$ and $\{ \vy_n \}$ are sequences in $\R^k$, $\{ \beta_n\}$ 
        is a sequence of real numbers, and $\vx_n \to \vx,\: \vy_n \to \vy,\: \beta_n \to \beta$. Then

        \begin{align*}
            \vx_n + \vy_n & \to \vx + \vy \\
            \langle \vx_n, \vy_n \rangle &\to \langle \vx, \vy \rangle \\
            \beta_n \vx_n & \to \beta \vx
        \end{align*}
    \end{enumerate}
\end{thm}

\begin{proof}

    \begin{enumerate}
        \item by $\lim_{n \to \infty}|\alpha_{j,n} - \alpha_j| = 0 \quad (1 \le j \le k)$, we got:

    \[
        \lim_{n \to \infty}\max_{j \le k}|\alpha_{j,n} - \alpha_j| = \max_{j \le k}\lim_{n \to \infty}|\alpha_{j,n} - \alpha_j| = 0
    \]

    which means $d_{l_{\infty}}(\vx_n, \vx) \to 0$, because 

    \begin{align*}
    0 \le d_{l_2}(\vx_n, \vx) \le kd_{l_{\infty}}(\vx_n, \vx)
    \end{align*}

    by squeeze test we got:

    \[
        \lim_{n \to \infty}d_{l_2}(\vx_n, \vx) = 0
    \]

    which means $\vx_n \to \vx$

    on the other hand, if $\vx_n \to \vx$, by


    \begin{align*}
    0 & \le d_{l_{\infty}}(\vx_n, \vx) \le  \sqrt{k} d_{l_1}(\vx_n, \vx) \le d_{l_2}(\vx_n, \vx) 
    \end{align*}

    we got

    \[
        0  \le |\alpha_{j,n} - \alpha_j| \le d_{l_\infty}(\vx_n, \vx) \quad 1 \le j\le k
    \]

    by squeeze test $\alpha_{j,n} \to \alpha_j \quad 1 \le j \le k$

    \item it is easy to prove by above theorem and laws of limitation
    \end{enumerate}
\end{proof}

\begin{thm}
    \label{thm:3-1-2}
    suppose $\{p_n\}$ is a sequence under metric space, $p$ is a subsequential limit of $\{p_n\}$ iff at least one of below condition holds:

    \begin{enumerate}
        \item $p$ occurs in $\{p_n\}$ infinitely many times.

        \item $p$ is a limit point of element set $\{ p_n \}$
    \end{enumerate}
\end{thm}

\begin{proof}
    The sufficiency is easily to prove. Let's prove necessity:

    Assume for contradiction, $p$ occurs in $\{ p_n \}$ finitely many times, and $p$ is a subsequential limit of $\{ p_n\}$,
    we prove that $p$ is a limit point of $\{ p_n \}$

    let $N$ be large enough so that $\forall n \ge N,\: p_n \ne p$, since we have $p_{n_k} \to p$, by remove
    finitely many elements we got $p_{n_k} \to p,\; p_{n_k} \ne p$. Then $p$ becomes a limit point
    of $\{ p_{n_k}\}$, since $\{ p_{n_k}\} \subseteq \{ p_n \}$, $p$ is also a limit point of $\{ p_n\}$

\end{proof}

\begin{thm}
    \begin{enumerate}
        \item if $\{p_n\}$ is a sequence in a compact metric space $X$, then some subsequence of $\{ p_n\}$ converges to a point of $X$

        \item Every bounded sequence in $\R^k$ contains a convergent subsequence.
    \end{enumerate}
\end{thm}

\begin{proof}

    \begin{enumerate}
        \item if element set of $\{ p_n\}$ is finite, then at least one element of $\{ p_n \}$ occurs infinitely many times.
   Then this element become a subsequential limit of $\{ p_n \}$

   if element set of $\{ p_n \}$ is infinite, since $X$ is compact, then $\{ p_n \}$ has at least
   one limit point $p$ of $X$. 

   Take $p_{n_k}$ meets: $0 < d(p_{n_k}, p) < 1/n$ and $n_k > n_{k-1}$, we can construct a subsequence $p_{n_k} \to p$. 
   Because $B(p, 1/n) \cap \{ p_n \}$ contains infinitely many points.

        \item this is a result from above. Since bounded sequence could be placed into a compact metric space.
    \end{enumerate}
\end{proof}

\begin{thm}
    \label{thm:3-1-4}
    The subsequential limits of a sequence $\{p_n \}$ in a metric space  $X$
    form a closed subset of $X$
\end{thm}

\begin{proof}
    \begin{enumerate}
        \item First Method


   let $E^*$ be the set of all subsequential limits of $\{ p_n \}$. We will prove $(E^*)^C$ is open. 
   Take $p \in (E^*)^C$, then $p$ occurs in $\{ p_n\}$ at most finitely many times. Which means there exists $N$ and
   $\forall n \ge N, p_n \ne p$

   $p$ also cannot be a limit point of $\{ p_n \}$, which means there exists $r > 0$, and 
   $B(p, r) \cap \{ p_n\} \subseteq \{ p \}$ and $B(p, r) \cap \{ p_n\}_{n \ge N} = \emptyset$

   now we pick $q \in B(p,r)$, then $q$ also occurs in $\{ p_n\}$ at most finitely many times. And 
   $q$ is also not a limit pont of $\{ x_n \}$. By \autoref{thm:3-1-2}, $q$ could not be a subsequential limit of $\{ p_n \}$.

   So we got $B(p,r) \subseteq (E^*)^C$, which indicates that $(E^*)^C$ is open.

        \item Second Method

    let $E^*$ be the set of all subsequential limits of $\{ p_n \}$. We will prove $(E^*)' \subseteq E^*$

    since $q \in (E^*)'$, take $a_1 \in E^*$ so that $d(a_1, q) < 1$, since $a_1 \in E^*$, we can pick $p_{n_1}$
    and $d(p_{n_1}, a_1) \le 1$. 
    
    Similarly, we can continue take $a_2 \in E^*$ so that $d(a_2, q) < 1/2$ and $p_{n_2}$
    so that $d(p_{n_2}, a_2) \le 1/2$, where $n_2 \ge n_1$, thus we construct $\{p_{n_k}\}$ repeated.
    \end{enumerate}

\end{proof}

\begin{remark}
    For a bounded sequence $\{ p_n\}$, it has limit point iff element set $\{ p_n\}$ is infinite.
\end{remark}

\begin{thm}
    \label{thm:3-1-6}
    \begin{enumerate}
        \item if $\overline{E}$ is the closure of a set $E$ in a metric space $X$, then

        \[
            \mathrm{diam} \overline{E} = \mathrm{diam} E
        \]

        \item if $K_n$ is a sequence of nonempty compact sets in $X$, such that $K_n \supseteq K_{n+1}$ and

        \[
            \lim_{n \to \infty} \mathrm{diam} K_n = 0
        \]

        then

        \[
            K = \bigcap_{n=1}^{\infty} K_n
        \]

        $K$ consists of exactly one point
    \end{enumerate}
\end{thm}

\begin{proof}
    \begin{enumerate}
        \item since $E \subseteq \overline{E}$, so

        \[
            \mathrm{diam} \overline{E} \ge \mathrm{diam} E
        \]

        pick $p,q \in \overline{E}$, then for any $\epsilon > 0$, there exists 
        $p',q' \in E$ so that $d(p,p') < \epsilon, d(q,q') < \epsilon$ Hence

        \begin{align*}
            d(p,q) &\le d(p,p') + d(p',q') + d(q',q) \le d(p',q') + 2\epsilon \\
            & \le \mathrm{diam} E + 2\epsilon
        \end{align*}

        since $p, q$ is arbitrary, take sup of $d(p,q)$ we got

        \[
            \mathrm{diam}(\overline{E}) \le \mathrm{diam} E + 2\epsilon
        \]

        take $\epsilon \to 0$:

        \[
            \mathrm{diam}(\overline{E}) \le \mathrm{diam} E
        \]

        \item By \autoref{col:2-3-6}, $K$ should not be empty. if $K$ contains two points, then $\mathrm{diam}(K) > 0$
        which is contradict with $\forall n,\: K \subseteq K_n $:

        \[
            \mathrm{diam}(K) \le \mathrm{diam}(K_n)
        \]
    \end{enumerate}
\end{proof}

\begin{thm}
    \begin{enumerate}
        \item In any metric space, every convergent sequence is a Cauchy sequence.

        \item Cauchy sequence converges iff it has at least one subsequential limit.
        
        \item If $X$ is a compact metric space and $\{ p_n \}$ is a Cauchy sequence in $X$, then $\{ p_n \}$ converges.

        \item In $\R^k$, every Cauchy sequence converges.
    \end{enumerate}
\end{thm}

\begin{proof}
    \begin{enumerate}
        \item It is easily to prove with usage of triangle inequality.

        \item Assume Cauchy sequence $p_n$ has a subsequential limit $p_{n_k} \to p$,
        by apply property of cauchy sequence and subsequential limit, we can pick $N$ large enough, and $n_k \ge N, d(p_{n_k}, p) \le \epsilon$ 
        
        And $\forall m \ge N$, $d(p_m, p) \le d(p_m , p_{n_k}) + d(p_{n_k}, p) \le 2\epsilon$

        \item 
        
        \begin{enumerate}
            \item Method 1


        if $\{ p_n \}$ is finite, then it has at least one subsequential limit ,by (2) $\{p_n\}$ converges.

        if $\{ p_n \}$ is infinite, then it has at least one limit point $p$ in $X$, which is also a subsequential limit,
        by (2) $\{ p_n\}$ converges


            \item Method 2

            Denote $E_n$ as:

            \[
                E_n := \{ p_k : k \ge n \}
            \]

            Since $\overline{E_n} \supseteq \overline{E_{n+1}}$ and $\overline{E_n}$ is compact, and 
            $\lim_{n \to \infty} \mathrm{diam}(E_n) = 0$, by \autoref{thm:3-1-6} we got

            \[
                \bigcap_{n=1}^{\infty}\overline{E_n} = \{ p \}
            \]

            so $\forall n$ by $p, p_n \in \overline{E_n}$, we got:

            \[
                0 \le d(p_n, p) \le \mathrm{diam}(\overline{E_n}) 
            \]

            which indicates $d(p_n, p) \to 0$ by squeeze test
        \end{enumerate}

        \item Since every Cauchy sequence is bounded. By (3), every Cauchy sequence in $\R^k$ converges.
    \end{enumerate}
\end{proof}

\subsection{Upper and Lower Limits}

\begin{definition}[Upper Limit]
    \label{def:3-2-1}
    let $\{ s_n \}$ be a sequence of real numbers, let $E$ be the subsequential limits of $\{s_n\}$, then the
    upper limit of $\{ s_n \}$, $s^* \in \R^*$ is defined as:

    \[
        s^* = \sup E
    \]

    Similarly for lower limit $s_* \in \R^*$:

    \[
        s_* = \inf E
    \]
\end{definition}

\begin{thm}
    By \autoref{def:3-2-1}, The following properties holds:

    \begin{enumerate}
        \item $s^* \in E$

        \item 

        \[
            s^* = \varlimsup_{n \to \infty} s_n
        \]

        \item if $x > s*$, then there finitely many $s_n$ meets $s_n \ge x$

        Moreover $s^*$ is only number meets (1) and (3)
    \end{enumerate}
\end{thm}

\begin{proof}
    \begin{enumerate}
        \item if $s^* = \infty$ 

        which means $E$ has no upper bound, so $\{ s_n \}$ has no upper bound. Hence
        we can construct a subsequential limit converges to $\infty$

        if $s^*$ is real number, by \autoref{thm:3-1-4}, $s^* \in E$

        if $s^* = -\infty$, which means $s_n = -\infty$, then $s_n \to s^*$, so $s^* \in E$

        \item if $s^* = \infty$


        which means $E$ has no upper bound, so we got:

        \[
            \varlimsup_{n \to \infty} s_n = \infty
        \]

        if $s^*$ is a real number, since $s^* \in E$, there exists subsequence $s_{n_k} \to s^*$, by
        $s_{n_k} \le \sup (s_m)_{m \ge n_k}$, we got:

        \[
            s^* \le \varlimsup_{n \to \infty} s_n
        \]

        on the other hand, let's define 

        \begin{align*}
            t_n &= \sup (s_m)_{m \ge n} \\
            \varlimsup_{n \to \infty}s_n = L^+
        \end{align*}

        and construct subsequence $s_{n_k}$:
        \begin{align*}
            t_{n_1} & \ge s_{n_1}  \ge t_1 - 1\quad (n_1 \ge 1) \\
            t_{n_2} & \ge s_{n_2}  \ge t_{n_1} - 1/2 \quad (n_2 \ge n_1) \\
            ... \\
            t_{n_k} & \ge s_{n_k}  \ge t_{n_{k-1}} - 1/k \quad (n_k \ge n_{k-1}) \\
        \end{align*}

        it is obviously that $s_{n_k} \to L^+$, so that $L^+ \in E$ and

        \[
            s^* \ge \varlimsup_{n \to \infty} s_n
        \]


        if $s^* = -\infty$, it is trivial because $\forall n,\: s_n = -\infty$

        \item if $x > s^*$, then $s^* < \infty$, assume there are infinitely many $s_n$ meets $s_n \ge x$,
        then we got

        \[
            \varlimsup_{n \to  \infty} s_n = L^+ \ge x > s^*
        \]

        which is contradict with $L^+ = s^*$
    \end{enumerate}
\end{proof}

\begin{exercise}
    prove $n^{1/n} \to 1$
\end{exercise}

\begin{proof}
    we prove a lemma at first, for $\epsilon \ge 0$:

    \[
        (1+\epsilon)^n \ge 1 + n\epsilon,\: \forall n \ge 1
    \]

    for $n =1$, obviously, assume holds for $n = k$, consider $n=k+1$

    \begin{align*}
        (1+\epsilon)^{k+1} & =(1+\epsilon)^k (1 + \epsilon)  \\
        & \ge (1+ k\epsilon)(1 + \epsilon) \\
        & \ge  1+ (k+1) \epsilon
    \end{align*}

    let's define $t_n = n^{1/n}$, then $t_n \ge 1$ obviously

    then we define $ \epsilon_n = t_n - 1,\: \epsilon_n \ge 0$, then we have

    \begin{align*}
        (1+\epsilon_n)^n &= n \ge 1 + n \epsilon_n \\
        \epsilon_n &\le (n-1)/n
    \end{align*}

    so we have $\epsilon_n \to 0$ and $t_n \to 1$
    
    another method is consider that:

    \begin{align*}
        (1+\epsilon_n)^n = n \ge \frac{n(n-1)}{2}\epsilon_n^2 \\
        \epsilon_n \le \sqrt{\frac{n-1}{2}}
    \end{align*}
\end{proof}

\begin{exercise}
    prove:

    \[
        \frac{n^{\alpha}}{(1+p)^n} \to 0
    \]

    fix $k > \alpha$ and $n > 2k$, then we have

    \begin{align*}
        (1+p)^n &> 1 + \binom{n}{k}p^k \ge 1 + \frac{n(n-1)..(n-(k-1))}{k!}p^k \\
        & \ge 1 + \frac{(n/2 )^k}{k!}p^k \\
        & \ge 1 + \frac{n^k}{2^k k!}p^k
    \end{align*}

    and 

    \[
        \frac{n^\alpha}{(1+p)^n} \le \frac{n^{\alpha} 2^k k!}{n^kp^k} \le \frac{2^k k!}{p^k} n^{\alpha - k} \le  \frac{2^k k!}{p^k} \frac{1}{n}
    \]

    take $n \to \infty$, we have

    \[
        \frac{n^\alpha}{(1+p)^n} \to 0
    \]
\end{exercise}

\subsection{Series}

\begin{thm}
    Some theorems about series:

    \begin{enumerate}
        \item $\sum_{n=1}^{\infty} a_n$ converges iff

        \[
            \lim_{k,m \to \infty} \lvert \sum_{n=k}^{m}a_n \rvert = 0
        \]

        \item if $\sum_{n=1}^{\infty} a_n$ converges then $a_n \to 0$

        \item if $a_n \ge 0$, then $\sum_{n=1}^{\infty} a_n$ converges iff $\sum_{n=1}^{N} a_n$ is bounded for $N \ge 1$
    \end{enumerate}
\end{thm}

\begin{proof}
    \begin{enumerate}
        \item this is a directly result of Cauchy sequence

        \item this is the result of (1)

        \item this is the result that: monotone sequence converges iff it is bounded
    \end{enumerate}
\end{proof}

\begin{thm}
    \label{thm:3-3-2}
    let $a_n \ge 0 $ and $a_1 \ge a_2 ..$, and we define

    \[
        S_n = \sum_{k=1}^{n}a_k
    \]

    and

    \[
        T_n = \sum_{k=0}^{n-1}2^ka_{2^k}
    \]

    then $S_n$ is bounded iff $T_n$ is bounded

\end{thm}

\begin{proof}
    steps:

    \begin{enumerate}
        \item $T_n$ bounded $\to$ $S_n$ bounded

        \begin{align*}
            S_{1 + 2 + 4 + .. 2^{n-1}} & = S_{2^{n} - 1} = a_1 + (a_2 + a_3) + .. (a_{2^{n-1}} + .. + a_{2^{n} - 1}) \\
                & \le a_1 + 2a_2 + .. + 2^{n-1}a_{2^{n-1}} \\
                & \le T_{n}
        \end{align*}

        \item $S_n$ bounded $\to$ $T_n$ bounded

        \begin{align*}
            S_{1 + 2 + 4 + .. 2^{n-1}} & = S_{2^{n} - 1} = a_1 + (a_2 + a_3) + .. (a_{2^{n-1}} + .. + a_{2^{n} - 1}) \\ 
            & \ge a_2 + 2a_4 + .. + 2^{n-1}a_{2^n} \\
            & \ge \frac{1}{2}T_{n+1} - \frac{1}{2}a_1
        \end{align*}
    \end{enumerate}
\end{proof}

\begin{thm}
    let $p>0$, then $\sum n^{-p}$ converges iff $p > 1$
\end{thm}

\begin{proof}
    since $n^{-p} \ge 0$, then $\sum n^{-p}$ converges iff $\sum n^{-p}$ is bounded.

    by \autoref{thm:3-3-2}, $\sum n^{-p}$ converges iff:

    \[
        T_n = \sum_{k=0}^{N-1}2^k (2^k)^{-p} = \sum_{k=0}^{N-1}(2^{(1-p)})^k
    \]

    and $T_n$ is bounded, which indicates $2^{1-p} < 1$, and thus $p > 1$
\end{proof}

\subsection{The Number e}

\begin{definition}[exponention]
    \[
        \e = \sum_{n=0}^{\infty}\frac{1}{n!}
    \]
\end{definition}

\begin{remark}
    The series $\sum_{n=0}^{\infty}\frac{1}{n!}$ converges because it is bounded.
    Because $\frac{1}{n!} < 2^{-n}$ but for serval finite beginner items:

    \begin{align*}
        \e & = \frac{1}{1} + \frac{1}{1} +  \frac{1}{2} + \frac{1}{2\cdot 3} + .. + \frac{1}{2\cdot 3 .. \cdot n} + ...\\
        & \le 1 + 1 + 2^{-1} + 2^{-2} + .. + 2^{-n} + ...\\
        & \le 3
    \end{align*}
\end{remark}

\begin{thm}
    \[
        \lim_{n \to \infty}(1+\frac{1}{n})^n = \e
    \]
\end{thm}

\begin{proof}
    we fix $m > 0$ at first, and let $n > m$
    \begin{align*}
        (1+\frac{1}{n})^{n} &= \sum_{k=0}^{n}\binom{n}{k}(\frac{1}{n})^k \\
        &= 1 + 1 + \frac{1}{2!}(1-\frac{1}{n}) +\frac{1}{3!}(1-\frac{1}{n})(1-\frac{2}{n}) + .. + \frac{1}{n!}(1-\frac{1}{n})..(1-\frac{n-1}{n}) \\
        &= 1 + 1 + \frac{1}{2!}(1-\frac{1}{n}) + .. + \frac{1}{m!}(1-\frac{1}{n})..(1-\frac{m-1}{n}) + .. + \frac{1}{n!}(1-\frac{1}{n})..(1-\frac{n-1}{n})\\
        & \ge 1 + 1 + \frac{1}{2!}(1-\frac{1}{n}) + .. + \frac{1}{m!}(1-\frac{1}{n})
    \end{align*}

    take $n \to \infty$, we have

    \begin{align*}
        \varliminf_{n \to \infty}(1+\frac{1}{n})^n &\ge 1 + 1 + \frac{1}{2!}(1-\frac{1}{n}) + .. + \frac{1}{m!}(1-\frac{1}{n}) \\
        &\ge 1 + 1 + \frac{1}{2!} + .. + \frac{1}{m!}
    \end{align*}

    since $m$ is arbitrary, take $m \to \infty$, we have

    \[
        \varliminf_{n \to \infty}(1+\frac{1}{n})^n \ge \e
    \]

    by the expansion of $(1+1/n)^n$, it is obviously that $(1+1/n)^n \le \sum_{k=0}^n(1/n!)$, which means

    \[
        \varlimsup_{n \to \infty}(1+\frac{1}{n})^n \le \e
    \]

    after all, we got:

    \[
        \varliminf_{n \to \infty}(1+\frac{1}{n})^n = \varlimsup_{n \to \infty}(1+\frac{1}{n})^n = \e
    \]

    which indicts that $(1+1/n)^n \to \e$
\end{proof}

\subsection{Root and Ratio Tests}

\begin{thm}[Root Test]
    \label{thm:thm-root-test}
    let sequence $a_n \in \R$ and $\alpha$ as

    \[
        \alpha = \varlimsup_{n \to \infty} \lvert a_n\rvert^{1/n}
    \]

    Then:

    \begin{enumerate}
        \item if $\alpha < 1$

        $\sum a_n$ converges absolutely

        \item if $\alpha > 1$

        $\sum a_n$ diverges


        \item if $\alpha = 1$

        may converge or diverge
    \end{enumerate}
\end{thm}

\begin{proof}
   \begin{enumerate}
    \item if $\alpha < 1$, by

    \[
        \varlimsup_{n \to \infty}\lvert a_n\rvert^{1/n} = \alpha < 1
    \]

    we pick $c: \alpha < c < 1$, by property of upper limit, then there exists finitely many $n$ such that

    \[
        \lvert a_n\rvert^{1/n} \ge c
    \]

    which means there exists $N$ and $\forall n \ge N$:

    \begin{align*}
        \lvert a_n\rvert^{1/n} &< c \\
        \lvert a_n\rvert &< c^n \\
    \end{align*}

    which indicates that $\sum \lvert a_n \rvert$ is bounded:

    \begin{align*}
        \sum_{n=1}^{N+k}\lvert a_n \rvert &= \sum_{n=1}^{N-1}\lvert a_n \rvert + \sum_{n=N}^{N+k}\lvert a_n \rvert \\
        & \le \sum_{n=1}^{N-1}\lvert a_n \rvert + \sum_{n=N}^{N+k}c^n \\
        & \le \sum_{n=1}^{N-1}\lvert a_n \rvert + c^N(\frac{1}{1-c})
    \end{align*}
   \end{enumerate} 

   \item if $\alpha > 1$, we pick $c: \alpha > c  > 1$, by property of upper limit, there must exists 
   infinitely $n$ such that $\lvert a_n \rvert ^{1/n} > c$ thus infinitely $\lvert  a_n \rvert \ge 1$. 
   which is contradict with $a_n \to 0$ if $\sum a_n$ converges

   \item if $\alpha = 1$
   
   consider $a_n = 1$, which $\sum a_n $ diverges and $\alpha = 1$

   then consider $a_n = 1/n^2$, where $\alpha = 1$, but $\sum a_n$ converges
\end{proof}

\begin{corollary}
    \label{thm:thm-root-test-complex}
    The above theorem could be also applied to complex sequence $z_n$ 
\end{corollary}

\begin{proof}
    It is easily to prove by take real part and imaginary part of $z_n$:

    \begin{align*}
        \lvert \mathrm{Re}(z_n) \rvert ^{1/n} &\le \lvert z_n \rvert ^{1/n} < 1 \\
        \lvert \mathrm{Im}(z_n) \rvert ^{1/n} & \le \lvert z_n \rvert ^{1/n} < 1 \\
\sum_{n = 1}^{\infty}\mathrm{Re}(z_n) &= L_1 \\
\sum_{n = 1}^{\infty}\mathrm{Im}(z_n) &= L_2 \\
        \sum_{n = 1}^{\infty}z_n & = \sum_{n = 1}^{\infty}\mathrm{Re}(z_n) + \mathrm{i}\left(\sum_{n = 1}^{\infty}\mathrm{Im}(z_n) \right) \\
        &= L_1 + \mathrm{i}L_2
    \end{align*}

    Similarly, if $\sum z_n$ converges, by property of cauchy sequence we got:

    \[
        \lim_{k,p \to \infty}\lvert \sum_{n=k}^{p} z_n \rvert = 0
    \]

    which indicates that $\lvert z_n \rvert \to 0$, so $\alpha > 1$ indicates $\sum z_n$ diverges
\end{proof}

\begin{thm}[Ratio Test]
    \label{thm:thm-ratio-test}
   Assume $a_n \ne 0$, then define $\alpha, \beta$ as

   \[
        \alpha = \varlimsup_{n \to \infty}\frac{\lvert a_{n+1}\rvert}{\lvert a_{n} \rvert}
   \]

   and


   \[
        \beta = \varliminf_{n \to \infty}\frac{\lvert a_{n+1}\rvert}{\lvert a_{n} \rvert}
   \]

    Then:

    \begin{enumerate}
        \item if $\alpha < 1$

        $\sum a_n$ converges absolutely

        \item if $\beta > 1$

        $\sum a_n$ diverges


        \item other cases

        may converge or diverge
    \end{enumerate}
\end{thm}

\begin{proof}
We will prove by: if $c_n$ is a sequence of positive real number, then:


\begin{align*}
\varliminf_{n \to \infty}\frac{c_{n+1}}{c_n} \le \varliminf_{n \to \infty} c_n ^{1/n} \le \varlimsup_{n \to \infty} c_n ^{1/n} \le \varlimsup_{n \to \infty} \frac{c_{n+1}}{c_n}
\end{align*}

Assume $c_n > 0$ then

let 

\[
\varlimsup_{n \to \infty}\frac{c_{n+1}}{c_n} = \alpha
\]


take $\epsilon > 0$ then exists $N$ : $\forall n \ge N$, $c_{n+1}/c_n \le \alpha + \epsilon$,
so $\forall k \ge 1$, $c_{N+k} \le c_N (\alpha+\epsilon)^k$,

take $N+k$ root at both side:

\[ \sqrt[N+k]{c_{N+k}} \le \sqrt[N+k]{c_{N}} (\alpha+\epsilon)^{k/(N+k)} \]

take upper limit on $k$:

\[ \varlimsup_{n \to \infty}c_n^{1/n} \le \alpha + \epsilon \]

since $\epsilon$ is arbitrary we got

\[
\varlimsup_{n \to \infty}c_n^{1/n} \le \alpha
\]

similarly, let 

\[
\varliminf_{n \to \infty}\frac{c_{n+1}}{c_n} = \beta
\]

then take $\epsilon > 0$, exists $N$:  $\forall n \ge N$ got $c_{n+1}/c_n \ge \beta - \epsilon$,
so $\forall k \ge 1$: $c_{N+k} \ge c_N (\beta - \epsilon)^k$,

take $N+k$ root at both side:

\[ \sqrt[N+k]{c_{N+k}} \ge \sqrt[N+k]{c_{N}} {(\beta-\epsilon)}^{k/(N+k)} \]

take lower limit on $k$:

\[ \varliminf_{n \to \infty}c_n^{1/n} \ge \beta - \epsilon \]    


since $\epsilon$ is arbitrary we got

\[ \varliminf_{n \to \infty}c_n^{1/n} \ge \beta \]    
\end{proof}

\begin{corollary}
    This theorem also apply for complex sequence $z_n, z_n > 0$. 
    It is easily to prove by take $c_n = \lvert z_n \rvert$,
    and then apply \autoref{thm:thm-ratio-test} and \autoref{thm:thm-root-test}.
\end{corollary}

\subsection{Power Series}

\begin{definition}
    Given a sequence $c_n$ of complex numbers, the series

    \[
        \sum_{n=0}^{\infty} c_nz^n
    \]

    is called a power series.
\end{definition}

\begin{thm}
    define $\alpha$ and $R$ as

    \[
        \alpha = \varlimsup_{n \to \infty} \lvert c_n\rvert^{1/n},\quad R = \frac{1}{\alpha}
    \]

    Then $\sum_{n=0}^{\infty} c_nz^n$ converges if $\lvert z \rvert < R$ and 
$\sum_{n=0}^{\infty} c_nz^n$ diverges if $\lvert z \rvert > R$

    $R$ is called the radius of $\sum_{n=0}^{\infty} c_nz^n$
\end{thm}

\begin{proof}
    It is easily to prove by apply \autoref{thm:thm-root-test-complex}
\end{proof}

\begin{thm}[Sum by parts]
    \label{thm:sum-by-parts} 

\begin{align*}
    A_n &= \sum_{k=0}^{n}A_k \\
    \sum_{n=k}^{p} a_nb_n &= \sum_{n=k}^{p-1}A_n(b_n - b_{n+1}) + A_pb_p - A_{k-1}b_k
\end{align*}
    
\end{thm}

\begin{proof}
    \begin{align*}
     \sum_{n=k}^{p} a_nb_n &= \sum_{n=k}^{p}(A_n - A_{n-1})b_n \\   
     &= \sum_{n=k}^{p}A_nb_n- \sum_{n=k}^{p}A_{n-1}b_n \\
     &= \sum_{n=k}^{p-1}A_nb_n + A_pb_p - \sum_{n=k-1}^{p-1}A_{n}b_{n+1} \\
     &= \sum_{n=k}^{p-1}A_nb_n + A_pb_p - \sum_{n=k}^{p-1}A_{n}b_{n+1} - A_{k-1}b_k \\
     &= \sum_{n=k}^{p-1}A_n(b_n - b_{n+1}) + A_pb_p - A_{k-1}b_k
    \end{align*} 
\end{proof}

\begin{thm}
    \label{thm:3-6-4}
   suppose:
   
   \begin{enumerate}
    \item partial sum of $\sum a_n$ is bounded
    
    \item $b_0 \ge b_1 \ge b_2 ..$

    \item $\lim_{n \to \infty}b_n = 0$
   \end{enumerate}

   Then $\sum a_nb_n$ converges
\end{thm}

\begin{proof}
    by \autoref{thm:sum-by-parts}:

    \begin{align*}
        \left| \sum_{n=k}^{N}a_nb_n \right| &= \left|\sum_{n=k}^{N-1}A_n(b_n - b_{n+1}) + A_Nb_N - A_{k-1}b_k \right| \\ 
    \end{align*}

    as $k,N \to \infty $

    by $A_N$ bounded and $b_N \to 0 $, so $A_Nb_N \to 0$ and $A_{k-1}b_k \to 0$

    and

    \begin{align*}
        \left|  \sum_{n=k}^{N-1}A_n(b_n - b_{n+1}) \right| &\le   \sum_{n=k}^{N-1} \left|A_n \right|(b_n - b_{n+1})  \\
        & \le \sum_{n=k}^{N-1}M(b_n - b_{n+1}) \\
        & \le M (b_k - b_N)
    \end{align*}

    so $\sum_{n=k}^{N-1}A_n(b_n - b_{n+1}) \to 0$

    after all, partial sum of $\sum a_n b_n$ is cauchy sequence, so $\sum a_n b_n$ converges
\end{proof}

\begin{corollary}
    This also holds when $a_n \in \mathbb{C}$
\end{corollary}

\begin{thm}
    suppose:

    \begin{enumerate}
        \item $\lvert c_1 \rvert \ge \lvert c_2 \rvert \ge \lvert c_3 \rvert \ge ...$

        \item $c_k \to 0$

        \item $c_{2m-1} \ge 0, c_{2m} \le 0$
    \end{enumerate}

    Then: $\sum c_n$ converges
\end{thm}

\begin{proof}
    define $a_n = (-1)^{n+1}$, and $b_n = \lvert c_n \rvert $, by \autoref{thm:3-6-4}

    \[
        \sum (-1)^{n+1} \lvert c_n \rvert = \sum  c_n
    \]

    converges
\end{proof}


\begin{thm}
    suppose radius of $\sum c_n z^n$ is $1$, and suppose 
    $c_0 \ge c_1 \ge c_2 \ge ... $, $c_n \to 0$

    Then $\sum c_n z^n$ converges at every point on circle $\lvert z \rvert = 1$,
    except possibly at $z = 1$
\end{thm}

\begin{proof}
   we prove by $\sum_{n=0}^{N-1}z^n$ is bounded:

   \begin{align*}
    \left| \sum_{n=0}^{N-1}z^n \right| &= \left| \frac{1-z^N}{1-z} \right| \\
    & \le \left|\frac{1}{1-z} \right| \left( 1 + \left| z\right|^N\right) \\
    & \le \left|\frac{2}{1-z} \right|
   \end{align*}

   and apply \autoref{thm:3-6-4}
\end{proof}

\begin{thm}
    let

    \[
        \sum_{n=0}^{\infty} \lvert a_n \rvert < \infty
    \]

    and

    \[
        \sum_{n=0}^{\infty} b_n
    \]

    converges

    Then:

    \[
        \sum_{N=0}^{\infty}(a_0b_N + a_1b_{N-1} + .. + a_Nb_0) = \left(\sum_{n=0}^{\infty} a_n \right)\left(\sum_{n=0}^{\infty} b_n \right)
    \]
\end{thm}

\begin{proof}
    let's define

    \begin{align*}
        A_n &= \sum_{k=0}^{n}a_k \\
        B_n &= \sum_{k=0}^{n}b_k \\
        \beta_n &= \sum_{k=0}^{\infty}b_k - B_n \\
        \sum_{k=0}^{\infty}\lvert a_k \rvert &= M \\
        A &= \sum_{k=0}^{\infty}a_k \\
        B &= \sum_{k=0}^{\infty}b_k \\
    \end{align*}

    \begin{align*}
        & a_0b_0 + (a_0b_1 + a_1b_0) + .. + (a_0b_N + a_1b_{N-1} + .. + a_Nb_0) \\
        & = a_0B_N + a_1B_{N-1} + .. + a_NB_0 \\
        &= a_0(B - \beta_N) + a_1(B-\beta_{N-1}) + .. + a_N(B-\beta_{0}) \\
        &= A_NB - \left( a_0\beta_N + a_1 \beta_{N-1} + .. + a_N\beta_{0}\right)
    \end{align*}    

    we fix $m > 0$ be large enough so that:

    \[
        \forall k \ge m,\: \lvert \beta_k \rvert \le \epsilon
    \]

    
    and let $N > m$, then we have

    \begin{align*}
        & \lvert a_0\beta_N + a_1 \beta_{N-1} + .. + a_N\beta_{0} \rvert \\
        & = \lvert a_0\beta_N + a_1 \beta_{N-1} + .. + a_m\beta_{N-m} + a_{m+1}\beta_{N-(m+1)} + .. + a_N\beta_{0} \rvert \\
        & \le \lvert \beta_0a_N + \beta_1 a_{N-1} + .. + \beta_ma_{N-m} \rvert + \lvert \beta_{m+1}a_{N-(m+1)} + .. + \beta_Na_{0} \rvert \\
        & \le \lvert \beta_0a_N + \beta_1 a_{N-1} + .. + \beta_ma_{N-m} \rvert + \epsilon \left(\sum_{k=0}^{N-(m+1)}\lvert a_k \rvert \right) \\
        & \le \lvert \beta_0a_N + \beta_1 a_{N-1} + .. + \beta_ma_{N-m} \rvert + \epsilon M
    \end{align*}

    since

    \begin{align*}
        \lvert \beta_0a_N + \beta_1 a_{N-1} + .. + \beta_ma_{N-m} \rvert \le \sum_{k=0}^{m} \lvert \beta_k \rvert \lvert a_{N-k}\rvert
    \end{align*}

    and $\lvert \beta_k \rvert \lvert a_{N-k} \rvert \to 0$ as $N \to \infty$, since above expression is finite, we got:

    \[
        \lim_{N \to \infty}\lvert \beta_0a_N + \beta_1 a_{N-1} + .. + \beta_ma_{N-m} \rvert = 0
    \]

    thus we got

    \[
        \varlimsup_{N \to \infty}\lvert a_0\beta_N + a_1 \beta_{N-1} + .. + a_N\beta_{0} \rvert \le \epsilon M
    \]

    since $\epsilon$ is arbitrary, take $\epsilon \to 0$, we have

    \[
        \varlimsup_{N \to \infty}\lvert a_0\beta_N + a_1 \beta_{N-1} + .. + a_N\beta_{0} \rvert \le 0
    \]

    so we have

    \begin{align*}
        &\lim_{N \to \infty}a_0b_0 + (a_0b_1 + a_1b_0) + .. + (a_0b_N + a_1b_{N-1} + .. + a_Nb_0) \\
        &= \sum_{N=0}^{\infty}\sum_{k=0}^{N}a_kb_{N-k} = AB
    \end{align*}
\end{proof}

\begin{thm}
    \label{thm:3-6-9}
    let $\sum a_k$ converges while not converge absolutely. Then $\sum a_n^+ = \infty, \sum a_n^- = -\infty$


    \begin{align*}
        a_n^+ &= \frac{\lvert a_n \rvert + a_n}{2} \\
        a_n^- &= \frac{a_n - \lvert a_n \rvert}{2} \\
    \end{align*}
\end{thm}

\begin{proof}
    because $\lim_{N \to \infty}\sum_{n=0}^{N}\lvert a_n \rvert = \infty$ and $\lim_{N \to \infty}\sum_{n=0}^{N}a_n = c$, then:

    by laws of limit:

    \[
        \sum_{n=0}^{\infty}a_n^+ = \lim_{N \to \infty}\sum_{n=0}^{\infty}\frac{\lvert a_n \rvert + a_n}{2} = \infty
    \]

    similarly for  $\sum_{n=0}^{\infty}a_n^-$
\end{proof}

\begin{thm}
    \label{thm:3-6-10}
    let $\sum a_n$ convergent while not absolutely convergent, and $-\infty \le \alpha \le \beta \le \infty$ 
    
    prove: exists a rearrangement $a_{n_k}$ such that

    \begin{align*}
        \varlimsup_{N \to \infty}\sum_{k=1}^{N}a_{n_k} = \beta \\
        \varliminf_{N \to \infty}\sum_{k=1}^{N}a_{n_k} = \alpha \\
    \end{align*}

\end{thm}

\begin{proof}
    let's define:

    \begin{align*}
        a_n^{+} &= \frac{|a_n| + a_n}{2} \\
        a_n^{-} &= \frac{|a_n| - a_n}{2} \\
    \end{align*}

    by \autoref{thm:3-6-9} we have:

    \begin{align*}
        \sum_{n=1}^{\infty}a_n^{+} = \sum_{n=1}^{\infty}a_n^{-} = \infty
    \end{align*}


    now we define $\alpha_n \to \alpha, \beta_n \to \beta, \alpha_n \le \beta_n$, and $p_1, p_2, ..$ be non-negative item of $a_n$ while $q_1, q_2, .. $
    be negative items. Both $p_k$ and $q_k$ should be infinite, otherwise $a_n$ would be absolutely convergent.

    now we define $m_1$ be minimal integer that satisfy:

    \[
        p_1 + p_2 + .. + p_{m_1} > \beta_1
    \]

    and $k_1$ be minimal integer that satisfy

    \[
        p_1 + p_2 + .. + p_{m_1} - q_1 -q_2 - .. -q_{k_1} < \alpha_1
    \]

    since $\sum p_k = \infty$ and $\sum q_k = \infty$, we can construct infinite $m_1,m_2, ...$ and $k_1,k_2, ..$

    then we define :

    \begin{align*}
        S_{m_1} & = P_1 = p_1 + p_2 + .. + p_{m_1} > \beta_1\\
        S_{m_1 + k_1} &= Q_1 = P_1 - q_1 -q_2 - .. - q_{k_1} < \alpha_1 \\
        S_{m_1 + k_1 + m_2}  &= P_2 = Q_1 + p_{m_1 + 1} + .. + p_{m_2} > \beta_2 \\
        S_{m_1 + k_1 + m_2 + k_2} =Q_2 &= P_2 - q_{k_1 + 1} -q_{k_1 + 2} - .. - q_{k_2} < \alpha_2 \\
        ...
    \end{align*}

    consider that:

    \begin{align*}
       & P_n - p_{m_n} \le \beta_n < P_n \\
       & |P_n - \beta_n | \le \lvert p_{m_n} \rvert \\
       & \lim_{n \to \infty} P_n  = \beta
    \end{align*}

    since $P_{n}$ is a subsequence of $S_n$, so

    \[
        \varlimsup_{n \to \infty}S_n \ge \beta
    \]

    pick any $\epsilon > 0$, and $N$ large enough so that $\forall n \ge N,\: \beta_n \le \beta + \epsilon$, by $P_n - p_{m_n} \le \beta_n \le \beta + \epsilon$,
    there are infinitely many $k$ so that $S_k \le \beta + \epsilon$, and thus
    
    \[
        \varlimsup_{n \to \infty}S_n \le \beta + \epsilon
    \]

    since $\epsilon$ is arbitrary, after all we proved:

    \[
        \varlimsup_{n \to \infty}S_n = \beta
    \]

    similarly we can prove

    \[
        \varliminf_{n \to \infty}S_n = \alpha
    \]

\end{proof}

\begin{thm}
    let $\sum a_n$ convergent while not absolutely convergent, and $-\infty \le L \le \infty$,
    
    prove: exists a rearrangement $a_{n_k}$ such that

    \[
        \sum_{n=0}^{\infty}a_n = L
    \]

\end{thm}

\begin{proof}
    \begin{enumerate}
        \item if $L = \infty$

        by \autoref{thm:3-6-10} we pick $\beta_n = n$ and $\alpha_n = n-1/n$


        \item if $L = -\infty$

        by \autoref{thm:3-6-10} we pick $\beta_n = -n$ and $\alpha_n = -n-1/n$


        \item if $L \in \R$

        by \autoref{thm:3-6-10} we pick $\beta_n = L + 1/n$ and $\alpha_n = L - 1/n$
    \end{enumerate}
\end{proof}

\begin{thm}
    let $a_n$ converges absolutely, and $a_{f(n)}$ is a rearrangement, then we have:

    \[
        \sum_{n=0}^{\infty}a_{f(n)} = \sum_{n=0}^{\infty}a_{n}
    \]
\end{thm}

\begin{proof}
    it is easily to prove that $\sum_{n=0}^{\infty}a_{f(n)}$ converges, since 
    $\sum_{n=0}^{N}\lvert a_{f(n)} \rvert$ is bounded

    Assume 

    \[
        \sum_{n=0}^{\infty}a_{n} = L
    \]

    and

    \begin{align*}
        \left| \sum_{n=0}^{N}a_{n} - L \right| &\le \epsilon \\
        \sum_{n=k}^{p} \lvert a_n \rvert & \le \epsilon
    \end{align*}

    when $N,k,p \ge N_{\epsilon}$
    
    let  
    
    \[ 
        M = \max_{n \le N_{\epsilon}}f^{-1}(n) 
    \]

    consider that:

    \begin{align*}
        X & = \{ n: n \le M+k, f(n) > N_{\epsilon} \} \\
        \sum_{n=0}^{M+k}a_{f(n)} &= \sum_{n=0}^{N_{\epsilon}}a_{n} + \sum_{n \in X}a_{f(n)} \\
    \end{align*}

    since

    \[
        \left| \sum_{n \in X}a_{f(n)} \right| \le \sum_{n \in X}\lvert a_{f(n)} \rvert \le \epsilon
    \]

    we got

    \[
        \left| \sum_{n=0}^{M+k}a_{f(n)} - \sum_{n=0}^{N_{\epsilon}}a_{n} \right| \le \epsilon
    \]

    by triangle inequality we got

    \[
        \left| \sum_{n=0}^{M+k}a_{f(n)} - L \right| \le 2\epsilon
    \]
\end{proof}

\begin{corollary}
    The above theorem could also be applied to complex number sequence $z_n \in \C$.
    And it is easily to prove by take real and image part of $z_n$
\end{corollary}

\section{Continuity}

\subsection{Limits of Functions}

\begin{definition}
    let $X$ and $Y$ be metric spaces, suppose $E \subseteq X$, $f$ maps $E$ to $Y$. 
    And $p$ is a limit point of $E$, We define $f(x) \to q$ as $x \to p$ or

    \[
        \lim_{x \to p}f(x) = q
    \]

    iff there exists $q \in Y$ with: $\forall \epsilon > 0$, exists $\delta > 0$ such that

    \[
        d_Y(f(x), q) < \epsilon \quad \forall x \in  \{x \in E: 0 < d_X(x,p) < \delta \}
    \]
\end{definition}


\begin{thm}
    \label{thm:limit-of-func-iff-limit-of-every-mapping-of-convergent-seq}
    Assume $f: E \to Y$, and $(Y, d_Y)$ is a metric space , $E$ is a subset of metric space $(X, d_X)$, and $f: E\to Y$. 
    $p$ is a limit point of $E$

    Then

    $\lim_{x \to p}f(x) = q$ iff for every sequence $p_n$ in $E \setminus \{p\}$, $f(p_n) \to q$  
\end{thm}

\begin{proof}
   sufficiency: 
   
   Assume $f(x)$ not converges to $q$ as $x \to p$ for contradiction. There should exists $\epsilon > 0$,
    and $\forall n$, exists $p_n$ such that $0 < d_X(p_n, p) < 1/n$ and $d_Y(f(p_n), q) \ge \epsilon$
    which is contradict with $f(p_n) \to q$

    necessity:

    take $n \to \infty$, we got $d_X(p_n, p) < \delta$, thus $d_Y(f(p_n), q) < \epsilon$
\end{proof}

\begin{corollary}
    limit of function is unique. and 

    \begin{align*}
        \lim_{x \to p}f(x) + g(x) &= \lim_{x \to p}f(x) + \lim_{x \to p}g(x) \\
        \lim_{x \to p}f(x) g(x) &= \lim_{x \to p}f(x)  \cdot \lim_{x \to p}g(x) \\
        \lim_{x \to p}f(x)/g(x) &= \lim_{x \to p}f(x)  / \lim_{x \to p}g(x) \\
    \end{align*}
\end{corollary}

\begin{proof}
    by laws of limit of sequences
\end{proof}

\begin{definition}[Upper and Lower Limit of Function]
    let $X$ be metric spaces, suppose $E \subseteq X$, $f$ maps $E$ to $\R$. 
    And $p$ is a limit point of $E$, We define 

    \[
        \varlimsup_{x \to p}f(x) = \lim_{r \to 0} \left( \sup f(B(p, r) \cap E \setminus \{ p \}) \right)
    \]

    as upper limit of function $f$ at point $p$. Similar for lower limit:

    \[
        \varliminf_{x \to p}f(x) = \lim_{r \to 0} \left(\inf f(B(p, r) \cap E \setminus \{ p \}) \right)
    \]
\end{definition}

\begin{thm}
    upper limit and lower limit of function exists in extended real number.

    And

    \[
        \varliminf_{x \to p}f(x) \le \varlimsup_{x \to p}f(x)
    \]
\end{thm}

\begin{proof}
    define $h: \R^+ \to \R$:

    \[
        h(r) = \sup f(B(p, r) \cap E \setminus \{ p \}
    \]

    then $h$ is monotonically increasing, by property of monotone function. we got:

    \[
        \lim_{r \to 0}h(r) = \inf_{r > 0} h(r) \in \R^*
    \]

    Similarly conclusion holds for lower limit, if we define $g(r)$ as

    
    \[
        g(r) = \inf f(B(p, r) \cap E \setminus \{ p \}
    \]

    And by $g(r) \le h(r)$, we got

    \[
        \varliminf_{x \to p}f(x) \le \varlimsup_{x \to p}f(x)
    \]
\end{proof}

\begin{thm}
    \label{thm:limsup-of-mapped-seq-le-limsup-of-func}
    let $X$ be metric spaces, suppose $E \subseteq X$, $f$ maps $E$ to $\R$. 
    And $p$ is a limit point of $E$, and $x_n \to p,\: x_n \ne p$, Then

    \[
        \varlimsup_{n \to \infty} f(x_n) \le \varlimsup_{x \to p}f(x)
    \]

    and


    \[
        \varliminf_{n \to \infty} f(x_n) \ge \varliminf_{x \to p}f(x)
    \]
\end{thm}

\begin{proof}
    define $h: \R^+ \to \R$:

    \begin{align*}
        h(r) &= \sup f(B(p, r) \cap E \setminus \{ p \}) \\
        L &= \lim_{r \to 0, r > 0} h(r) = \varlimsup_{x \to p}f(x)
    \end{align*}

    and

    \[
        r(x_n) = d(p, x_n) + \frac{1}{n}
    \]

    By $x_n \to p$ and $h(r(x_n)) \ge f(x_n)$. then

    \[
        \varlimsup_{n \to \infty} f(x_n) \le \varlimsup_{n \to \infty}h(r(x_n))
    \]

    since $r(x_n) \to 0$ and $h$ has limit $L$ at $0$, so

    \[
        \lim_{n \to \infty}h(r(x_n)) = L = \varlimsup_{n \to \infty}h(r(x_n))
    \]

    and

    \[
        \varlimsup_{n \to \infty} f(x_n) \le L
    \]

    Similarly for $\varliminf_{n \to \infty}f(x_n)$
\end{proof}

\begin{thm}
    let $X$ be metric spaces, suppose $E \subseteq X$, $f$ maps $E$ to $\R$. 
    And $p$ is a limit point of $E$, If

    \[
        \varlimsup_{x \to p}f(x) =  \varliminf_{x \to p}f(x)
    \]

    Then

    \[
        \lim_{x \to p}f(x)= \varlimsup_{x \to p}f(x) = \varliminf_{x \to p}f(x)
    \]
\end{thm}

\begin{proof}
    define:

    \[
        \varlimsup_{x \to p}f(x) =  \varliminf_{x \to p}f(x) = L
    \]

    Take arbitrary $x_n \to p$, by \autoref{thm:limsup-of-mapped-seq-le-limsup-of-func}, we got:
    
    \[
       L \le \varliminf_{n \to \infty}f(x_n) \le  \varlimsup_{n \to \infty}f(x_n) \le L
    \]

    which indicates:

    \[
        \lim_{n \to \infty}f(x_n) = L
    \]

    since $x_n$ is arbitrary, by \autoref{thm:limit-of-func-iff-limit-of-every-mapping-of-convergent-seq},

    \[
        \lim_{x \to p}f(x) = L
    \]
\end{proof}

\subsection{Continuous}

\begin{definition}[Continuity at Point]
    $(X,d_X)$ and $(Y,d_Y)$  are metric spaces. $E \subseteq X$ and $p \in E$.

    and $f:E \to Y$, Then $f$ is said to be continuous at $p$ iff $\forall \epsilon > 0$, 
    there exists $\delta > 0$ such that

    \[
        d_Y(f(x), f(p)) < \epsilon \quad \forall x \in \{x \in E: d_X(x,p) < \delta \}
    \]
\end{definition}

\begin{corollary}
    if $p$ is limit point of $E$, then $f$ is continuous at $p$ iff

    \[
        \lim_{x \to p}f(x) = f(p)
    \]
\end{corollary}

\begin{thm}
    if $f: E \to Y$, where $E \subseteq X$, and $(X, d_X), (Y, d_Y)$ are both metric space. 
    And $p \in E$

    Then $f$ is continuous at $p$ iff one of below condition meets:

    \begin{enumerate}
        \item $p$ is not a limit point of $E$

        \item $\lim_{x \to p}f(x) = f(p)$
    \end{enumerate}
\end{thm}

\begin{proof}
    sufficiency:

    if $p$ is not a limit point of $E$, then there exists $\delta$ and $B(p, \delta) \cap E = \{ p \}$.
    Then $\delta$ is what we want, for continuity at $p$

    if $f(x) \to f(p)$ as $x \to p$, then $f$ is continuous at $p$ definition of continuity 

    necessity:

    it is easily to prove by discuss on $p \in E'$
\end{proof}

\begin{thm}
    \label{thm:4-2-4}
    if $f: E \to Y$, where $E \subseteq X$, and $(X, d_X), (Y, d_Y)$ are both metric space. 
    And $p \in E$

    Then $f$ is continuous at $p$ iff every sequence $p_n \in E, p_n \to p$ we have $f(p_n) \to f(p)$
\end{thm}

\begin{proof}
    it is easily to prove by discuss on $p \in E'$
\end{proof}

\begin{thm}
    suppose $X, Y,Z$ are metric space, and $E \subseteq X$, $f: E \to Y$, $g: F \to Z$, where $f(E) \subseteq F$

    then $h =  g \circ f$ is continuous at $p$, if $f$ is continuous at $p$ and $g$ is continuous at $f(p)$
\end{thm}

\begin{proof}
    pick arbitrary $p_n \to p,  p_n \in E$, by \autoref{thm:4-2-4}, we got $f(p_n) \to f(p)$,
    since $g$ is continuous at $f(p)$, we got $g(f(p_n)) \to g(f(p))$, after all, we got
    $h(p_n) \to h(p)$, since $p_n$ is arbitrary, so $h$ is continuous at $p$
\end{proof}

\begin{thm}
    A mapping $f$ of a metric space $X$ to metric space $Y$ is continuous on $X$
    iff $f^{-1}(V)$ is open in $X$ for every open set $V$ in $Y$
\end{thm}

\begin{proof}
    sufficiency:

    pick $p \in X, p_n \to p$, since for any $\epsilon > 0$, $f^{-1}(B_Y(f(p), \epsilon))$ is open.
    and $p \in f^{-1}(B_Y(f(p), \epsilon))$, we can pick $N$ large enough so that $\forall n,\: p_n \in f^{-1}(B_Y(f(p), \epsilon))$
    which indicates $d_Y(f(p_n), f(p)) < \epsilon $, and thus $f(p_n) \to p$

    necessity:

    we pick $V \in Y$, and $p \in f^{-1}(V)$, we will prove $p$ is interior point.
    since $f(p) \in V$ and $V$ is open, there exists $\epsilon > 0$ so that $B(f(p), \epsilon) \subseteq V$

    by definition of continuous, there should exists $\delta > 0$, so that $f(B(p, \delta)) \subseteq B(f(p), \epsilon) \subseteq V$

    thus we got:

    \[
        B(p, \delta) \subseteq f^{-1}(f(B(p, \delta))) \subseteq f^{-1}(V)
    \]

    so $p$ is interior point of $f^{-1}(V)$
\end{proof}

\begin{thm}
    let $f,g: X \to Y$ are both complex continuous mapping from metric space $X$ to metric space $Y$,
    then $f+g, \: fg \: f/g$ are continuous on $X$
\end{thm}

\begin{proof}
    by laws of limit and \autoref{thm:4-2-4}
\end{proof}

\begin{thm}
    let $f_1,f_2,...,f_k: X \to \R$ be real functions, and $\mathbf{f}: X \to \R^k$ defined by: 

    \[
        \mathbf{f}(\vx) = [f_1(\vx), f_2(\vx),...,f_k(\vx)]
    \]

    then $\mathbf{f}$ is continuous iff each of $f_1,f_2,...,f_k$  is continuous
\end{thm}

\begin{proof}
    sufficiency:
    
    by \autoref{thm:4-2-4} and \autoref{thm:3-1-1}

    necessity:


    by \autoref{thm:3-1-1} and \autoref{thm:4-2-4}
\end{proof}

\begin{thm}
    if $\mathbf{f}, \mathbf{g}: X \to \R^k$ both be continuous, then $\mathbf{f} +  \mathbf{g}, \: \langle \mathbf{f}, \mathbf{g} \rangle$
    are both continuous
\end{thm}

\begin{proof}
    by \autoref{thm:4-2-4} and \autoref{thm:3-1-1}
\end{proof}

\subsection{Continuity and Compactness}

\begin{definition}
    $f: E \to \R^k$ is said to be bounded if exists real number $M$ such that
    $\| f(x)\| \le M,\quad \forall x \in E$
\end{definition}

\begin{thm}
    \label{thm:compact-image}
    suppose $f$ is a continuous mapping from a compact metric space $X$
    into a metric space $Y$, then $f(X)$ is compact
\end{thm}

\begin{proof}
    \begin{align*}
        f(X) &\subseteq \bigcup_{\alpha \in I}V_{\alpha} \\
        X & \subseteq f^{-1}(f(X)) \subseteq \bigcup_{\alpha \in I}f^{-1}(V_{\alpha})
    \end{align*}

    since $f^{-1}(V_{\alpha})$ is open and $X$ is compact, there exists finite sub cover:

    \begin{align*}
        X  &\subseteq \bigcup_{n=1}^{N}f^{-1}(V_{n}) \\
        f(X) &\subseteq f(\bigcup_{n=1}^{N}f^{-1}(V_{n})) \subseteq \bigcup_{n=1}^{N}f(f^{-1}(V_{n})) \\
        & \subseteq \bigcup_{n=1}^{N}V_{n}
    \end{align*}
\end{proof}

\begin{corollary}
    if $\mathbf{f}: X \to \R^k$ is a continuous mapping from a compact metric space, 
    then $f(X)$ is closed and bounded.
\end{corollary}

\begin{corollary}
    if $\mathbf{f}: X \to \R$ is a continuous mapping from a compact metric space, 

    and $M = \sup f(X), m = \inf f(X)$,  then $M \in f(X), m \in f(X)$
\end{corollary}

\begin{thm}
    let $f$ is a continuous and bijective mapping from a compact metric space $X$ 
    to a metric space $Y$. Then the inverse mapping $f^{-1}: Y \to X$ is also a continuous mapping.
\end{thm}

\begin{proof}
    assume $V$ is closed, since $V \subseteq X$ and $X$ is compact, then $V$ is also compact by \autoref{thm:2-3-4}.
    by \autoref{thm:compact-image}, $f(V)$ is compact and hence closed. let's define $g = f^{-1}$, then we got

    $g^{-1}(V)$ is closed for every closed set in $X$, therefore $g$ is continuous
\end{proof}


\begin{definition}
    let $f$ be a mapping from metric space $X$ to metric space $Y$, we way $f$ is uniformly continuous on $X$,
    iff $\forall \epsilon > 0, \exists \delta, \forall p,q \in X, d_X(p,q) < \delta$, we got $d_Y(f(p),f(q)) < \epsilon$
\end{definition}


\begin{thm}
    let $f$ be a continuous mapping from a compact metric space $X$ into a metric space $Y$.
    Then $f$ is uniformly continuous on $X$
\end{thm}

\begin{proof}
    pick $\epsilon > 0$,since $f$ is continuous and $X$ could be covered by:

    \[
        X \subseteq \bigcup_{x \in X}B(x, \delta_x/2) \quad f(B(x, \delta_x)) \subseteq B(f(x), \epsilon)
    \]

    by $X$ is compact, there exists a finite sub cover:

    \[
        X \subseteq \bigcup_{n=1}^{N}B(x_n, \delta_n/2) \quad f(B(x_n, \delta_n)) \subseteq B(f(x_n), \epsilon)
    \]

    now let's pick 

    \[
        \delta = \frac{1}{2}\min_{n=1}^{N} \delta_n
    \]

    consider any $p,q \in X$ and $d(p,q) < \delta$, since there exists $m \le N$ such that $d(x_m, p) < \delta_n / 2$.
    and by $d(p,q) < \delta < \delta_n / 2$ and triangle inequality we got $d(q, x_m) < \delta_n$ 

    which indicates that $f(p), f(q) \in B(f(x_m), \epsilon)$, by triangle infinitely, we got:

    $d(f(p), f(q)) < 2\epsilon$

    since $\epsilon$ is arbitrary, so $f$ is uniformly continuous.
\end{proof}

\begin{thm}
    let $E$ be a noncompact set in $\R$, Then

    \begin{enumerate}
        \item exists a continuous function on $E$ which is not bounded

        \item there exists a continuous and bounded function on $E$ which has no maximum



        \item  If, in addition, $E$ is bounded, then exists a continuous function on $E$ which is not uniformly continuous
    \end{enumerate}

    
\end{thm}

\begin{proof}
    \begin{enumerate}
        \item for (1), we discuss on $E$. Since $E$ cannot be both closed and bounded.

        Assume $E$ is not closed, pick limit point $p$ of $E$, while $p \notin E$, define 

        \[
            f(x) = \frac{1}{x-p}
        \]

        then $f$ is not bounded


        Assume $E$ is not bounded, define $f(x) = x$, then $f$ is not bounded


        \item for (2), if $E$ is not closed, assume $p \in E' \setminus E$ define $g$ as, 

        \[
            g(x) = \frac{1}{1+ (x-p)^2}
        \]

        if $E$ is not bounded, define $g$ as

        \[
            g(x) = \arctan \lvert x \rvert
        \]

        \item for (3), since $E$ is bounded, then $E$ cannot be closed, assume $p \in E' \setminus E$, define $f$ as:

        \[
            f(x) = \frac{1}{x-p}
        \]

        then $f$ is not uniformly continuous since, define $t(x) = x-p$

        \begin{align*}
            \lvert f(x+\delta) -f(x) \rvert &= \left| \frac{1}{t+\delta} - \frac{1}{t} \right| \\
            &= \left|\frac{\delta}{t(t+\delta)} \right|
        \end{align*}

        when $t(x) = x-p $ meets $\lvert t \rvert < \delta / N$, $\lvert f(x+\delta) -f(x) \rvert > N^2/((N+1)\delta)$,
        since $N$ can be arbitrary large, so $\lvert f(x+\delta) -f(x) \rvert$ can be arbitrary large, while $x+\delta, x$ are very close.
        So $f$ is not uniformly continuous.
    \end{enumerate}
\end{proof}

\begin{thm}
    If $f$ is a continuous mapping of a metric space $X$ into a metric space $Y$
    , and if $E$ is a connected subset of $X$, then $f(E)$ is connected.
\end{thm}

\begin{proof}
    assume $f(E)$ is not connected, and let $f(E) = A \cup B$ where $A, B$ is separated and both not empty

    and we define

    \begin{align*}
        G &= E \cap f^{-1}(A) \\
        H &= E \cap f^{-1}(B) \\
    \end{align*}

    then we have

    \begin{align*}
        G \cup H &= E \cap (f^{-1}(A) \cup f^{-1}(B))  \\
        &= E \cap f^{-1}(A \cup B) = E \cap f^{-1}(f(E)) \\
        & = E
    \end{align*}

    so $G \cup H = E$

    and both $G, H$ is not empty:

    for $G$ we have, since $A$ is not empty, there exists $y \in A$:

    \begin{align*}
        \{y \} & \subseteq A \subseteq f(E) \\
        f^{-1}(\{ y \}) & \cap E \subseteq f^{-1}(A) \cap E
    \end{align*}

    since $y \in f(E)$ there should exists $x \in E$ such that $f(x) = y$,so

    \[
        \{ x \} \subseteq f^{-1}(A) \cap E
    \]


    and we have:

    \begin{align*}
        G &\subseteq f^{-1}(A) \subseteq f^{-1}(\overline{A}) \\
        \overline{G} & \subseteq f^{-1}(\overline{A}) \\
        f(\overline{G}) & \subseteq f(f^{-1}(\overline{A})) \subseteq \overline{A} \\
        f(H) & \subseteq f(f^{-1}(B)) \subseteq B \\
        f(\overline{G} \cap H) & \subseteq f(\overline{G}) \cap f(H) = \emptyset
    \end{align*}

    then we have $\overline{G} \cap H = \emptyset$, otherwise we have at least one point $x \in E$ and

    \[
        f(\{ x \}) = \{ f(x)\} \subseteq f(\overline{G} \cap H)
    \]

    similarly we can prove $G \cap \overline{H} = \emptyset$, 
\end{proof}

\begin{corollary}
    let $f$ be a continuous real function on the interval $[a,b]$, if $f(a) < f(b)$ and $c$
    is a number such that $f(a) < c < f(b)$,then there exists a point $x \in (a,b)$ such that 
    $f(x) = c$
\end{corollary}

\begin{proof}
    let $I = [a,b]$, since $I$ is connected and $f$ is continuous, so $f(I)$ is also connected.
    by $f(a), f(b) \in f(I)$ and $f(a) < c < f(b)$, and $f(I)$ is connected, 
    so $c \in f(I)$
\end{proof}

\begin{thm}
    let $f$ be monotonically increasing on $(a,b)$. Then $f(x+)$ and $f(x-)$ 
    exist at every point $x $ of $(a,b)$. More precisely

    \begin{enumerate}
        \item $f(x-) = \sup_{a < t <x}f(t)$
        \item $f(x+) = \inf_{x < t <b}f(t)$
        \item $f(x-) \le f(x) \le f(x+)$
    \end{enumerate}

    Furthermore, if $a < x < y < b$, then

    \[
        f(x+) \le f(y-)
    \]
\end{thm}

\begin{proof}
    we will prove:

    \[
        f(x-) = \sup_{a < t <x}f(t)
    \]

    and define

    \[
        L = \sup_{a < t <x}f(t)
    \]

    
    pick $\epsilon > 0$, by definition of sup there should 
    exists $x' \in (a,x)$ so that $f(x') \ge L - \epsilon$, now let's define $\delta = x -x'$

    for any $t \in B(x, \delta) \cap (a,x)$ which is $x' < t < x$

    \begin{align*}
        \left| f(t) - L \right| &= L - f(t) \le L - f(x') \\
        & \le \epsilon
    \end{align*}

    similarly we can prove 


    \[
        f(x+) = \inf_{x < t <b}f(t)
    \]

    $f(x-) \le f(x)$ because $f(x)$ is upper bound of $\{ f(t): a<t<x \}$

    and


    $f(x) \le f(x+)$ because $f(x)$ is lower bound of $\{ f(t): x<t<b \}$

    if $x < y$, pick $x < c < y$, Then:

    \begin{align}
        f(x^+) \le f(c-) \le f(c) \le f(c+) \le f(y-)
    \end{align}

    consider sup of superset, and inf of superset

\end{proof}