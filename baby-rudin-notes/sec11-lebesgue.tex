\section{The Lebesgue Theory}

\subsection{Outer Measure}

\begin{definition}[Open box]
    An open box $B$ in $\mathbb{R}^n$ is any set of the form:
    
    \[
        B = \prod_{i=1}^{n}(a_i, b_i) = \{(x_1,x_2,...,x_n): x_i \in (a_i, b_i) \}
    \]

    where $a_i \le b_i$ is real number.

    And we define volume of $B$ as:

    \[
        \mathrm{vol}(B) = \prod_{i=1}^{n}(b_i - a_i)
    \]

    for empty set we define $\mathrm{vol}(\emptyset) = 0$, and if interior of $A$ is an open box,

    we define $\mathrm{vol}(A) = \mathrm{vol}(\mathrm{int}(A))$
\end{definition}


\begin{definition}[Outer measure]
    If $\Omega \subseteq \mathbb{R}^n$ is a set, we define the outer measure $m^*(\Omega)$ 
    as:

    \[
        m^*(\Omega) = \inf \{ \sum_{j \in J} \mathrm{vol}(B_j): \Omega \subseteq \bigcup_{j \in J}B_j\}
    \]

    where $J$ is at most countable
\end{definition}

\begin{thm}
    properties of outer measure

    \begin{enumerate}
        \item outer measure of empty set is $0$.

        \item positivity $0 \le m^*(\Omega) \le \infty$

        \item If $A \subseteq B$, then $m^*(A) \le m^*(B)$

        \item finite sub-additivity

        \[
            m^*(\bigcup_{j \in J}A_j) \le \sum_{j \in J}m^*(A_j)
        \]

        where $J$ is a finite set

        \item countable sub-additivity

        \[
            m^*(\bigcup_{j \in J}A_j) \le \sum_{j \in J}m^*(A_j)
        \]

        where $J$ is at most countable

        \item translation invariance

        \[
            m^*(x + \Omega) = m^*(\Omega)
        \]
    \end{enumerate}
\end{thm}

\begin{proof}
    proofs:

    \begin{enumerate}
        \item $m^*(\emptyset) = 0$
        
        by

        \[
            \emptyset \subseteq \prod_{i=1}^{n}(0,\epsilon)
        \]

        thus

        \[
            m^*(\emptyset) \le \epsilon^n
        \]

        take $\epsilon \to 0$

        \item $0 \le m^*(\Omega) \le \infty$

        since summation of volume is non-negative, thus its limit point should be non-negative.

        \item monotonicity

        assume any set of open box covers $B$, it also covers $A$, define the summation as $S(J)$,
        where $J$ contains open box covers $B$

        \[
            m^*(A) \le S(J)
        \]

        then take $S(J) \to m^*(B)$

        \item we will prove $m^*(A \cup B) \le m^*(A) + m^*(B)$

        assume $J_A$ and $J_B$ contains open box covers $A$ and $B$, it is easily to verify that
        $J_A \cup J_B$ covers $A \cup B$

        thus we have:

        \[
            m^*(A\cup B)\le S(J_A \cup J_B) \le S(J_A) + S(J_B)
        \]

        and take $S(J_A) \to m^*(A)$ and $S(J_B) \to m^*(B)$

        then we can prove finite case by induction

        \item countable sub-additivity

        let $J_1, J_2, ...$ contains open box which covers $A_1, A_2, ...$ respectively.

        And we have
        
        \[
        S(J_i) \le m^*(A_i) + \frac{1}{2^i} \epsilon
        \]

        
        now $J_1 \cup J_2 \cup ... $ covers $A_1 \cup A_2 ...$, thus

        \begin{align*}
            m^*\left(\bigcup_{i=1}^{\infty}A_i \right) \le S\left(\bigcup_{i=1}^{\infty} J_i \right)
        \end{align*}

        since 

        \[
            J = \bigcup_{i=1}^{\infty} J_i
        \]

        is countable and non negative, consider its finite sum:

        \[
            \mathrm{vol}(B_1) +\mathrm{vol}(B_2) + ... + \mathrm{vol}(B_n) \le \sum_{i=1}^{\infty}S(J_i)
        \]

        take side left to $S(J)$, thus we got

        \[
            S(J) \le \sum_{i=1}^{\infty}S(J_i)
        \]

        and


        \begin{align*}
            m^*\left(\bigcup_{i=1}^{\infty}A_i \right) & \le S(J) \le \sum_{i=1}^{\infty}S(J_i) \\
            & \le \sum_{i=1}^{\infty}m^*(A_i) + \epsilon
        \end{align*}

        since $\epsilon$ is arbitrary, we have

        \[
m^*\left(\bigcup_{i=1}^{\infty}A_i \right) \le \sum_{i=1}^{\infty}m^*(A_i)
        \]
    \end{enumerate}
\end{proof}

\begin{thm}
    The outer measure of closed box is 

    let
    \[
        \overline{B} = \prod_{i=1}^{n}[a_i,b_i]
    \]

    Then

    \[
        m^*(\overline{B}) = \prod_{i=1}^{n}(b_i - a_i)
    \]
\end{thm}

\begin{proof}

    we will prove a lemma at first, that is, if $\overline{B}$ is covered by
    finite open box $B_1, B_2, ..., B_N$.

    Then

    \[
    \mathrm{vol}(\overline{B}) \le \sum_{i=1}^{N} \mathrm{vol}(B_i) 
    \]

    we prove this by use induction on dimension $n$.

    when $n=1$, $\overline{B}$ becomes an interval, let define it as $[a,b]$.

    And $B_1,B_2,...,B_N$ becomes open interval, we can express those open
    box as function $\chi_{B_1},\chi_{B_2},...$, and thus

    by $[a,b]$ covered by $B_1 \cup B_2 \cup ... \cup B_N$, we have

    \[
        \chi_{[a,b]} \le \sum_{i=1}^{N} \chi_{B_i}
    \]

    and then, we take integration on both side, since they are simple functions.

    \begin{align*}
        \mathrm{vol}(\overline{B}) &= \int_{\mathbb{R}} \chi_{[a,b]} \\
        & \le \int_{\mathbb{R}} \sum_{i=1}^{N} \chi_{B_i} \le  \sum_{i=1}^{N} \int_{\mathbb{R}} \chi_{B_i} \\
        & \le \sum_{i=1}^{N} \mathrm{vol}(B_i)
    \end{align*}


    when $n=k+1$, we will define a set function $P_t$ as:

    \[
        P_t(U) = U \cap \{ (x_1,x_2,...,x_{k+1}) \in \mathbb{R}^{k+1}: x_{k+1} = t\}
    \]

    it is easily to verify $P$ is monotone, and maps from an open or closed box in $\mathbb{R}^{k+1}$ to an open or closed box or empty set in $\mathbb{R}^k$, and we have $P_t(U \cup V) = P_t(U) \cup P_t(V)$

    Thus, if a closed box be covered by finite open box in $\mathbb{R}^{k+1}$:

    \[
        \overline{B} \subseteq \bigcup_{i=1}^{N} B_i
    \]

    then we apply $P_t$ on both side:


    \[
        P_t(\overline{B}) \subseteq P_t\left(\bigcup_{i=1}^{N} B_i\right) \subseteq \bigcup_{i=1}^{N}P_t(B_i)
    \]

    by hypothesis of induction we have:

    \begin{align*}
        \mathrm{vol}(P_t(\overline{B})) &\le \sum_{i=1}^{N} \mathrm{vol}(P_t(B_i))
    \end{align*}

    The above inequality holds when $t$ is arbitrary.

    thus $\mathrm{vol}(P_t(B_i))$ and $\mathrm{vol}(P_t(\overline{B}))$ becomes piecewise constant function of $t$, so we can take integration at both side:

    \begin{align*}
        \int_{\mathbb{R}}\mathrm{vol}(P_t(\overline{B})) \mathrm{d}t  &\le \int_{\mathbb{R}}\sum_{i=1}^{N} \mathrm{vol}(P_t(B_i)) \mathrm{d}t \\
        & \le \sum_{i=1}^{N} \int_{\mathbb{R}}\mathrm{vol}(P_t(B_i)) \mathrm{d}t \\
        & \le \sum_{i=1}^{N}\mathrm{vol}(B_i)
    \end{align*}

    by

    \[
\int_{\mathbb{R}}\mathrm{vol}(P_t(\overline{B})) \mathrm{d}t = \mathrm{vol}(\overline{B})
    \]

    we proved our lemma.


    Since closed box is compact, which means every open cover contains a finite sub cover.
    
    Let $J$ be any open cover of $\overline{B}$ which contains open box, and $S(J)$ be summation of volume.

    And $J'$ be finite sub cover of $J$, thus we have

    \[
        \mathrm{vol}(\overline{B}) \le S(J') \le S(J)
    \]

    take $S(J) \to m^*(\overline{B})$, then we have

    \[
        \mathrm{vol}(\overline{B}) \le m^*(\overline{B})
    \]

    on the other hand, we can cover $\overline{B}$ by any open box which extend 
    its interval on each dimension for $\epsilon$ longer.

    thus we have

    \[
        m^*(\overline{B}) \le \mathrm{vol}(\overline{B}) + M\epsilon
    \]

    where $M$ is an upper bound which is only related with $n$ and $\mathrm{vol}(\overline{B})$

    take $\epsilon \to 0$, we got


    \[
        m^*(\overline{B}) \le \mathrm{vol}(\overline{B})
    \]

    after all, we proved that

    \[
        m^*(\overline{B}) = \mathrm{vol}(\overline{B})
    \]
\end{proof}

\begin{corollary}
    The outer measure of open box $B$ is its volume.
\end{corollary}

\begin{proof}
    by $m^*(B) \le m^*(\overline{B})$ and use $\overline{B} \subseteq B$, and 
    let $\overline{B}$ approximate $B$.
\end{proof}


\subsection{Measurable Sets}

\begin{thm}
    Some operation which preserve measurable.

    if $E,E_1, E_2\subseteq \mathbb{R}^n$ are measurable then:

    \begin{enumerate}
        \item $E + x$ is measurable.

        \item $E^C$ is measurable.

        \item $E_1 \cap E_2$ is measurable.

        \item $E_1 \cup E_2$ is measurable.

        \item if $m(E_1 \cap E_2) = 0$, then

        \[
            m(E_1 \cup E_2) = m(E_1) + m(E_2)
        \]

        \item if $E_n$ is countable measurable set, Then

        \[
            \bigcup_{n=1}^{\infty}E_n
        \]

        and


        \[
            \bigcap_{n=1}^{\infty}E_n
        \]

        is measurable.

        also, if $E_n$ is disjoint. we have

        \[
            m(\bigcup_{n=1}^{\infty}E_n) = \sum_{n=1}^{\infty}m(E_n)
        \]
    \end{enumerate}
\end{thm}

\begin{proof}
    \begin{enumerate}
        \item $E+x$ is measurable

        for any $A$, consider

        \begin{align}
            m^*(A \cap (E+x)) + m^*(A^C \cap (E+x)) &= m^*(\left(A -x\right)\cap E) + x) + m^*(((A^C-x) \cap E) + x) \\
            &= m^*(\left(A -x\right)\cap E) + m^*((A^C-x) \cap E) \\
            &= m^*(A-x)
        \end{align}

        since $A$ is arbitrary, we can prove by replace $A$ with $A +x$.

        \item $E^C$ is measurable

        by symmetric

        \item $E_1 \cap E_2$ is measurable

        \begin{align*}
            m^*(A \cap E_1) & = m^*(A \cap E_1\cap E_2) + m^*(A \cap E_1 \cap E_2^{C}) \\ 
            m^*(A \cap E_1^C) & = m^*(A \cap E_1^C \cap E_2) + m^*(A \cap E_1^{C} \cap E_2^C) \\ 
        \end{align*}

by $E_1$ is measurable, consider

\[
    (E_1 \cap E_2)^C = E_1 \setminus E_2 \cup E_2 \setminus E_1 \cup E_1^C \cap E_2^C
\]

\begin{align*}
m^*(A) &= m^*(A \cap E_1 \cap E_2) + m^*(A \cap E_1 \cap E_2^{C}) + m^*(A \cap E_1^{C} \cap E_2) + m^*(A \cap E_1^{C} \cap E_2^C) \\
& \ge m^*(A \cap E_1 \cap E_2) + m^*(A \cap (E_1 \cap E_2)^C) 
\end{align*}

so $E_1 \cap E_2$ is measurable.

    \item $E_1 \cup E_2$ is measurable
    
    since $\cup$ is combination of complement and intersection.


    \item if $m(E_1 \cap E_2) = \emptyset$, then

    let $A$ be arbitrary set, we will prove

    \[
        m^*\left( A \cap \left( E_1 \cup E_2\right) \right) = m^*(A \cap E_1) + m^*(A \cap E_2)
    \]

    define $X = E_1 \cup E_2$, by $E_2$ is measurable:

    \begin{align}
        m^*(A \cap X) &= m^*( A \cap X \cap E_2) + m^*( (A \cap X) \cap E_2^C) \\
        &= m^*(A \cap E_2)  + m^*(A \cap E_1)
    \end{align}

    also we can define 

    \[
        X = \left(\bigcup_{k=1}^{n-1}E_k \right) \cup E_n
    \]

    prove

    \[
        m^*\left(A \cap \bigcup_{k=1}^{n}E_k\right) = \sum_{k=1}^{n}m^*(A \cap E_k)
    \]

    by induction.

    by define $A = \mathbb{R}^n$, we can prove that

    \[
        m\left(\bigcup_{k=1}^{n}E_k\right) = \sum_{k=1}^{n} m(E_k)
    \]

    \item we will prove, for any $A \subseteq \mathbb{R}^n$,

    let 

    \[
        E = \bigcup_{n=1}^{\infty}E_n
    \]
    
    \[
        m^*(A) = m^*(A \cap E) + m^*(A \cap E^C)
    \]

    let's define 

    \[
        F_N = \bigcup_{n=1}^{N}E_n
    \]

    consider

    \[
        A \cap E = A \cap \bigcup_{n=1}^{\infty}E_n  = \bigcup_{n=1}^{\infty} A \cap E_n
    \]

    and

    \[
        m^*\left(A \cap F_N \right) = m^*\left(\bigcup_{n=1}^{N} A \cap E_n \right) = \sum_{n=1}^{N}m^*(A \cap E_n)
    \]

    which means

    \[
        \lim_{N \to \infty}m^*(A \cap F_N) = \sum_{n=1}^{\infty}m^*(A \cap E_n) \ge m^*(A \cap E)
    \]

    and

    \[
        m^*(A \setminus F_N) \ge m^*(A \setminus E)
    \]

    also we have

    \[
        m^*(A) = m^*(A \cap F_N) + m^*(A \setminus F_N)
    \]

    take $N \to \infty$, we got

    \[
        m^*(A) \ge m^*(A \cap E) + m^*(A \setminus E)
    \]

    so $E$ is measurable.

    and by

    \begin{align*}
        \sum_{n=1}^{N}E_n = m^*(F_N) \le m^*(E)
    \end{align*}

    take $N \to \infty$, we got

    \begin{align*}
        \sum_{n=1}^{\infty}E_n  \le m^*(E)
    \end{align*}

    after all, we got

    \[
        m^*(E) = \sum_{n=1}^{\infty}E_n
    \]

    \end{enumerate}  
\end{proof}

\begin{thm}
    let $E_{n+1} \supseteq E_{n}$ and $E_n$ be measurable, Then 

    \[
        m\left(\bigcup_{n=1}^{\infty}E_n \right) = \lim_{n \to \infty}m(E_n)
    \]
\end{thm}

\begin{proof}
    define 

    \begin{align*}
        F_1 &= E_1 \\
        F_{n+1} &= E_{n+1} \setminus E_{n}
    \end{align*}

    Then $F_n$ be disjoint. Thus we can apply countable additivity:

    \begin{align*}
       m\left(\bigcup_{n=1}^{\infty}E_n\right) &= m\left(\bigcup_{n=1}^{\infty}F_n\right) = \sum_{n=1}^{\infty}m(F_n) \\
       &= m(E_1) + \sum_{n=2}^{\infty}m(E_{n} \setminus E_{n-1}) \\
       &= m(E_1) + \sum_{n=2}^{\infty}\left(m(E_{n}) - m(E_{n-1}) \right)\\
       &= \lim_{n \to \infty}m(E_n)
    \end{align*}

    We use $m(E_{n}) - m(E_{n-1})$ here because  $\forall n, \:m(E_n) < \infty$, otherwise the conclusion is obviously.
\end{proof}

\begin{corollary}
    let $E_{n+1} \subseteq E_{n}$ and $E_n$ be measurable, where $m(E_1) < \infty$,
    Then

    \[
        m\left(\bigcap_{n=1}^{\infty}E_n \right) = \lim_{n \to \infty}m(E_n)
    \]
\end{corollary}

\begin{proof}
    we define $F_n$ as:

    \begin{align*}
        F_1 &= \emptyset \\
        F_n &= E_1 \setminus E_n \\
    \end{align*}

    thus $F_n$ becomes increasing, so we can apply previous theorem

    \[
        m\left(\bigcup_{n=1}^{\infty}F_n\right) = \lim_{n\to \infty}m(F_n)
    \]

    and

    \begin{align*}
        m\left(\bigcup_{n=1}^{\infty}F_n\right) &= m\left(\bigcup_{n=1}^{\infty}E_1 \setminus E_n \right) \\
        &= m\left(E_1 \cap \bigcup_{n=1}^{\infty} E_n^C \right) = m\left(E_1 \setminus \bigcap_{n=1}^{\infty} E_n \right) \\
        &= m(E_1) - m\left(\bigcap_{n=1}^{\infty} E_n \right) \\
        \lim_{n \to \infty}m(F_n) &= \lim_{n \to \infty}m(E_1 \setminus E_n) = \lim_{n \to \infty}m(E_1)- m(E_n) \\
        &= m(E_1) - \lim_{n \to \infty}m(E_n)
    \end{align*}
\end{proof}

\subsection{Measurable Function}

\subsection{simple functions}

\begin{definition}
    let $\Omega$ be a measurable subset of $\mathbb{R}^n$, and $f: \Omega \to \mathbb{R}$
    be a measurable function. We say $f$ is a simple function if its image $f(\Omega)$ is finite.
\end{definition}

\begin{thm}
    A measurable function is simple iff it could be represents as finite linear combination of characteristic
    function of measurable set.
\end{thm}

\begin{proof}
    Sufficiency is obviously, since addition of finite measurable function is measurable. And image is also finite.

    If a measurable function is simple, assume $f(\Omega) = \{ c_1,c_2,...,c_n\}$, and 

    $f^{-1}(\{ c_1\}) = E_1, f^{-1}(\{ c_2\}) = E_2, ... , f^{-1}(\{ c_n\}) = E_n$, Then

    \[
        f = \sum_{k=1}^{n} c_k \chi_{E_k}
    \]
\end{proof}

\begin{thm}
    addition of two simple function is also simple, also for multiplication.
\end{thm}

\begin{proof}
    let 

    \begin{align*}
        f &= \sum_{k=1}^{n}c_k \chi_{E_k} \\
        g &= \sum_{j=1}^{m}d_j \chi_{F_j} \\
    \end{align*}

    Then

    \begin{align*}
        f g &= \sum_{k=1}^{n}\sum_{j=1}^{m} c_kd_j \chi_{E_k \cap F_j} \\
        f + g &= \sum_{k=1}^{n} c_k \chi_{E_k} + \sum_{j=1}^{m}d_j \chi_{F_j}
    \end{align*}
\end{proof}

\begin{definition}[Lebesgue integral of simple function]
    Let $\Omega$ be a measurable subset of $\mathbb{R}^n$, and let $f: \Omega \to \mathbb{R}$ be a simple measurable function which is non-negative.
    Thus the image $f(\Omega)$ is finite and contained in $[0, \infty)$. We then define
the Lebesgue integral $\int_{\Omega}$ of $f$on $\Omega$ by

\[
    \int_{\Omega}f = \sum_{\lambda \in f(\Omega); \lambda > 0} \lambda m(\{ x \in \Omega: f(x) = \lambda \})
\]
\end{definition}

\begin{thm}
    \label{thm:7e089ab3-04aa-44e3-9ead-32b28d6b95db}
    if $f: \Omega \to \mathbb{R}$ is a simple non-negative, measurable function. And it could be expressed as linear combination of characteristic function of measurable set group $\{ F_k\}$,
    and $\{ F_k\}$ is disjoint

    \[
        f = \sum_{k=1}^{m} d_k \chi_{F_k}
    \]

    Then

    \[
        \int_{\Omega} f = \sum_{k=1}^{m}d_k m(F_k)
    \]

    Note $d_k$ may not be distinct
\end{thm}

\begin{proof}
    assume $f(\Omega) = \{ c_1,c_2,...,c_n\}$, and 

    $f^{-1}(\{ c_1\}) = E_1, f^{-1}(\{ c_2\}) = E_2, ... , f^{-1}(\{ c_n\}) = E_n$, Then

    we can group $d_k$ by $c_1,c_2,...,c_n$, because the set $\{ d: d = d_1,d_2,...,d_m\} $ is $\{c_1,c_2,...,c_n\}$ actually.

    \begin{align*}
        \sum_{k=1}^{m}d_k m(F_k) &= \sum_{j=1}^{n}\sum_{k: d_k = c_j}d_k m(F_k) \\
        &= \sum_{j=1}^{n}c_j \sum_{k: d_k = c_j}m(F_k) \\
        &= \sum_{j=1}^{n}c_j m\left(\bigcup_{k: d_k = c_j} F_k \right) \\
    \end{align*}

    consider 

    \[
        \bigcup_{k: d_k = c_j} F_k
    \]

    by $d_k = c_j$ we have $F_k \subseteq E_j$, by $c_j = d_k$, and $F_k$ is disjoint, there exists at least one 
    $d_k = c_j$, consider if $x$ in

    \[
        E_j \setminus \left(\bigcup_{k: d_k = c_j} F_k \right)
    \]

    then $f(x) = c_j$ and $f(x) \ne c_j$ both holds.

    thus

    \begin{align*}
        \sum_{k=1}^{m}d_k m(F_k) &= \sum_{j=1}^{n}c_j m\left(\bigcup_{k: d_k = c_j} F_k \right) \\
        &= \sum_{j=1}^{n}c_j m\left(E_j \right) = \int_{\Omega} f
    \end{align*}

\end{proof}

\begin{thm}
    Let $\Omega$ be a measurable set, and let $f: \Omega \to \mathbb{R}$
    and $g: \Omega \to \mathbb{R}$ be non-negative simple measurable functions.

    Then:

    \begin{enumerate}
        \item $0 \le \int_{\Omega}f \le \infty$, and $\int_{\Omega}f = 0$ iff $m(\{ x: f(x) \ne 0\}) = 0$

        \item $\int_{\Omega}(f+g) = \int_{\Omega}f  + \int_{\Omega}g $

        \item $\int_{\Omega}cf = c\int_{\Omega}f$

        \item if $ f \le g$, then $\int_{\Omega}f \le \int_{\Omega}g$

        \item if $\Omega_1 \cup \Omega_2$ is disjoint, and $\Omega_1 \cup \Omega_2 = \Omega$,then

        \[
            \int_{\Omega_1} f  + \int_{\Omega_2} f = \int_{\Omega} f
        \]
    \end{enumerate}
\end{thm}

\begin{proof}
    proofs:

    \begin{enumerate}
        \item $0 \le \int_{\Omega}f \le \infty$, and $\int_{\Omega}f = 0$ iff $m(\{ x: f(x) \ne 0\}) = 0$

        obviously by definition

        \item $\int_{\Omega}(f+g) = \int_{\Omega}f  + \int_{\Omega}g $

        \begin{align*}
            \int_{\Omega}f  + \int_{\Omega}g  &= \sum_{j=1}^{n}c_j m(E_j) + \sum_{k=1}^{s}d_k m(F_k) \\
            &= \sum_{j=1}^{n}\sum_{k=1}^{s}c_j m(E_j \cap F_k) + \sum_{k=1}^{s}\sum_{j=1}^{n}d_k m(F_k \cap E_j) \\
            &= \sum_{j=1}^{n}\sum_{k=1}^{s}\left(c_j + d_k \right) m(E_j \cap F_k)
        \end{align*}

        since 

        \[
            h = \sum_{j=1}^{n}\sum_{k=1}^{s}\left(c_j + d_k \right)\chi_{E_j \cap F_k}
        \]

        by \autoref{thm:7e089ab3-04aa-44e3-9ead-32b28d6b95db}, we proved

        \item $\int_{\Omega}cf = c\int_{\Omega}f$

        easily to prove like (2)

        \item if $ f \le g$, then $\int_{\Omega}f \le \int_{\Omega}g$

        by $h = g -f$ is non-negative

        \[
            h = \sum_{j=1}^{n}\sum_{k=1}^{s}\left(d_k -f_j\right)\chi_{E_j \cap F_k}
        \]

        thus

        \[
            \sum_{j=1}^{n}\sum_{k=1}^{s}\left(d_k - c_j \right) m(E_j \cap F_k) \ge 0
        \]

        simplify, we got

        \[
            \int_{\Omega}f \le \int_{\Omega}g
        \]

        \item if $\Omega_1 \cup \Omega_2$ is disjoint, and $\Omega_1 \cup \Omega_2 = \Omega$ Then

        assume when $x \in \Omega$

        \[
            f(x) = \sum_{j=1}^{n} c_j \chi_{E_j}(x)
        \]

        for $x \in \Omega_1$, we got


        \[
            f(x) = \sum_{j=1}^{n} c_j \chi_{E_j \cap \Omega_1}(x)
        \]

        thus

        \[
            \int_{\Omega_1} f = \sum_{j=1}^{n}c_j m(E_j \cap \Omega_1)
        \]

        similarly for $\Omega_2$, thus we got

        \begin{align*}
            \int_{\Omega_1} f + \int_{\Omega_2} f&= \sum_{j=1}^{n}c_j m(E_j \cap \Omega_1) +  \sum_{j=1}^{n}c_j m(E_j \cap \Omega_2) \\
            &= \sum_{j=1}^{n}c_j \left(m(E_j \cap \Omega_2) + m(E_j \cap \Omega_2) \right)\\
            &= \sum_{j=1}^{n}c_j m(E_j)  = \int_{\Omega} f
        \end{align*}

    \end{enumerate}

\end{proof}

\subsection{Integration of Non-negative Measurable Functions}

\begin{definition}
    Let $f: \Omega \to \mathbb{R}$ and $g: \Omega \to \mathbb{R}$ be functions. We
say that $f$ majorizes $g$, or$g$ minorizes$f$ , if we have $f (x) \ge g(x)$ for all $x \in \Omega$.
\end{definition}

\begin{definition}[Lebesgue integral for non-negative functions]
   Let be a measurable subset of $\mathbb{R}^n$, and let $f: \Omega \to [0,\infty]$ be measurable and non-negative. Then 

   \[
    \int_{\Omega} f = \sup \{ \int_{\Omega} s,\: s \text{ is simple and non-negative, and minorizes } f \}
   \]
\end{definition}

\begin{thm}
    Let $\Omega$ be a measurable set, and let $f,g: \Omega \to [0,\infty]$ be non-negative measurable functions.
    Then:

    \begin{enumerate}
        \item We have $0 \le \int_{\Omega}f \le \infty$. Furthermore, we have $\int_{\Omega} f = 0$ if and only if $f(x) = 0$
for almost every $x \in \Omega$

        \item For any non-negative number $c$, we have $c \int_{\Omega}f  = \int_{\Omega}cf$

        \item  If $f (x) \le g(x)$ for all $x \in \Omega$ , then we have $\int_{\Omega} f \le \int_{\Omega} g$.

        \item If $f (x)= g(x)$ for almost every $\in \Omega$, then $\int_{\Omega}f = \int_{\Omega} g$.

        \item If $\Omega' \subseteq \Omega$ is measurable, then

        \[
            \int_{\Omega} f \chi_{\Omega'} \le \int_{\Omega} f
        \]
    \end{enumerate}
\end{thm}

\begin{proof}
    proofs:

    \begin{enumerate}
        \item $0 \le \int_{\Omega}f \le \infty$ and $\int_{\Omega} f$ if and only if $f(x) = 0$
for almost every $x \in \Omega$

        obviously by definition. If $m(\{ x: f(x) > 0\}) = 0$, for any $s$ minorizes $f$, we have $\int_{\Omega}s = 0$.

        also, If $m(\{ x: f(x) > 0\}) > 0$, there should exists $n$, such that $m(\{ x: f(x) > 1/n\}) > 0$, define $E = \{ x: f(x) > 1/n\}$ 

        and thus exists some simple function $s, \: s(x) = \chi_{E} 1/n$ minorizes $f$, and $\int_{\Omega}s > 0$.

        \item $c \int_{\Omega}f  = \int_{\Omega}cf$

        when $c = 0$, it is trivial

        let $s_n$ minorizes $f$ by $cs_n \le cf$
        
        \[
            c\int_{\Omega}s_n = \int_{\Omega}cs_n
        \]

        take $\int_{\Omega}s_n \to \int_{\Omega}f$, we got

        \begin{align*}
            \varlimsup_{n \to \infty}c\int_{\Omega}s_n &= c\int_{\Omega}f = \varlimsup_{n \to \infty}\int_{\Omega}cs_n \\
            & \le \int_{\Omega}cf
        \end{align*}

        similarly, take $s_n$ minorizes $ cf$, we got $s_n/c$ minorizes $f$

        \[
            \frac{1}{c}\int_{\Omega}s_n = \int_{\Omega}\frac{1}{c}s_n
        \]

        take $\int_{\Omega}s_n \to \int_{\Omega}cf$, we got

        \[
            \frac{1}{c} \int_{\Omega}cf \le \int_{\Omega}f
        \]

        after all, we  got

        \[
        \int_{\Omega}cf = c\int_{\Omega}f
        \]

        \item If $f (x) \le g(x)$ for all $x \in \Omega$ , then we have $\int_{\Omega} f \le \int_{\Omega} g$

        take $s_n$ minorizes $f$, which also minorizes $g$ by $f \le g$.

        \[
            \int_{\Omega}s_n \le \int_{\Omega} g
        \]

        take $\int_{\Omega}s_n \to \int_{\Omega}f$, Then

        \[
            \int_{\Omega}f \le \int_{\Omega} g
        \]

        \item If $f (x)= g(x)$ for almost every $\in \Omega$, then $\int_{\Omega}f = \int_{\Omega} g$.

        consider if $s_n$ minorizes $f$, by redefine $s_n(x) = 0, \forall x \in \{x: f(x) \ne g(x)\}$,
        we got $s_n^*$ which minorizes $g$. Thus

        \[
            \int_{\Omega}s_n  = \int_{\Omega}s_n^* \le \int_{\Omega}g
        \]

        take $\int_{\Omega}s_n \to \int_{\Omega}f$, we got

        \[
          \int_{\Omega}f  \le \int_{\Omega}g
        \]

        similarly, we can prove


        \[
          \int_{\Omega}g  \le \int_{\Omega}f
        \]

        thus

        \[
          \int_{\Omega}f = \int_{\Omega}g
        \]


        \item If $\Omega' \subseteq \Omega$ is measurable, by 
        
        \[
        f\chi_{\Omega'} \le f\chi_{\Omega} \le f
        \]

        and (3)

        we have

        \[
            \int_{\Omega} f \chi_{\Omega'} \le \int_{\Omega} f
        \]
    \end{enumerate}
\end{proof}

\begin{thm}[Lebesgue monotone convergence theorem]
    \label{thm:lebesgue-MCT-17b9b7af-2c23-4932-abc4-509e9a8e199f}
    let $\Omega$ be a measurable subset of $\mathbb{R}^n$, and let $\{ f_n \}$
    be a sequence of non-negative measurable functions from $\Omega $ to $[0,\infty]$
    which are increasing in the sense that

    \[
        0 \le f_1(x) \le f_2(x) \le f_3(x) \le ... \quad \text{for all } x \in \Omega
    \]

    Thus

    \[
        \lim_{n \to \infty}\int_{\Omega}f_n = \int_{\Omega} \left(\lim_{n \to \infty}f_n \right) 
    \]
\end{thm}

\begin{proof}
    define 

    \[
        f = \lim_{n \to \infty}f_n
    \]

    by 

    \[
        \int_{\Omega} f_n \le \int_{\Omega} f
    \]

    then

    \[
        \lim_{n \to \infty}\int_{\Omega} f_n \le \int_{\Omega} f
    \]

    fix $0 < \epsilon < 1$, we will prove, for any simple measurable function $s$ minorizes $f$, we have

    \[
        \epsilon\int_{\Omega} s \le \lim_{n \to \infty}\int_{\Omega} f_n
    \]

    let's define $E_n$ as:

    \[
        E_n = \{ x \in \Omega: f_n(x) \ge \epsilon s(x) \}
    \]

    by $f_n \to f,\: s(x) \le f(x), \: 0 < \epsilon < 1$  we have

    \[
        \bigcup_{n=1}^{\infty}E_n = \Omega
    \]

    expand $s$ as 

    \[
        s = \sum_{j=1}^{m}c_j \chi_{F_j}
    \]

    where $c_j \ge 0$ and $F_j$ be disjoint.

    now, let's consider:

    \begin{align*}
        \int_{\Omega} f_n  &\ge \int_{\Omega} f_n \chi_{E_n} \ge \epsilon \int_{\Omega} s \chi_{E_n} \\
        & \ge \epsilon \int_{\Omega}\sum_{j=1}^{m}c_j \chi_{F_j \cap E_n} \\
        & \ge \epsilon \sum_{j=1}^{m}c_j m(F_j \cap E_n)
    \end{align*}

    and by $E_{n+1} \supseteq E_n$, and $E_1 \cup E_2 \cup ... = \Omega$

    \begin{align*}
        \lim_{n \to \infty}  \sum_{j=1}^{m}c_j m(F_j \cap E_n) &= \sum_{j=1}^{m} \lim_{n \to \infty} c_j m(F_j \cap E_n) \\
        & = \sum_{j=1}^{m} c_j m(F_j) \\
        &= \int_{\Omega} s
    \end{align*}

    thus we got

    \[
        \lim_{n \to \infty}\int_{\Omega} f_n  \ge \epsilon \int_{\Omega} s
    \]

    take $\epsilon \to 1$, we got


    \[
        \lim_{n \to \infty}\int_{\Omega} f_n  \ge \int_{\Omega} s
    \]

    since $s$ is arbitrary simple measurable function minorizes $f$, we got


    \[
        \lim_{n \to \infty}\int_{\Omega} f_n  \ge \int_{\Omega} f
    \]

    after all,

    we got

    \[
        \lim_{n \to \infty}\int_{\Omega} f_n  = \int_{\Omega} f
    \]
\end{proof}

\begin{thm}
    \label{fbb1399e-f592-4e23-be54-aa7b9783005b}
    let $f: \Omega \to [0, \infty]$ be non-negative measurable function.

    Then there exists sequence of non-negative, simple, measurable functions from $\Omega $ to $[0,\infty)$
    which are increasing in the sense that

    \[
        0 \le f_1(x) \le f_2(x) \le f_3(x) \le ... \quad \text{for all } x \in \Omega
    \]

    and

    \[
        \lim_{n \to \infty}f_n(x) = f(x) \quad \text{for all } x \in \Omega
    \]

\end{thm}

\begin{proof}
   we define $f_n$ as:

   \begin{align*}
    f_n(x) & = n  \quad \text{ when } f(x) > n \\
    f_n(x) & = \lfloor f(x) \rfloor + \frac{\lfloor2^n \{ f(x) \} \rfloor}{2^n}  \quad \text{ when } f(x) \le n \\
   \end{align*}
   
   where $[x]$ is integer part of $x$, and $\{x\} = x - \lfloor x \rfloor$

   first, $f_n \to f$, is obviously, assume $f(x) = c$, if $c = \infty$, then $f_n(x) = n$, where $n \to \infty$.

   if $c < \infty$, when $n \ge \lfloor c \rfloor + 1$, we have 
   
   \[
 \lfloor c \rfloor + \{ c \} - \frac{1}{2^n} < \lfloor c \rfloor + \frac{\lfloor 2^n \{ c \} \rfloor}{2^n} \le \lfloor c \rfloor + \{ c \}
   \]

   thus $f_n(x) \to c$ obviously.

   now, let's prove $f_{n+1}(x) \ge f_n(x)$:
   
   if $x > n$, we have $f_{n}(x) = n$ and:
   
   \begin{enumerate}
    \item if $f(x) > n+1$

    $f_{n+1}(x) = n+1 \ge n$

    \item if $n < f(x) \le n+1$

    $f_{n+1}(x) \ge \lfloor n \rfloor \ge n$ 
   \end{enumerate}

   so $f_n(x) \le f_{n+1}(x)$ when $f(x) > n$

   if $f(x) \le n$, let's check $f_{n+1}(x) - f_n(x)$, and define $c = f(x)$

   by $\lfloor x+y \rfloor \ge \lfloor x \rfloor + \lfloor y \rfloor$

   \begin{align*}
    f_{n+1}(x) &=  \frac{\lfloor 2^{n+1} \{ c \}\rfloor}{2^{n+1}} = \frac{\lfloor 2^{n} \{ c \} + 2^{n} \{ c \}\rfloor}{2^{n+1}} \\
    & \ge\frac{\lfloor 2^{n} \{ c \}] + \lfloor 2^{n} \{ c \}]}{2^{n+1}} \ge \frac{\lfloor 2^{n} \{ c \}\rfloor}{2^{n}}
   \end{align*}

   which means 
   
   \[
   f_{n+1}(x) -  f_n(x) = \frac{\lfloor 2^{n+1} \{ c \}  ]}{2^{n+1}}- \frac{\lfloor 2^{n} \{ c \}\rfloor}{2^{n}} \ge 0
   \]
\end{proof}


\begin{thm}[Interchange of addition and integration]
    \label{a1203caa-2e3d-4549-b385-8acd12b2d13f}
    Let $\Omega$ be a measurable subset of $\mathbb{R}^n$, and let $f,g: \Omega \to [0, \infty]$  be measurable functions.  
    
    Then

    \[
        \int_{\Omega} \left( f + g \right) = \int_{\Omega}  f + \int_{\Omega}  g
    \]
\end{thm}

\begin{proof}
    let $f_n, g_n$ be simple, non-negative, measurable functions, where $f_n \to f, \: g_n \to g$. By 
    \autoref{fbb1399e-f592-4e23-be54-aa7b9783005b}, we got

    \begin{align*}
        \lim_{n \to \infty}\int_{\Omega} f_n &= \int_{\Omega} f \\
        \lim_{n \to \infty}\int_{\Omega} g_n &= \int_{\Omega} g \\
    \end{align*}

    and thus


    \begin{align*}
        \lim_{n \to \infty}\left(\int_{\Omega} (f_n + g_n) \right) &= \int_{\Omega} f + \int_{\Omega} g\\
    \end{align*}

    consider $f_n + g_n \to f + g$ and $f_n + g_n$ is also increasing, thus

    \[
        \lim_{n \to \infty}\left(\int_{\Omega} (s_n + g_n) \right) = \int_{\Omega} \left(\lim_{n \to \infty}(s_n + g_n) \right) = \int_{\Omega}(f+g)
    \]

    combine them, we got

    \[
        \int_{\Omega} \left( f + g \right) = \int_{\Omega}  f + \int_{\Omega}  g
    \]
\end{proof}

\begin{thm}
    Let $\Omega$ be a measurable subset of $\mathbb{R}^n$, and let $f: \Omega \to [0, \infty]$  be measurable functions. 
    And disjoint set $\Omega_1 \cup \Omega_2 = \Omega$, Then

    \[
        \int_{\Omega_1} f + \int_{\Omega_2}f = \int_{\Omega}f
    \]

\end{thm}

\begin{proof}
    we just need to prove:

    \[
        \int_{\Omega_1} f = \int_{\Omega} f \chi_{\Omega_1}
    \]

    and then use \autoref{a1203caa-2e3d-4549-b385-8acd12b2d13f},

    assume arbitrary simple measurable, non-negative $s_n$ minorizes $f$ on $\Omega_1$, we can extend it to $t_n: \Omega \to \mathbb{R}$ as
    $t_n = s_n \chi_{\Omega_1}$

    Thus we have

    \[
        \int_{\Omega_1}s_n = \int_{\Omega}t_n \le \int_{\Omega} f \chi_{\Omega_1}
    \]

    take $\int_{\Omega_1}s_n \to \int_{\Omega_1} f$, we got

    \[
\int_{\Omega_1} f \le \int_{\Omega} f \chi_{\Omega_1}
    \]

    similarly, for any $t_n$ minorizes $f \chi_{\Omega}$, it also minorizes $f$ on $\Omega_1$, 
    use same step, define $s_n = t_n \chi_{\Omega_1}$ we got

    \begin{align*}
        \int_{\Omega_1}t_n &= \int_{\Omega}t_n - \int_{\Omega \setminus \Omega_1}t_n \\
        &=\int_{\Omega}t_n \le \int_{\Omega_1}f 
    \end{align*}

    and take $\int_{\Omega}t_n \to \int_{\Omega}f \chi_{\Omega_1}$

    \[
        \int_{\Omega}f\chi_{\Omega_1} \le \int_{\Omega_1}f 
    \]
\end{proof}

\begin{thm}[Fatou’s lemma]
    \label{thm:fatous-lemma}
     Let $\Omega$ be a measurable subset of $\mathbb{R}^n$, and let
     $f_1,f_2,...$ be a sequence of non-negative functions from $\Omega$ to $[0, \infty]$.
     Then

     \[
        \int_{\Omega} \varliminf_{n \to \infty} f_n \le \varliminf_{n \to \infty} \int_{\Omega}  f_n
     \]
\end{thm}

\begin{proof}
    consider that 

    \[
        \varliminf_{n \to \infty}f_n(x) = \lim_{n \to \infty} \inf_{k \ge n}f_k(x)
    \]

    and we define

    \[
        h_n(x) =\inf_{k \ge n}f_k(x)
    \]

    thus $h_n$ is increasing, by \autoref{thm:lebesgue-MCT-17b9b7af-2c23-4932-abc4-509e9a8e199f}, we got

    \[
        \int_{\Omega} \lim_{n \to \infty}h_n = \lim_{n \to \infty} \int_{\Omega} h_n
    \]

    by $h_n \le f_n$, we have

    \[
        \int_{\Omega} h_n \le \int_{\Omega} f_n
    \]

    and thus

    \[
        \varliminf_{n \to \infty}\int_{\Omega} h_n \le \varliminf_{n \to \infty}\int_{\Omega} f_n
    \]

    since

    \[
\int_{\Omega} h_{n+1} \ge \int_{\Omega} h_{n}
    \]

    we got

    \[
\varliminf_{n \to \infty}\int_{\Omega} h_n = \lim_{n \to \infty}\int_{\Omega} h_n
    \]

    thus we got

    \[
 \int_{\Omega} \varliminf_{n \to \infty} f_n  =       \int_{\Omega} \lim_{n \to \infty}h_n = \lim_{n \to \infty} \int_{\Omega} h_n \le \varliminf_{n \to \infty} \int_{\Omega} f_n
    \]
\end{proof}

\begin{thm}
    Let $\Omega$ be a measurable subset of $\mathbb{R}^n$, and let $f: \Omega \to [0, \infty]$ be a  non-negative measurable function such that $\int_{\Omega}f$ is finite. Then $f$ is finite almost everywhere (i.e., the set $\{x \in \Omega : f(x) = \infty\}$ has measure zero).
\end{thm}

\begin{proof}
    assume 
    
    \begin{align*}
        E &= \{x \in \Omega : f(x) = \infty\} \\
        m(E) & > 0 \\
    \end{align*}

    and we define $s_n$ as:

    \begin{align*}
        s_n(x) = 0 & \quad x \notin E \\
        s_n(x) = n & \quad x \in E \\
    \end{align*}

    thus $s_n$ minorizes $f$, and 
    
    \[
     n \cdot m(E) \le \int_{\Omega}s_n \le \int_{\Omega}f 
    \]

    take $n \to \infty$, we got $\int_{\Omega}f = \infty$
\end{proof}

\begin{thm}[Borel–Cantelli lemma]
    let $\Omega_1,\Omega_2, ...$ be measurable subsets of $\mathbb{R}^n$ such that 

    \[
        \sum_{n=1}^{\infty}m(\Omega_n) < \infty
    \]

    Then the set

    \[
        E = \{ x \in \mathbb{R}^n: x \in \Omega_n \text{ for infinitely many } n \}
    \]

    is a set of measure zero.
\end{thm}

\begin{proof}
    define 

    \[
        \Omega = \bigcup_{n=1}^{\infty}\Omega_n
    \]

    and define $f: \Omega \to [0, \infty]$ as

    \[
        f(x) = \sum_{n=1}^{\infty}\chi_{\Omega_n}
    \]

    thus 

    \[
        E = \{x \in \Omega: f(x) = \infty \}
    \]

    by \autoref{thm:lebesgue-MCT-17b9b7af-2c23-4932-abc4-509e9a8e199f}

    \[
        \int_{\Omega}\sum_{n=1}^{\infty}\chi_{\Omega_n} = \sum_{n=1}^{\infty}\int_{\Omega}\chi_{\Omega_n} = \sum_{n=1}^{\infty} m(\Omega_n) < \infty
    \]

    Thus $E$ has measure of zero, otherwise $\int_{\Omega} f = \infty$
\end{proof}

\subsection{Integration of Absolutely Integrable Functions}

\begin{definition}[Absolutely integrable functions]
    Let $\Omega$ be a measurable subset of $\mathbb{R}^n$. 
    A measurable function $f: \Omega \to \mathbb{R}^*$ is said  
    to be absolutely integrable if the  integral

    \[
        \int_{\Omega} |f| 
    \]

    is finite
\end{definition}

\begin{definition}[Lebesgue integral]
    let $f: \Omega \to \mathbb{R}^*$ be an absolutely integrable  function. We define the Lebesgue integral
    $\int_{\Omega}f $ as

    \[
        \int_{\Omega}f = \int_{\Omega}f^+ - \int_{\Omega}f^-
    \]

    where

    \begin{align*}
        f^+ &= \frac{|f| + f}{2} \\
        f^- &= \frac{|f| - f}{2} \\
    \end{align*}

    Note that

    \[
\int_{\Omega}f^+ \le \int_{\Omega}|f| < \infty
    \]

    thus we are never encountering the indeterminate form $\infty - \infty$
\end{definition}

\begin{thm}
    let $\Omega$ be a measurable set, and let $f \Omega \to \mathbb{R}^*$ and $g: \Omega \to \mathbb{R}^*$ be absolutely integrable functions.

    Then

    \begin{enumerate}
        \item For any real number $c$ (positive, zero, or negative), we have that $cf$ is absolutely  integrable and

        \[
            \int_{\Omega} cf = c\int_{\Omega} f
        \]

        \item If $\Omega_1 \cup \Omega_2 = \Omega$ and $\Omega_1 \cap \Omega_2 = \emptyset$.
        
        Then $f$ is absolutely integrable on $\Omega$ iff $f$ is absolutely on $\Omega_1$ and $\Omega_2$ and

        \[
            \int_{\Omega_1}f +\int_{\Omega_2}f = \int_{\Omega}f
        \]

        \item $f+g$ is absolutely integrable and

        \[
            \int_{\Omega}(f+g) = \int_{\Omega}f +  \int_{\Omega}g
        \]

        \item If $f \le g$ for all $x \in \Omega$, then $\int_{\Omega}f \le \int_{\Omega} g$


        \item If $f = g$ a.e. $x \in \Omega$, then $\int_{\Omega}f = \int_{\Omega} g$
    \end{enumerate}
\end{thm}

\begin{proof}
    proofs:
    
    \begin{enumerate}
        \item consider:
        
        \[
            \int_{\Omega}|cf| = \int_{\Omega}|c||f| = |c|\int_{\Omega}|f| < \infty
        \]

        so $cf$ is absolutely integrable.

        let's assume $c$ is non-negative

        \begin{align*}
            \int_{\Omega}(cf)^+ &= \int_{\Omega}cf^+ = c\int_{\Omega}f^+ \\
            \int_{\Omega}(cf)^- &= \int_{\Omega}cf^- = c\int_{\Omega}f^- \\
\int_{\Omega}(cf)^+ - \int_{\Omega}(cf)^- &= c\int_{\Omega}f^+ - c\int_{\Omega}f^- \\
&= c\left(\int_{\Omega}f^+ - \int_{\Omega}f^- \right) = c\int_{\Omega} f
        \end{align*}

        when $c$ is negative, we just need some modification like $(cf)^+ = |c|f^-$ and $(cf)^- = |c| f^+$

        \item if $f$ is absolutely integrable on $\Omega$. Then 

        \[
           \int_{\Omega_1}|f| + \int_{\Omega_2}|f|  = \int_{\Omega}|f| 
        \]

        which show $f$ is absolutely integrable on $\Omega$ iff it is absolutely integrable on both $\Omega_1, \Omega_2$

        and

        \begin{align*}
            \int_{\Omega}f  &= \int_{\Omega}f^+ - \int_{\Omega}f^- \\
            &= \int_{\Omega_1}f^+ + \int_{\Omega_2}f^+ - (\int_{\Omega_1}f^- + \int_{\Omega_2}f^-) \\
            &= \int_{\Omega_1} f + \int_{\Omega_2} f
        \end{align*}

        \item for $f+g$ 

        it is easily to prove $f+g$ is absolutely integrable by triangle inequality.

        let's define 

        \begin{align*}
            E_1 &= \{ x: |f(x)| < \infty\} \\
            E_2 &= \{ x: |g(x)| < \infty\} \\
        \end{align*}

        because $m(\Omega \setminus E_1) = 0$ and $m(\Omega \setminus E_2) = 0$, we have

        $m(\Omega \setminus(E_1 \cap E_2)) = 0$, thus we can discuss on integration over $E = E_1 \cap E_2$, 
        so as avoid indeterminate $\infty - \infty$


        define

        \begin{align*}
            h &= f+ g\\
        \end{align*}

        then

        \begin{align*}
            h &= h^+ - h^- = f^+ + g^+ - (f^- + g^-) \\
            h^+ + f^- + g^- &= f^+ + g^+ + h^- 
        \end{align*}

        and integration at both side

        \[
            \int_{E}h^+ + \int_{E}f^- + \int_{E}g^- = \int_{E}f^+ + \int_{E}g^+ +\int_{E}h^-
        \]

        thus

        \begin{align*}
\int_{E}h^+ - \int_{E}h^- &= \int_{E}f^+ + \int_{E}g^+- \int_{E}f^- - \int_{E}g^- \\
&= \int_{E}f^+ - \int_{E}f^- + \int_{E}g^+ - \int_{E}g^- \\
&= \int_{E}f + \int_{E}g
        \end{align*}

        now, we can replace $E$ with $\Omega$ by(2).
        
        \item If $f \le g$ for all $x \in \Omega$, then $\int_{\Omega}f \le \int_{\Omega} g$

        \[
            \int_{\Omega}f  - \int_{\Omega}g = \int_{\Omega}(f -g) \ge 0
        \]


        \item If $f = g$ a.e. $x \in \Omega$, then $\int_{\Omega}f = \int_{\Omega} g$

        define 

        \[
            \Omega_1 = \{ x \in \Omega: f(x) \ne g(x)\}
        \]

        Then

        \begin{align*}
            \int_{\Omega}f &= \int_{\Omega_1}f + \int_{\Omega \setminus \Omega_1}f = \int_{\Omega \setminus \Omega_1}g \\
            \int_{\Omega}g &= \int_{\Omega_1}g + \int_{\Omega \setminus \Omega_1}g = \int_{\Omega \setminus \Omega_1}g
        \end{align*}
    \end{enumerate}
\end{proof}

\begin{thm}
    \label{thm:lebesgue-DCT-744c7182-13fa-4e6c-910a-a68f67181b29}
    let $\Omega$ be a measurable subset of $\mathbb{R}^n$, and let $f_1,f_2,...$ be a sequence of
    measurable functions from $\Omega$ to $\mathbb{R}^*$ which converge pointwise. Suppose 
    also that there is an absolutely integrable function $F: \Omega \to [0, \infty]$ such that
    $|f_n(x)| \le F(x),\: \forall x \in \Omega$ and all $n=1,2,3,...$. Then

    \[
        \int_{\Omega}\lim_{n \to \infty} f_n = \lim_{n \to \infty} \int_{\Omega} f_n
    \]

\end{thm}

\begin{proof}
    since $F$ is absolutely measurable, the set

    \[
        E = \{ x \in \Omega: |F(x)| = \infty \}
    \]

    has measure of zero, so we can discuss on $\Omega' = \Omega \setminus E$, without loss of generality, 
    we will use $\Omega$ as $\Omega'$, since $\Omega$ and $\Omega'$ are interchangeable when used for integration.

    let's define $f: \Omega \to \mathbb{R}^*$

    \[
        f(x) = \lim_{n \to \infty}f_n(x)
    \]

    by $f_n$ is measurable and $f_n \to f$, so $f$ is also measurable. consider

    \[
        \{x: f(x) < a \} = \bigcup_{m=1}^{\infty} \bigcup_{n=1}^{\infty} \bigcap_{k=n}^{\infty} \{x: f_k(x) \le a - \frac{1}{m}\}
    \]

    consider $F + f_n$ are non-negative and converge pointwise to $F+f$, will not increasing, we can apply 

    \autoref{thm:fatous-lemma}, by $F$ and $f$ are both absolutely integrable, so that

    \begin{align*}
        \int_{\Omega} f + F &\le \varliminf_{n \to \infty} \int_{\Omega} f_n + F \\
        & \le \int_{\Omega}F + \varliminf_{n \to \infty} \int_{\Omega} f_n
    \end{align*}

    after apply cancellation law, we got

    \[
        \int_{\Omega}f \le \varliminf_{n \to \infty} \int_{\Omega} f_n
    \]

    and $F - f_n$ is also non negative, thus we got

    \[
        \int_{\Omega} F - f \le \varliminf_{n \to \infty}\int_{\Omega} F - f_n
    \]

    after apply cancellation law, we got

    \[
        \varlimsup_{n \to \infty} \int_{\Omega} f_n \le \int_{\Omega}f
    \]

    after all, we got

    \[
 \int_{\Omega}f \le \varliminf_{n \to \infty} \int_{\Omega} f_n \le       \varlimsup_{n \to \infty} \int_{\Omega} f_n \le \int_{\Omega}f 
    \]

    so $\int_{\Omega} f_n$ converges and

    \[
\lim_{n \to \infty} \int_{\Omega} f_n = \int_{\Omega} f
    \]

\end{proof}

\begin{definition}[Upper and lower Lebesgue integral]
    Let $\Omega$ be a measurable subset of $\mathbb{R}^n$, and let $f: \Omega \to \mathbb{R}$ 
    be a function(not necessarily measurable). We define the  upper Lebesgue integral as

    \[
        \overline{\int_{\Omega}} f = \inf \{ \int_{\Omega}g : g \text{ is absolutely integrable function, g majorizes f}\}
    \]

    and


    \[
        \underline{\int_{\Omega}} f = \sup \{ \int_{\Omega}g : g \text{ is absolutely integrable function, g minorizes f}\}
    \]
\end{definition}

\begin{thm}
    let $g,h : \Omega \to \mathbb{R}^*$ be absolutely integrable, and $g \le f \le h$,

    Then $f$ is also absolutely integrable
\end{thm}

\begin{proof}
    consider that 

    \begin{align*}
        |f| &= |f - g + g| \\
        & \le f-g + |g| \le h-g + |g| \\
        & \le h + |g| \le |h| + |g|
    \end{align*}

    thus we have

    \[
        |f| \le |g| + |h|
    \]

    and

    \[
        \int_{\Omega}|f| \le \int_{\Omega}|g| + |h| = \int_{\Omega}|g| + \int_{\Omega}|h| < \infty
    \]
\end{proof}

\begin{thm}
    let $f,g,h: \Omega \to \mathbb{R}^*$, if $g \le f \le h$ and $g = h$ a.e. Then $f = g = h$ a.e.
\end{thm}

\begin{proof}

    when $g(x) < f(x)$, by $g(x) < f(x) \le h(x)$, thus $g(x) < h(x)$, so

    \begin{align*}
        m(\{x: g(x) \ne f(x)\}) & \le m(\{x: g(x) \ne f(x)\}) = 0
    \end{align*}

    similarly for $f(x) < h(x)$
\end{proof}

\begin{thm}
    Let be a measurable subset of $\mathbb{R}^n$, and let $f: \Omega \to \mathbb{R}$ 
    be a function  (not necessarily measurable). Let $A$ be a real number, and suppose

    \[
        \overline{\int_{\Omega}}f =\underline{\int_{\Omega}}f = A
    \]
    
    Then f is absolutely integrable, and 

    \[
        \int_{\Omega}f  = A
    \]
\end{thm}

\begin{proof}
    let $f^+_n, f^-_n$ be absolutely integrable, while $f_n^+$ majorizes $f$, $f^-_n$ minorizes $f$, so as

    \begin{align*}
        \lim_{n \to \infty}\int_{\Omega} f^+_n &= A  \\
        \lim_{n \to \infty}\int_{\Omega} f^-_n &= A  \\
    \end{align*}

    and define

    \begin{align*}
        F^+(x) = \inf_{n} f^+_n(x) \\
        F^-(x) = \sup_{n} f^-_n(x) \\
    \end{align*}

    thus $F^+$ is absolutely integrable by $f^-_1 \le F^+ \le f^+_1$, similarly for $F^-$. They are
    also measurable, since $\inf$ and $\sup$ operator on function sequence preserve measurable.

    since they are both absolutely integrable, we can discuss on their integration. By $f^-_n \le F^- \le F^+ \le f_n$. We have

    \[
  \int_{\Omega}f^-_n \le \int_{\Omega} F^- \le \int_{\Omega}F^+ \le \int_{\Omega}f^+_n
    \]

    take $n \to \infty$, we have

    \[
\int_{\Omega} F^- = \int_{\Omega}F^+
    \]

    thus 

    \[
        \int_{\Omega}F^+ - F^- = 0
    \]

    since $F^+ - F^- \ge 0$, we must have $F^+ = F^-$ a.e. By $F^- \le f \le F^+$, we have $f = F^- = F^+$ a.e.

    Thus $f$ is absolutely integrable and

    \[
        \int_{\Omega}f = A
    \]
\end{proof}



\subsection{Comparison with the Riemann Integral}

\begin{thm}
    let $[a,b] \subseteq \mathbb{R}$ be a bounded interval, and $f: [a,b] \to \mathbb{R}$
    be a riemann integrable function. Then $f$ is also absolutely integrable,
    and

    \[
        \int_a^b f \mathrm{d}m = \int_a^b f \mathrm{d}x
    \]

    we use $\mathrm{d}m$ to denote lebesgue integral, and $\mathrm{d}x$ denote riemann integral here,
\end{thm}

\begin{proof}
    let's define

    \[
        \int_a^b f \mathrm{d}x = A
    \]

    since $f$ is riemann integrable. There exists a partition sequence $P_n$, such that

    \begin{align*}
        \lim_{n \to \infty}U(P_n, f)  &= A \\
        \lim_{n \to \infty} L(P_n, f) & =  A \\
    \end{align*}

    where $U(P_n, f)$ and $L(P_n, f)$ represents riemann upper sum and riemann lower sum here.

    Since $f$ is riemann integrable, and thus bounded, assume $|f| \le M$

    Now, we can construction simple function $s_n$ majorizes $f$ by $P_n$:

    \[
        s_n = \sum_{[x_{i-1},x_i] \in P_n} \left(\sup_{[x_{i-1}, x_{i}]} f \right) \chi_{[x_{i-1}, x_i)}
    \]

    we don't need to define when $x = b$, since it will not affect integration of $s_n$

    by $|f| \le M$, we have $|s_n| \le M$, so $s_n$ is 
    absolutely measurable.  And

    \[
        \int_{a}^b s_n \mathrm{d}m = U(P_n, f)
    \]

    Similarly, we can define $t_n$ as


    \[
        t_n = \sum_{[x_{i-1},x_i] \in P_n} \left(\inf_{[x_{i-1}, x_{i}]} f \right) \chi_{[x_{i-1}, x_i)} 
    \]

    Thus


    \[
        \int_{a}^b t_n \mathrm{d}m = L(P_n, f)
    \]

    Now we have two absolutely integrable function sequence, which majorizes and minorizes $f$ separately.
    And both of their integration converges to same real number $A$. So $f$ should also 
    be absolutely integrable, and its integration is $A$.

\end{proof}

\subsection{Fubini's Theorem}

\begin{thm}
    Let $f: \mathbb{R}^2 \to \mathbb{R}$ be an absolutely integrable function.
    Then there exists absolutely integrable functions $F: \mathbb{R} \to \mathbb{R}$
    and $G: \mathbb{R} \to \mathbb{R}$ such that for a.e. $x$, $f(x,y)$ is absolutely
    integrable in $y$ with

    \[
        F(x) = \int_{\mathbb{R}} f(x,y) \mathrm{d}y
    \]

    and for a.e. $y$, $f(x,y)$ is absolutely integrable in $x$ with


    \[
        G(x) = \int_{\mathbb{R}} f(x,y) \mathrm{d}x
    \]

    Finally, we have

    \[
        \int_{\mathbb{R}} F(x) \mathrm{d}x =\int_{\mathbb{R}} G(y) \mathrm{d}y = \int_{\mathbb{R}^2} f
    \]
\end{thm}

\begin{proof}
    Steps:

    \begin{enumerate}
        \item when $f(x,y)$ is indicator function on $[-N, N] \times [-N, N]$


    \end{enumerate}
\end{proof}