\section{Differentia}

\subsection{The derivative of a real function}

\begin{definition}
    let $f: [a,b] \to \R$ For any $x \in [a,b]$ form the quotient:

    \[
        \phi(t) = \frac{f(t) -f(x)}{t-x} \quad (a < t <b,\: t \ne x)
    \]

    and define

    \[
        f'(x) = \lim_{t \to x} \phi(t)
    \]
\end{definition}

\begin{thm}
    if $f$ be defined on $[a,b]$,  if $f$ is differentiable at a point $x \in [a,b]$.
    then $f$ is continuous at $x$
\end{thm}

\begin{proof}
    by definition, $x$ is a limit point of $[a,b]$, assume $x_n \to x$.

    by 
    
    \[
    \frac{f(x_n) - f(x)}{  x_n - x}
    \] converges, we got:

    \begin{align}
        \lim_{n \to \infty}f(x_n) - f(x) &= \lim_{n \to \infty}\frac{f(x_n) - f(x)}{x_n -x}(x_n - x) \\
        &= \left(\lim_{n \to \infty}\frac{f(x_n) - f(x)}{x_n -x}\right) \left(\lim_{n \to \infty}x_n -x\right) \\
        &= 0
    \end{align}
\end{proof}

\begin{thm}
    let $f,g$ are defined on $[a,b]$ and are differentiable at $x \in [a,b]$.

    Then $f+g, fg, f/g$ are differentiable at $x$ and

    \begin{enumerate}
        \item $(f+g)'(x) = f'(x) + g'(x)$
        \item $(fg)'(x) = f'(x)g(x) + f(x)g'(x)$ 
        \item for $f/g$

        \[
            (\frac{f}{g})'(x) = \frac{f'(x)g(x) - f(x)g'(x)}{g^2(x)}
        \]
    \end{enumerate}
\end{thm}

\begin{proof}
    for (1), easily to prove by laws of limit

    for (2):

    \begin{align*}
        \lim_{t \to x}\frac{f(t)g(t) - f(x)g(x)}{t-x} &= \lim_{t \to x}\frac{f(t)g(t) + f(t)g(x) - f(t)g(x) - f(x)g(x)}{t-x} \\
        &= \lim_{t \to x}\frac{f(t)g(t) - f(t)g(x)}{t-x} + \lim_{t \to x}\frac{f(t)g(x) - f(x)g(x)}{t-x} \\
        &= \left(\lim_{t \to x}f(t) \right)\left( \lim_{t \to x}\frac{g(t) - g(x)}{t-x} \right) + g(x)f'(x) \\
        &= f(x)g'(x) + g(x)f'(x)
    \end{align*}

    for (3):

    by: $(f/g \cdot g)' = f'$

    \begin{align*}
        (f/g)'g + g'(f/g) &= f' \\
        (f/g)' &= (f' - g'(f/g))/g \\
        (f/g)' &= (f'g - g'f)/g^2 \\
    \end{align*}
\end{proof}

\begin{thm}
    suppose $f$ is continuous on $[a,b]$, $f'(x)$ exists at some point $x \in [a,b]$
    , $g$ is defined on an interval $I$ which contains the range of $f$, and $g$ is differentiable
    at the point $f(x)$. If

    \[
        h(t) = g (f(t))
    \]

    then $h$ is differentiable at $x$, and 

    \[
        h'(x) = g'(f(x))f'(x)
    \]
\end{thm}

\begin{proof}
   pick $x_n \in [a,b]$  and $x_n \to x, x_n \ne x$, since $f$ is continuous on $[a,b]$. we got $f(x_n) \to f(x)$


   Assume $f(x_n) = f(x)$ occurs finitely times, Then:

   \begin{align*}
        \lim_{n \to \infty}\frac{h(x_n) - h(x)}{x_n -x} &= \lim_{n \to \infty}\frac{g(f(x_n)) - g(f(x))}{x_n -x} \\
        &=  \lim_{n \to \infty}\frac{g(f(x_n)) - g(f(x))}{f(x_n) -f(x)}\frac{f(x_n) - f(x)}{x_n -x} \\
        &= g'(f(x)) f'(x)
   \end{align*}

   if $f(x_n) = f(x)$ occurs infinitely many times, consider:

   \[
    f'(x) = \lim_{n \to \infty}\frac{f(x_n) - f(x)}{x_n -x}
   \]

   which means $0$ is a subsequential limit of:
   
   \[
        \frac{f(x_n) - f(x)}{x_n -x}
   \]

   thus we got $f'(x) = 0$, and we divide $\N$ by $X$ and $Y$:

   \begin{align*}
        X &= \{ n \in \N: f(x_n) - f(x) = 0\} \\
        Y &= \N \setminus X
   \end{align*}

   and both $X$ and $Y$ is infinite, and

   \begin{align*}
        \lim_{n \to \infty, n \in X}\frac{h(x_n) - h(x)}{x_n -x} &= 0 \\
   \end{align*}

   \begin{align*}
        & \lim_{n \to \infty, n \in Y}\frac{h(x_n) - h(x)}{x_n -x} \\
        = &\lim_{n \to \infty, n \in Y}\frac{g(f(x_n)) - g(f(x))}{f(x_n) -f(x)}\frac{f(x_n) - f(x)}{x_n -x} \\ 
        = & g'(f(x))f'(x) = 0
   \end{align*}

   combine $X$ and $Y$, we got

   \[
        \lim_{n \to \infty}\frac{h(x_n) - h(x)}{x_n -x} = 0 \\
   \]

   and $h'(x) = 0 = g'(f(x))f'(x)$

\end{proof}


\subsection{Mean Value Theorems}

\begin{definition}[Local Maximum]
    let $f: X \to \R$ be a real value function defined on a metric space $X$,
    We say $f$ has a local maximum at a point $p \in X$ iff there exists $\delta > 0$  
    such that $f(q) \le f(p)$ for all $q \in Y \cap B(p, \delta)$
\end{definition}

\begin{thm}
    \label{thm:local-maximum-derive-zero}
    let $f$ be defined on $[a,b]$, if $f$ has a local maximum at a point $x \in (a,b)$,
    and if $f'(x)$ exists, then $f'(x) = 0$
\end{thm}

\begin{proof}
   pick $x_n = x  + 1/n$  and $y_n = x - 1/n$ then:

   \begin{align*}
        f'(x) &= \lim_{n \to \infty}\frac{f(x_n) - f(x)}{1/n} \le 0 \\
        f'(x) &= \lim_{n \to \infty}\frac{f(y_n) - f(x)}{-1/n} \ge 0 \\
   \end{align*}
   
   thus we got $f'(x) = 0$
\end{proof}

\begin{thm}[Rolle's Theorem]
    \label{thm:Rolle-Theorem}
    let $f: [a,b] \to \R$ is continuous and differentiable on $(a,b)$ and $f(a) = f(b)$,
    then there exists $x \in (a,b)$ so that $f'(x) = 0$
\end{thm}

\begin{proof}
    let 
    
    \begin{align*}
        M &= \sup_{x \in [a,b]} f(x) \\
        m &= \inf_{x \in [a,b]} f(x) \\
    \end{align*}

    if $m = M$ then $f$ is constant, and $\forall x \in (a,b), \: f'(x) = 0$

    if $m < M$, then at least one of $m \ne f(a), M \ne f(a)$ meets. Assume $M \ne f(a)$, then 
    there exists $c \in (a,b)$ and $f(c) = M$, then $c$ becomes a local maximum, and $f'(c) = 0$
\end{proof}


\begin{thm}[Mean Value Theorem I]
    let $f: [a,b] \to \R$ is continuous and differentiable on $(a,b)$
    then there exists $c \in (a,b)$ so that $f'(c) = (f(b) - f(a))/(b-a)$
\end{thm}

\begin{proof}
    define 

    \[
        g = f(x) - \frac{f(b) - f(a)}{b-a}(x-a)
    \]

    then $g$ is continuous on $[a,b]$ and differentiable on $(a,b)$, and $g(a) = g(b) = f(a)$

    by \autoref{thm:Rolle-Theorem}, there exists $c \in (a,b)$ so that

    \[
        g'(c) = f'(c) - \frac{f(b) - f(a)}{b-a} =  0
    \]

    and 

    \[
        f'(c) = \frac{f(b) - f(a)}{b-a}
    \]
\end{proof}

\begin{thm}[Mean Value Theorem II]
    let $f,g: [a,b] \to \R$ is continuous and differentiable on $(a,b)$
    and $g(b) \ne g(a), g'(x) \ne 0$

    then there exists $c \in (a,b)$ so that:

    \[
        \frac{f(b)- f(a)}{g(b) -g(a)} = \frac{f'(c)}{g'(c)}
    \]
\end{thm}

\begin{proof}
    define:

    \[
        h(x)= \frac{f(b)- f(a)}{g(b) -g(a)}\left( g(x) - g(a) \right) - f(x)
    \]

    then $h(x) $ is continuous on $[a,b]$ and differentiable on $(a,b)$ and

    \[
        h(a) = h(b) = -f(a)
    \]


    by \autoref{thm:Rolle-Theorem}, there exists $c \in (a,b)$ so that

    \[
        h'(c) =\frac{f(b)- f(a)}{g(b) -g(a)} g'(c) - f'(c) = 0 
    \]

    which indicates

    \[
        \frac{f(b)- f(a)}{g(b) -g(a)} = \frac{f'(c)}{g'(c)}
    \]
\end{proof}

\begin{thm}
    suppose $f$ is differentiable in $(a,b)$

    \begin{enumerate}
        \item if $f'(x) \ge 0, \forall x \in (a,b)$, then $f$ is monotonically increasing
        \item if $f'(x) = 0, \forall x \in (a,b)$, then $f$ is constant
        \item if $f'(x) \le 0, \forall x \in (a,b)$, then $f$ is monotonically decreasing
    \end{enumerate}
\end{thm}

\begin{proof}
    for (1):

    assume for contradiction that: exists $x,y \in (a,b),\: x < y$ and $f(x) > f(y)$, by mean theorem,
    there exists $c \in (x,y)$ and $f'(c) = (f(y) - f(x))/(y-x) < 0$, which is contradict with $f'(x) \ge 0$

    Similarly for (2) and (3)
\end{proof}

\subsection{The Continuity of Derivatives}

\begin{thm}
    suppose $f$ is a real differentiable function on $[a,b]$ and suppose
    $f'(a) < \lambda < f'(b)$ Then there is a point $x \in (a,b)$ such that
    $f'(x) = \lambda$
\end{thm}

\begin{proof}
    let $g(t) = f(t) - \lambda t$, and we got $g'(a) < 0$ and $g'(b) > 0$,
    take $t_n = a + 1/n$ and

    \begin{align*}
        \lim_{n \to \infty}\frac{g(t_n) - g(a)}{t_n - a} = g'(a) < 0
    \end{align*}

    which means there are infinitely $n$ such that $g(t_n) < g(a)$

    similarly, if we assign $t_n = b - 1/n$, there are infinitely $n$
    such that $g(t_n) > g(b)$, let's pick $a < t_1 < t_2 < b$ and 
    $g(a) < g(t_1), g(t_2) < g(b)$, so $g$ attains its minimum on $ c \in (a,b)$ 

    by  \autoref{thm:local-maximum-derive-zero}, $g'(c)= f'(c) - \lambda = 0$
\end{proof}

\subsection{L'h\^{o}pital's Rule}

\begin{thm}
    \label{thm:lhopital}
    suppose $f,g$ are real and differentiable in $(a,b)$, and $g'(x) \ne 0$ for all $x \in (a,b)$
    ,where $-\infty \le a < b \le \infty$, and $L \in \R^*$. Suppose

    \[
        \frac{f'(x)}{g'(x)} \to L \quad \mathrm{as} \: x \to a
    \]

    If one of:

    \begin{enumerate}
        \item $f(x) \to 0$ and $g(x) \to 0$ as $x \to a$

        \item $g(x) \to \infty$ as $x \to a$
    \end{enumerate}

    Then:

    \[
        \frac{f(x)}{g(x)} \to L \quad \mathrm{as}\: x \to a
    \]
\end{thm}

\begin{proof}
    At first, we need to prove $g(x)$ is injective, it is easy to prove by mean value theorem since $g'(x) \ne 0$.

    Now assume $-\infty \le A < \infty$, we pick $A < r$, and we will prove that:

    \[
        \varlimsup_{x \to a}\frac{f(x)}{g(x)} \le r
    \]


    since $f'(x)/g'(x) \to A < r$, there exists $c_1$ so that for all $x \in (a,c_1),\: f'(x)/g'(x) \le r$.
    Now let's pick any $x,y \in (a,c_1),\: x < y$, by mean value theorem: \: (We can do so because $g(x) \ne g(y)$)

    \[
        \frac{f(x) - f(y)}{g(x) - g(y)} = \frac{f'(t)}{g'(t)} \le r \quad \text{for some} \: t \in (x,y)
    \]

    let's fix $y$

    If (1) holds:

    take $x \to a$, then we got:

    \[
        \frac{f(y)}{g(y)} \le r
    \]


    since $y \in (a,c_1)$ is arbitrary, take $y \to a$, then we got:

    \[
        \varlimsup_{y \to a}\frac{f(y)}{g(y)} \le r
    \]

    Or if (2) holds:

    since $g(x) \to \infty$ as $x \to a$, we can pick $x$ close enough to $a$, so that

    \begin{align*}
        g(x) &> g(y) \\
        g(x) &> 0 
    \end{align*}

    and we got:

    \begin{align*}
        \frac{f(x) - f(y)}{g(x) - g(y)} & \le r \\
       \frac{f(x) - f(y)}{g(x)} & \le r \frac{g(x) -g(y)}{g(x)} \\
       \frac{f(x)}{g(x)}& \le r + \frac{f(y) - r g(y)}{g(x)}
    \end{align*}

    take $x \to a$, we got

    \[
        \varlimsup_{x \to a} \frac{f(x)}{g(x)} \le  r
    \]

    because $r$ can be arbitrary close to $A$, thus we got:

    \[
        \varlimsup_{x \to a}\frac{f(x)}{g(x)} \le L
    \]

    Similarly, if $-\infty < A \le \infty$, we can pick $-\infty < p < A$, and prove that:

    \[
        \varliminf_{x \to a} \frac{f(x)}{g(x)} \ge  p
    \]

    since $p$ can be arbitrary close to $L$, and thus


    \[
        \varliminf_{x \to a} \frac{f(x)}{g(x)} \ge L
    \]

    after all,
    
    if $L \in \R$, then

    \[
        L \le \varliminf_{x \to a}\frac{f(x)}{g(x)} \le \varlimsup_{x \to a}\frac{f(x)}{g(x)} \le L
    \]


    which indicates:

    \[
        \lim_{x \to a}\frac{f(x)}{g(x)} = A
    \]


    Similarly conclusion holds if $L = \infty$ or $L = -\infty$
\end{proof}

\begin{corollary}
   \autoref{thm:lhopital} also apply to case when:
   
   \begin{enumerate}
    \item $x \to b$

    Or

    \item $g(x) \to -\infty$ as $x \to a$ or $x \to b$
   \end{enumerate}
\end{corollary}

\begin{proof}
    just replace $x \to a$ with $x \to b$, for $g(x) \to -\infty$, prove

    \[
        \lim_{x \to a}\frac{f(x)}{-g(x)} = -A
    \]

    and apply law of limit
\end{proof}


\subsection{Taylor's Theorem}

\begin{thm}
    suppose $f$ is a real function on $[a,b]$, $n$ is a positive integer,
    $f^{(n-1)}$ is continuous on $[a,b]$ and $f^{(n)}(t)$ exists
    for every $t \in (a,b)$. Let $\alpha, \beta$ be distinct points of $[a,b]$
    , and define

    \[
        P(t) = \sum_{k=0}^{n-1}\frac{f^{(k)}(\alpha)}{k!}(t-\alpha)^k
    \]

    Then there exists a point $x$ between $\alpha$ and $\beta$ such that

    \[
        f(\beta) = P(\beta) + \frac{f^{(n)}(x)}{n!}(\beta - \alpha)^n
    \]
\end{thm}

\begin{proof}
    define:

    \[
        g(t) = f(t) - P(t) - M(t-\alpha)^n
    \]

    we apply differentiate for $n$ times, Then:

    \[
        g^{(n)}(t)= f^{(n)}(t) - Mn!
    \]

    when $t = \alpha$, by:

    \begin{align*}
        P^{(m)}(t) &= \sum_{k=m}^{n-1} \frac{f^{(k)}(\alpha)}{(k-m)!}(t-\alpha)^{k-m} \\
        &= \sum_{k=0}^{n-1-m} \frac{f^{(k+m)}(\alpha)}{k!}(t-\alpha)^{k}
    \end{align*}

    thus:

    \[
        P^{(m)}(\alpha) = f^{(m)}(\alpha) \\
    \]

    And

    \begin{align*}
        g^{(k)}(\alpha) &= f^{(k)}(\alpha) - P^{(k)}(\alpha) = 0,\quad \forall k \le n-1
    \end{align*}

    since $g(\beta) = 0$, there exists $\alpha < x_1 < \beta$ so that $g^{(1)}(x_1) = 0$,

    since $g^{(1)}(\alpha) = g^{(1)}(x_1) = 0$, there exists $\alpha < x_2 < x_1$ so that $g^{(2)}(x_2) = 0$,

    after several steps, we got:

    \begin{align*}
        g^{(n-1)}(\alpha) &= g^{(n-1)}(x_{n-1}) = 0 \\
        g^{(n)}(x_n) &= 0 \quad x_n \in (\alpha, x_{n-1})
    \end{align*}

    which indicates

    \[
        M = \frac{f^{(n)}(x_n)}{n!}
    \]

    by replace $t$ as $\beta$, we got

    \[
        f(\beta) = P(\beta) + \frac{f^{(n)}(x_n)}{n!}(\beta - \alpha)^n
    \]
\end{proof}


\begin{thm}
    suppose $\mathbf{f}: [a,b] \to \R^n$ and $\mathbf{f}$ is differentiable
    in $(a,b)$. Then there exists $x \in (a,b)$ such that:

    \[
        \left| \mathbf{f}(b) - \mathbf{f}(a)\right| \le (b-a) \left| \mathbf{f}'(x) \right|
    \]
\end{thm}

\begin{proof}
    define $\mathbf{z} = \mathbf{f}(b) - \mathbf{f}(a)$ and $\phi: [a,b] \to \R$ by:

    \[
        \phi(t) = \langle \mathbf{z}, \mathbf{f}(t) \rangle
    \]

    it is easily to prove that $\phi$ is continuous in $[a,b]$ and differentiable in $(a,b)$, and that:

    \[
        \phi'(t) = \langle \mathbf{z}, \mathbf{f}'(t) \rangle
    \]
    
    by mean theorem,
    there exists $x \in (a,b)$ and

    \begin{align*}
        \phi(b) - \phi(a) &= (b-a)\phi'(x) =(b-a)\langle \mathbf{z}, \mathbf{f}'(x) \rangle \\
        &= \langle \mathbf{z}, \mathbf{f}(b) \rangle - \langle \mathbf{z}, \mathbf{f}(a) \rangle \\
        &= \langle \mathbf{z}, \mathbf{z} \rangle 
    \end{align*}

    by Cauchy inequality:

    \[
        \langle \mathbf{z}, \mathbf{z} \rangle = \left| (b-a)\langle \mathbf{z}, \mathbf{f}'(x) \rangle \right| \le (b-a) \lvert \mathbf{z} \rvert  \lvert \mathbf{f}'(x) \rvert
    \]

    by discuss on $\lvert \mathbf{z} \rvert = 0$ or not, we got

    \[
        \lvert\mathbf{f}(b) - \mathbf{f}(a) \rvert = \lvert \mathbf{z} \rvert \le (b-a)\lvert \mathbf{f}'(x) \rvert
    \]
\end{proof}