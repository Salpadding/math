\section{Sequences and Series of Functions}

\subsection{Definitions}

\begin{definition}[Pointwise Convergence]
    let $\{ f_n \} $ be a sequence of functions defined on set $E$.
    if for every $x \in E$, $f_n(x) \to f(x)$. Then we can define function $f: E \to Y$:

    \[
        f(x) = \lim_{n \to \infty} f_n(x)
    \]
\end{definition}


\begin{definition}[Uniform Convergence]
    let $\{ f_n \} $ be a sequence of functions defined on set $E$.
    We say $\{ f_n \}$ converges to $f$ uniformly iff: for every $\epsilon > 0$, exists $N$,
    and for all $n \ge N$, $\| f_n - f \|_{\infty} \le \epsilon $
\end{definition}

\begin{thm}
    \label{thm:uniform-convergence-by-cauchy-sequence}
    let $Y$ be a complete metric space, if $\{ f_n \}, f_n: E \to Y$ is a cauchy sequence defined by metric:

    \[
        \| f - g\|_\infty = \sup_{x \in E} \left|  f(x) - g(x) \right|
    \]

    then $\{ f_n\}$ converges uniformly
\end{thm}

\begin{proof}

    At first, we prove $f_n$ converges to $f$ pointwise.

    Fix $x \in E$, since $f_n(x)$ is cauchy, and $Y$ is complete, we can define $f: E \to Y$:

    \[
        f(x) = \lim_{n \to \infty}f_n(x)
    \]

    Pick any $\epsilon > 0$, there exists $N$ and for all $n,m \ge N$:

    \[
         \left| f_n(x) - f_m(x) \right| \le \epsilon,\: \forall x \in E
    \]

    let's fix $n$ at first and take $m \to \infty$:

    \[
        \left| f_n(x) - f(x) \right| \le \epsilon,\: \forall x \in E
    \]

    since $x$ is arbitrary, which means

    \[
        \| f_n - f \|_\infty \le \epsilon
    \]


\end{proof}

\begin{thm}
    suppose $\{ f_n\}$ is a sequence of functions defined on $E$, and $M_n \in \mathbb{R}$, suppose

    \[
        \left| f_n(x)\right| \le M_n
    \]

    Then

    \[
        \sum_{n=0}^{\infty} f_n 
    \]
    converges uniformly if $\sum M_n$ converges.
\end{thm}

\begin{proof}
    define partial sum $s_N$:

    \[
        s_N = \sum_{n=0}^{N} f_n
    \]

    we will prove $s_N$ is a cauchy sequence:

    \begin{align*}
        \|s_N -  s_M\|_{\infty} &= \sup_{x \in E} \left| \sum_{n=N+1}^{M} f_n \right| \\
        & \le \sup_{x \in E} \sum_{n=N+1}^{M} \left| f_n  \right| \\
        & \le  \sum_{n=N+1}^{M} M_n 
    \end{align*}

    because:

    \[
        \lim_{N,M \to \infty} \sum_{n=N+1}^{M} M_n = 0
    \]

    by $\sum M_n$ converges, which indicates:

    \[
        \lim_{N,M \to \infty}  \left|s_N - s_M\right|_\infty = 0
    \]

    so $S_N = \sum_{n=0}^{N}f_n$ is a cauchy sequence under metric $\| \|_\infty$, and hence converges uniformly.
\end{proof}

\subsection{Uniform Convergence And Continuity}

\begin{thm}
    suppose $f_n \to f$ uniformly on a set $E$ in a metric space, Let $x$
    be a limit point of $E$. and suppose that:

    \[
        \lim_{t \to x}f_n(t) = A_n
    \]

    Then $\{ A_n \} $ converges, and

    \[
        \lim_{t \to x}f(t) = \lim_{n \to \infty}A_n
    \]

    In other words, we have:

    \[
        \lim_{t \to x} \left(\lim_{n \to \infty}f_n(x) \right) = \lim_{n \to \infty} \left(\lim_{t \to x}f_n(x) \right)
    \]
\end{thm}

\begin{proof}
    given any $x_m \to x, x_m \ne x$, Then:

    We fix $n$ at first:

    \begin{align*}
        \lim_{m \to \infty} f_n(x_m) = A_n
    \end{align*}

    as $n$ be large enough, we have $f_n(x_m) \to f(x_m)$ for any $m$
    
    by pick arbitrary $\epsilon > 0$ and sufficient large $N$, for any $k,p \ge N$, we got

    \begin{align*}
        \lim_{m \to \infty}f_k(x_m) &= A_k \\
        \lim_{m \to \infty}f_p(x_m) &= A_p \\
        \left| A_k - A_p \right| &= \left| \lim_{m \to \infty}f_k(x_m) - \lim_{m \to \infty}f_p(x_m) \right| \\
        & = \left| \lim_{m \to \infty}f_k(x_m) -f_p(x_m) \right| \\
        & = \left| \lim_{m \to \infty}f_k(x_m) -f(x_m) + f(x_m) -f_p(x_m) \right| \\
        & \le 2\epsilon 
    \end{align*}

    which indicates $A_n$ converges.

    Then we define $A$ as

    \[
        \lim_{n \to \infty}A_n = A
    \]

    Now, by pick $\epsilon$ and $N$ like above, consider when $n \ge N$:

    \begin{align*}
        A &= \varlimsup_{n \to \infty} \varlimsup_{m \to \infty}f_n(x_m) \\
            & \ge \varlimsup_{n \to \infty} \varlimsup_{m \to \infty}f(x_m) - \epsilon \\
            & \ge \varlimsup_{m \to \infty}f(x_m) - \epsilon
    \end{align*}

    Similarly, we have:

    \[
       A \le \varliminf_{m \to \infty}f(x_m) + \epsilon 
    \]

    Since $\epsilon$ is arbitrary, we have

    \[
\varliminf_{m \to \infty}f(x_m) =  \varlimsup_{m \to \infty}f(x_m) = A
    \]
\end{proof}

\begin{corollary}
    another proof
\end{corollary}

\begin{proof}
    assume $x_m \to x, x_m \ne x$, when $k,p$ be sufficient large, we got

    \begin{align*}
        \left| a_k - a_p \right| &= \left|\lim_{m \to \infty}f_k(x_m)- \lim_{m \to \infty}f_p(x_m) \right| \\
        & =  \left|\lim_{m \to \infty}f_k(x_m)-f_p(x_m) \right| \\
        & =  \lim_{m \to \infty}\left|f_k(x_m)-f_p(x_m) \right| \\
        & \le \| f_k - f_p\|_\infty
    \end{align*}

    thus $a_n$ is cauchy, and hence converges, let's define

    \[
        a = \lim_{n \to \infty}a_n
    \]

    we cant let $N$ be large enough, so that $\| f_k  - f_p\|_\infty \le \epsilon, \forall k,p \ge N $ let's consider:

    \begin{align*}
        \varlimsup_{m \to \infty} \lim_{n \to \infty}\left|f_n(x_m) - a \right| &=\varlimsup_{m \to \infty} \lim_{n \to \infty}\left|f_n(x_m) - f_N(x_m) + f_N(x_m) -a \right| \\
        & \le \varlimsup_{m \to \infty} \lim_{n \to \infty}\left|f_n(x_m) - f_N(x_m)\right| + \left|f_N(x_m) -a \right| \\
        & \le \epsilon + \left| a_N -a \right|
    \end{align*}

    take $\epsilon \to 0$ with $N \to \infty$, we got

    \[
        \varlimsup_{m \to \infty} \lim_{n \to \infty}\left|f_n(x_m) - a \right| = 0
    \]

    since $x_m$ is an arbitrary sequence, thus

    \[
        \lim_{t \to x}f(x) = a = \lim_{n \to \infty}a_n
    \]
\end{proof}

\begin{corollary}
    \label{thm:uniform-convergence-reserve-continuity}
    If $\{ f_n \}$ is a sequence of continuous functions on $E$, and if $f_n \to f$
    uniformly on $E$, then $f$ is continuous on $E$
\end{corollary}


\begin{thm}
    Suppose $K$ is compact, and 

    \begin{enumerate}
        \item $\{ f_n\}$ is a sequence of continuous functions on $K$

        \item $\{ f_n\}$ converges pointwise to a continuous function $f$ on $K$

        \item $f_n \ge f_{n+1}$ for all $x \in K$
    \end{enumerate}

    Then $f_n \to f$ uniformly on $K$
\end{thm}

\begin{proof}
    define $g_n = f_n -f$, fix $\epsilon > 0$, and define $K_n = \{ x: g_n(x) \ge \epsilon \}$. 
    Then $K_n$ is closed because $g_n$ is continuous and hence compact($K_n \subseteq K$). 

    It is obviously that $K_n \supseteq K_{n+1}$ and

    \[
        \bigcap_{n=1}^{\infty}K_n = \emptyset
    \]

    so there exists $N$ such that $K_N = \emptyset$, and $\forall n \ge N$:

    \begin{align*}
        g_n(x) &< \epsilon \\
        f_n -f  &= \left| f_n - f\right| < \epsilon
    \end{align*}

\end{proof}

\begin{definition}
    if $X$ is a metric space, $\mathscr{C}(X)$ will denote the set of all complex
    valued, continuous, bounded functions with domain $X$
\end{definition}

\begin{thm}
    $\mathscr{C}(X)$ is a  metric space by distance:

    \[
        d(f,g) = \sup_{x \in X} \left| f(x) - g(x) \right|
    \]
\end{thm}

\begin{proof}
    it is easily to verify:

    \begin{enumerate}
        \item $d(f,f) = 0$

        \item $d(f,g) = d(g,f)$

        \item $d(f,h) \le d(f,g) + d(g,h)$

        \item $f = g$ iff $d(f,g) = 0$
    \end{enumerate}
\end{proof}

\begin{thm}
    $\mathscr{C}(X)$ is a complete metric space.
\end{thm}

\begin{proof}
    by \autoref{thm:uniform-convergence-by-cauchy-sequence} and \autoref{thm:uniform-convergence-reserve-continuity}
\end{proof}

\subsection{Uniform Convergence and Integration}

\begin{thm}
    let $\alpha$ be monotonically increasing on $[a,b]$. Suppose $f_n \in \mathscr{R}(\alpha)$ on $[a,b]$,
    for $n =1,2,3,..$ and suppose $f_n \to f$ uniformly on $[a,b]$.
    Then $f \in \mathscr{R}(\alpha)$ on $[a,b]$, and

    \[
        \int_a^b \left( \lim_{n \to \infty} f_n\right) \mathrm{d} \alpha =  \lim_{n \to \infty} \left(\int_a^b f_n \mathrm{d} \alpha \right)
    \]

\end{thm}

\begin{proof}
    Steps:

    \begin{enumerate}
        \item  We prove $f$ is integrable at first:

    let $N$ be large enough so that $\| f_n - f \|_\infty \le \epsilon, \: \forall n \ge N$, and $P$ be 
    arbitrary partition of $[a,b]$, them

    \[
        U(P,f,\alpha) - L(P,f,\alpha) \le U(P, f_n, \alpha) - L(P,f_n,\alpha) + 2\epsilon(\alpha(b) - \alpha(a))
    \]

    since $P$ could be arbitrary and $f_n$ is integrable, then:


    \[
        \inf_{P}\left(U(P,f,\alpha) - L(P,f,\alpha) \right) \le   2\epsilon(\alpha(b) - \alpha(a))
    \]

    take $\epsilon \to 0$, we got

    \[
        \inf_{P}\left(U(P,f,\alpha) - L(P,f,\alpha) \right) = 0
    \]

    which shows $f$ is integrable.

        \item define:

        \[
            S_n = \int_a^b f_n \mathrm{d} \alpha
        \]

        we will prove $S_n$ converges, consider $\epsilon > 0$, and $N$ large enough so that 
        $\| f_n - f_m\|_\infty \le \epsilon, \forall n,m \ge N$, Then

        \begin{align*}
            \left| S_n - S_m\right| &= \left| \int_a^b \left(f_n - f_m\right) \mathrm{d} \alpha\right| \\
            & \le \int_a^b \left|f_n - f_m\right| \mathrm{d} \alpha \\
            & \le  \epsilon(\alpha(b) - \alpha(a))
        \end{align*}

        since $\epsilon$ is arbitrary, so $S_n$ is cauchy sequence, and hence converges

    \item let's define

    \[
        S = \int_a^b f \mathrm{d} \alpha
    \]
    and consider $\left| S_n - S\right|$ when $n$ be sufficient large:

    \begin{align*}
        \left|\int_a^b f_n  - \int_a^b f \mathrm{d} \alpha \right| & = \left|\int_a^b \left(f_n - f\right) \mathrm{d} \alpha \right| \\
        & \le  \int_a^b \left|f_n - f\right| \mathrm{d} \alpha \\
        & \le \epsilon(\alpha(b) - \alpha(a))
    \end{align*}
        
    which shows $S_n \to S$
    \end{enumerate}

\end{proof}

\subsection{Uniform Convergence and Differentiation}

\begin{thm}
    suppose $\{ f_n \}$ is a sequence of differentiable functions on $[a,b]$.
    And exists $x_0 \in [a,b]$ such that $f_n(x_0) \to y_0$ for some $y_0$. If $\{ f_n' \}$
    converges uniformly on $[a,b]$, then $\{ f_n\}$ converges uniformly on $[a,b]$, and

    \[
        \left( \lim_{n \to \infty}f_n \right)' = \lim_{n \to \infty} f_n'
    \]
\end{thm}

\begin{proof}
    Steps:
    
    \begin{enumerate}
        \item we prove $f_n$ converges uniformly at first

        define $h = f_n - f_m$, fix $x \in [a,b]$ by mean value theorem, we got

        \begin{align*}
            h(x) - h(x_0) &= h'(\xi)(x - x_0) \\
            f_n(x) -  f_m(x) - \left(f_n(x_0) - f_m(x_0) \right) &= \left(f_n'(\xi) -f_m'(\xi) \right)(x-x_0)
        \end{align*}

        when $n,m$ are both large enough, by:
        
        \begin{align*}
            \left| f_n(x_0) - f_m(x_0) \right| & \le \epsilon \\
            \left| f'_n(\xi) - f'_m(\xi) \right| & \le \epsilon \\
            \left| x -x_0 \right| & \le (b-a) 
        \end{align*}

        we got:

        \[
            \lim_{n,m \to \infty} \| f_n - f_m\|_\infty = 0
        \]

        so $f_n$ converges to function $f$ uniformly.

        \item prove $f$ is differentiable
        
        we fix $x \in [a,b]$, define function $h_n$ as:

        \[
            h_n(t) = \begin{cases}
                (f_n(t) - f_n(x)) / (t- x) & t \ne x \\
                f_n'(x) & t = x
            \end{cases}
        \]

        since $f_n$ is differentiable on $[a,b]$, then $h_n$ is continuous.

        And when $n,m$ are both large enough, for $t \ne x$:

        \begin{align*}
            \left|h_n(t) - h_m(t) \right| =  \left|  \frac{f_n(t) - f_n(x) - \left( f_m(t) - f_m(x) \right)}{t-x}\right|
        \end{align*}

        by mean value theorem, we have:

        \begin{align*}
            f_n(t) - f_n(x) - \left( f_m(t) - f_m(x) \right) &= f_n(t) - f_m(t) - \left( f_n(x) - f_m(x) \right) \\
            &= (t-x)\left( f_n'(\xi) - f_m'(\xi) \right)
        \end{align*}

        so

        \begin{align*}
            \left|h_n(t) - h_m(t) \right| = \left| f_n'(\xi) - f_m'(\xi) \right|
        \end{align*}

        so $h_n$ converges uniformly. And

        \[
            h_n(t) \to h = \begin{cases}
                (f(t) - f(x))/(t-x) & t \ne x \\
                \lim_{n \to \infty} f_n'(x) & t = x
            \end{cases}
        \]

        since $h_n$ is continuous and $h_n$ converges uniformly, hence $h$ is also continuous. And

        \[
            \lim_{t \to x}\frac{f(t) - f(x)}{t-x} = f'(x) = \lim_{n \to \infty} f_n'(x)
        \]
    \end{enumerate}
\end{proof}

\begin{thm}
    \label{thm:pointwise-bounded-on-countable-has-subsequence}
    if $\{ f_n \}$ is a pointwise bounded sequence of complex functions
    on a countable set $E$, then $\{f_n\}$ has a subsequence $\{ f_{n_k} \}$ 
    such that $f_{n_k}$ converges pointwise.
\end{thm}

\begin{proof}
   let $E = \{  x_1,x_2,...\}$ 
   
   Since $f_n(x_1)$ is bounded, there exists a subsequence, which we shall
   denote by $\{ f_{1,k}\}$ such that $\{ f_{1,k}(x_1)\}$ converges as $k \to \infty$

   Let us now consider sequences $S_1, S_2, S_3,...$, which we represent by the array.

   \begin{align*}
    \begin{bmatrix}
    S_1 : &  f_{1,1}  & f_{1,2} & f_{1,3} & f_{1,4} & ... \\
    S_2 : &  f_{2,1}  & f_{2,2} & f_{2,3} & f_{2,4} & ...\\
    S_3 : &  f_{3,1}  & f_{3,2} & f_{3,3} & f_{3,4} & ...\\
    ... & ... & ... & ...  & ...  & ... 
    \end{bmatrix}
   \end{align*}

   and:

   \begin{enumerate}
    \item $S_n$ is a subsequence of $S_{n-1}$
    \item $\{f_{n,k}(x_n)\}$ converges as $k \to \infty$
   \end{enumerate}

   Then $f_{n,n}$ converges pointwise.

\end{proof}

\begin{thm}
    If $K$ is a compact metric space, if $f_n \in \mathscr{C}(K)$ for $n = 1,2,3,...$
    and if $\{ f_n\}$ converges uniformly on $K$. then $\{ f_n\}$ is equicontinuous on $K$.
\end{thm}

\begin{proof}
    fix $\epsilon > 0$, since $\{ f_n\}$ converges uniformly, there exists integer $N$
    such that for all $n \ge N$

    \[
        \| f_n - f_N\|_\infty \le \epsilon
    \]

    Since continuous functions are uniformly continuous on compact sets, there is a $\delta > 0$ such that

    \[
        \left| f_i(x) - f_i(y) \right| \le \epsilon
    \]

    for all $1 \le i \le N$ and $d(x,y) \le \delta$

    If $n \ge N$ and $d(x,y) \le \delta$, then:

    \begin{align*}
        \left| f_n(x) - f_n(y) \right| &= \left| f_n(x) - f_N(x) + f_N(y) - f_n(y) + f_N(x) - f_N(y)\right| \\
        & \le \left| f_n(x) - f_N(x) \right| + \left| f_N(y) - f_n(y) \right|  +\left| f_N(x) - f_N(y) \right| \\
        & \le 3\epsilon
    \end{align*}
\end{proof}

\begin{thm}
    if $K$ is compact, if $f_n \in \mathscr{C}(K)$ for $n= 1,2,3,...,$
    and if $\{ f_n\}$ is pointwise bounded and equicontinuous on $K$, then

    \begin{enumerate}
        \item $\{ f_n\}$ is uniformly bounded on $K$

        \item $\{ f_n\}$ contains a uniformly convergent subsequence
    \end{enumerate}
\end{thm}

\begin{proof}
    Steps:

    \begin{enumerate}
        \item we prove $\{ f_n\}$ is uniformly bounded at first
    by definition of equicontinuous, for any $\epsilon > 0$, there exists $\delta > 0$,
    such that:

    \[
        \left| f_n(x) - f_n(y) \right| \le \epsilon
    \]

    for all $n = 1,2,3,...$ and $x,y \in K, d(x,y) \le \delta$

    because $K$ has a open cover:

    \[
        K \subseteq \bigcup_{p \in K}B(p, \delta)
    \]

    since $K$ is compact, it should contains a finite sub cover:

    \[
        K \subseteq \bigcup_{k=1}^{N}B(p_k, \delta)
    \]

    since $f_n$ is pointwise bounded, assume:

    \begin{align*}
        \left| f_n(p_k) \right| \le M_k \quad (1 \le k \le N)
    \end{align*}

    And define:

    \[
        M = \max_{1 \le k \le M}M_k
    \]

    for any $x \in K$, by:

    \[
        x \in \bigcup_{k=1}^{N}B(p, \delta)
    \]
    we got $| x - p_s| \le \epsilon$ for some $s \in [1,N] \cap \N$, and $\left|p_s \right| \le M$,
    which indicates $|x| \le M + \epsilon$ for all $x \in K$

    \item prove a lemma: $K$ has a countable dense subset $E$, which means $\overline{E} = K$

    we pick $n = 1,2,3,...$ and let $K$ be covered by finite open sets:

    \[
        K \subseteq \bigcup_{\alpha \in I_n} B(x^{(n)}_{\alpha}, 1/n)
    \]

    where $I_n$ is finite.

    let's consider $E$

    \[
        E = \bigcup_{n=1}^{\infty}\bigcup_{\alpha \in I_n} B(x^{(n)}_{\alpha}, 1/n)
    \]

    then $E$ is countable, pick $y \in K$ and $\epsilon > 0$, there should exists $n$
    so that $1/n < \epsilon$, and hence $E$ contains an element $x^{(n)}_{\alpha}$ so that 

    \[
        d(x^{(n)}_{\alpha}, y) \le 1/n \le \epsilon
    \]

    since $\epsilon$ is arbitrary, which means $y \in \overline{E}$, and hence $K \subseteq \overline{E}$.

    \item we prove remained part

    by \autoref{thm:pointwise-bounded-on-countable-has-subsequence}, $\{ f_n\}$
    has a subsequence $\{ f_{n_i}\}$ converges for every $x \in E$.

    let $\epsilon > 0$ and $\delta > 0$ for equicontinuous of $\{ f_n\}$.

    since $E$ is dense in compact set $K$, there are finitely many points $x_1,x_2,...,x_m \in E$, such that

    \[
        K \subseteq B(x_1,\delta) \cup B(x_2, \delta) \cup ... \cup B(x_m, \delta)
    \]

    since $f_{n_i}$ converges for every $x_s \in E$, and $m$ is finite, there is a integer $N$ such that

    \[
        \left| f_{n_i}(x_s) - f_{n_j}(x_s) \right| \le \epsilon
    \]

    for any $x_s \in \{ x_1,x_2,x_3,...,x_m \}$.

    Then for any $x \in K$, when $i,j \ge N$, consider

    \begin{align*}
        d(x, x_s) & \le \delta \\
        \left| f_{n_i}(x) - f_{n_i}(x_s) \right| & \le \epsilon \\
        \left| f_{n_j}(x) - f_{n_j}(x_s) \right| & \le \epsilon \\
        \left| f_{n_i}(x_s) - f_{n_j}(x_s) \right| &\le \epsilon
    \end{align*}

    combine above, we got:

    \[
        \left| f_{n_i}(x) - f_{n_j}(x) \right|  \le 3\epsilon \\
    \]

    since $x$ is arbitrary, so $f_{n_i}$ converges uniformly.

    \end{enumerate}
\end{proof}

\begin{thm}[existence of polynomial kernel]
    There exists $c_n \ge 0$ meets:

    \begin{enumerate}
        \item

        \[
            \int_{[-1,1]}c_n(1-x^2)^n \mathrm{d}x = 1
        \]

        \item $\forall \epsilon > 0, 0 < \delta \le 1$, exists $N$, for all $n \ge N$

        \[
            \int_{[-1,1]}c_n(1-x^2)^n \mathrm{d}x = 1
        \]

        and

        \[
            c_n(1-x^2)^n \le \epsilon \quad \forall x \in [\delta, 1]
        \]
        

    \end{enumerate}
\end{thm}

\begin{proof}
    Steps

    \begin{enumerate}
        \item we prove $c_n$ exists, and $c_n \le (n+1)/2$


    consider:

    \begin{align*}
        \int_{[-1,1]}(1-x^2)^n\mathrm{d}x &\ge \int_{[-1,1]}(1-|x|)^n\mathrm{d}x \\
        & \ge 2 \frac{1}{n+1}
    \end{align*}

    if we assign:


    \[
        c_n = \left(\int_{[-1,1]}(1-x^2)^n\mathrm{d}x \right)^{-1}
    \]

    then we got $c_n \le \frac{n+1}{2}$

    \item prove $c_n(1-x^2)^n \to 0, \forall 0 < |x| \le 1$:
    
    consider any $0 < |x| \le 1$, then we got $0 \le 1-x^2 < 1$ and
    $c_n(1-x^2)^n \le \frac{n+1}{2}(1-x^2)^n$

    let $\epsilon = 1-x^2$ and $\epsilon < 1$, then we got

    \[
        \varlimsup_{n \to \infty}c_n(1-x^2)^n \le \varlimsup_{n \to \infty}\frac{n+1}{2}\epsilon ^n 
    \]

    because by ratio test:

    \[
         \varlimsup_{n \to \infty}\frac{n+2}{n+1}\epsilon = \epsilon < 1
    \]

    which shows 

    \[
        \varlimsup_{n \to \infty}c_n(1-x^2)^n \le \varlimsup_{n \to \infty}\frac{n+1}{2}\epsilon ^n = 0
    \]

    \item we fix $\epsilon$ and $\delta$ at first.
    
    by $c_n(1-\delta^2)^n \to 0$, there should exists $N$, and forall $n \ge N$

    \[
        c_n(1-\delta^2)^n \le \epsilon
    \]

    let's consider when $|x| \ge |\delta|$ we got $1-x^2 \le 1-\delta^2$ and hence

    \[
        c_n(1-x^2)^n \le c_n(1-\delta^2)^n \le \epsilon
    \]

    \end{enumerate}


\end{proof}


\begin{thm}
    let $f_n: \mathbb{R} \to [0, \infty)$ meets:

    \begin{enumerate}
        \item $f_n$ is continuous and supported on $[-1,1]$

        which means $\forall |x| > 1,\: f_n(x)= 0$
        

        \item $f_n$ approximate Dirac function


$\forall \epsilon > 0$ and $ 0 < \delta \le 1$, exists $N$, and for all $n \ge N$:

        \begin{enumerate}
            \item $\forall \delta \le |x| \le 1,\: f_n(x) \le \epsilon$

            \item integration


    \[
        \int_{\mathbb{R}} f_n(x) \mathrm{d}x = \int_{-1}^{1} f_n(x) \mathrm{d}x = 1
    \]
        \end{enumerate}


    \end{enumerate}


let $g: \mathbb{R} \to \mathbb{R}$ be continuous and supported on $[0,1]$, prove:

\[
    f_n \ast g \to g \quad \mathrm{uniformly}
\]
\end{thm}

\begin{proof}
    Steps:

    let $|g(x)| \le M$, since $g$ is continuous on $[0,1]$, hence bounded and uniformly continuous.

    we pick any $\epsilon > 0$ 
    and $\delta > 0$ so that 
    for all $x,y \in [0,1],\: |x-y| \le \delta $  results in $|g(x) -g(y)| \le \epsilon$

    by definition of $f_n$, we pick $N$, so that for all $n \ge N$, 
    $\forall  \delta \le |x| \le 1,\:  f(x) \le \epsilon$, 
    then we got

    \begin{align*}
        \left| (f_n \ast g)(x) - g(x) \right| &= \left| \int_{\mathbb{R}}f_n(t)g(x-t) \mathrm{d}t  - g(x)\right| = \left| \int_{[-1,1]}f_n(t)g(x-t) \mathrm{d}t  - \int_{[-1,1]}g(x) f_n(t) \mathrm{d}t \right| \\ 
        & = \left| \int_{\mathbb{R}}f_n(t)g(x-t) \mathrm{d}t  - g(x)\right| = \left| \int_{[-1,1]}f_n(t)\left[g(x-t) -g(x) \right]\mathrm{d}t   \right| \\
        & \le \left| \int_{[-1,-\delta]}f_n(t)\left[g(x-t) -g(x) \right]\mathrm{d}t   \right| + \left| \int_{[-\delta,\delta]}f_n(t)\left[g(x-t) -g(x) \right]\mathrm{d}t   \right| \\
        & + \left| \int_{[\delta,1]}f_n(t)\left[g(x-t) -g(x) \right]\mathrm{d}t   \right|\\
        & \le 4M\epsilon(1-\delta) + 2\epsilon \int_{[-\delta,\delta]}f_n(t)\mathrm{d}t    \\
        & \le 4M\epsilon(1-\delta) + 2\epsilon
    \end{align*}

    since $\epsilon, \delta \to 0$ as $n \to \infty$, we got

    \[
        (f_n \ast g)(x) \to g(x)
    \]


\end{proof}

\begin{thm}
    convolution of two compact supported function $f, g$ is defined as:

    \[
        f \ast g (x) = \int_{-\infty}^{\infty} f(t) g(x-t) \mathrm{d}t
    \]

    and we have

    \[
        f \ast g = g \ast f
    \]

\end{thm}

\begin{proof}
    let $u(t) = x-t$
   \begin{align*}
    \int_{\R} f(t) g(x-t) \mathrm{d}t &=\int_{-\infty}^{\infty}f(x-u)g(u) \mathrm{d} \left(- u \right) \\
    &= -\int_{-\infty}^{\infty}f(x-u)g(u) \mathrm{d} \left( u \right) \\
    &= -\int_{\infty}^{-\infty}f(x-t)g(t) \mathrm{d} t \\
    &= \int_{-\infty}^{\infty}f(x-t)g(t) \mathrm{d} t \\
    &= g \ast f
   \end{align*} 
\end{proof}

\begin{thm}[Stone-Weierstrass Theorem]
    If $f$ is a continuous function on $[a,b]$, there exists a sequence
    of polynomial $P_n$ such that

    \[
        \lim_{n \to \infty}P_n = f
    \]

    uniformly on $[a,b]$
\end{thm}

\begin{proof}
    we will prove if $g$ is polynomial on $[-1,1]$ and $f$ is continuous function supported on $[0,1]$.
    Then $f \ast g$ is a polynomial on $[0,1]$.

    define $g$ as:

    \[
        g(x) = \sum_{j=0}^{n}c_j x^j
    \]

    then

    \begin{align*}
        \int_{-\infty}^{\infty}f(t)g(x-t) \mathrm{d}t &= \int_{0}^{1}f(t)g(x-t) \mathrm{d}t \\
        &= \int_{0}^{1}f(t)\sum_{j=0}^{n}c_j (x-t)^j \mathrm{d}t \\
        &= \int_{0}^{1}f(t)\sum_{j=0}^{n}c_j \left(\sum_{i=0}^{j}\binom{j}{i}x^{i}(-t)^{j-i} \right) \mathrm{d}t \\
        &= \sum_{j=0}^{n}\sum_{i=0}^{j}  c_j \binom{j}{i} x^i \int_{0}^{1} f(t)(-t)^{j-i} \mathrm{d}t
    \end{align*}

    then we create function $h: [0,1] \to g([a,b])$:

    $h(x)= g(a + (b-a)x)$

    by above theorems,there exists polynomial kernel $P_n$ such that

    \[
        P_n \ast h \to h
    \]

    and $P_n \ast h$ is also a kernel, then

    \[
        P_n \ast h ((x-a)/(b-a))
    \]

    is what we want
\end{proof}

\subsection{Algebra of Functions}

\begin{definition}
    A family $\mathscr{A}$ of complex functions defined on a set $E$
    is said to be algebra if:

    \begin{enumerate}
        \item $f+g \in \mathscr{A}$
        \item $fg \in \mathscr{A}$
        \item $cf \in \mathscr{A}, \forall c \in \C$
    \end{enumerate}

\end{definition}

\begin{thm}
    let $\mathscr{B}$ be the uniform closure of an algebra $\mathscr{A}$
    of bounded functions. Then $\mathscr{B}$ is uniformly closed.
\end{thm}

\begin{proof}
    let $f, g \in \mathscr{B}$, there exists uniformly convergent sequences
    $\{ f_n\} \in \mathscr{A}$ and $\{ g_n\} \in \mathscr{A}$ such that $f_n \to f, g_n \to g$ uniformly

    by $f_n + g_n \to f+g$, $f_ng_n \to fg, cf_n \to cf$, $\mathscr{B}$ is an algebra.

    let's assume $f_n \in \mathscr{B}$ and $f_n \to f$ uniformly. we will prove $f \in \mathscr{B}$

    consider:

    \begin{align*}
        g^{(1)}_k &\to f_1 \\
        g^{(2)}_k &\to f_2 \\
        g^{(3)}_k &\to f_3 \\
        .. & .. \\
    \end{align*}

    where $k=1,2,3,...$ and $g^{(n)}_k \in \mathscr{A}$
    
    we select $k_1, k_2, k_3, ...$ as:

    \begin{align*}
        k_{i} &> k_{i-1} \\
        d(g^{(i)}_{k_i}, f_i) & \le 1/i
    \end{align*}

    let's consider $g^{(i)}_{k_i}$, when $i$ is large enough, we have

    \begin{align*}
        d(f_i, f) & \le \epsilon \quad \forall i \ge N \\
        d(g^{(i)}_{k_i}, f_i) & \le 1/i  \quad \forall i \ge N\\
    \end{align*}

    which indicates

    \[
        d(g^{(i)}_{k_i}, f)  \le \epsilon + 1/i  \quad \forall i \ge N\\
    \]
\end{proof}

\begin{definition}
    let $\mathscr{A}$ be a family of functions on a set $E$. Then $\mathscr{A}$
    is said to separate points on $E$ if to every distinct pair $x_1,x_2 \in E$. 
    there corresponds a function $f \in \mathscr{A}$ such that $f(x_1) \ne f(x_2)$.

    If to each $x \in E$ there corresponds a function $g$ such that $g(x) \ne 0$,
    we say that $\mathscr{A}$ vanishes at no point of $E$
\end{definition}

\begin{thm}
    suppose $\mathscr{A}$ is an algebra of functions on a set $E$, $\mathscr{A}$ separates
    points on $E$, and $\mathscr{A}$ vanishes at no point of $E$

    suppose $x_1,x_2$ are distinct points of $E$, and $c_1,c_2$ are constants.
    Then $\mathscr{A}$ contains a function $f$ such that:

    \[
        f(x_1) = c_1, f(x_2) = c_2
    \]

\end{thm}

\begin{proof}
    The assumption show that $\mathscr{A}$ functions $g,h,k$ such that
    $g(x_1) \ne g(x_2), h(x_1) \ne 0, k(x_2) \ne 0$

    Put

    \begin{align*}
        u &= gk - g(x_1)k \\
        v &= gh - g(x_2) h
    \end{align*}

    Then $u,v \in \mathscr{A}, u(x_1) = v(x_2) = 0$ and $u(x_2) \ne 0, v(x_1) \ne 0$

    Therefore

    \[
        f = c_1 \frac{v}{v(x_1)}  + c_2 \frac{u}{u(x_2)}
    \]

    is what we want.
\end{proof}

\begin{thm}
    let $\mathscr{A}$ be an algebra of real continuous functions on a compact set $K$.
    If $\mathscr{A}$ separates points on $K$ and if $\mathscr{A}$ vanishes at no point of $K$,
    then the uniform closure $\mathscr{B}$ of $\mathscr{A}$ consists of all real continuous 
    functions on $K$
\end{thm}

\begin{proof}
    \begin{enumerate}
        \item we prove $|f| \in \mathscr{B}, \forall f \in \mathscr{B}$

        since $|f|$ is bounded, let $|f| \le M$, and let polynomial $P_n$ converges to 
        $|x|$ on $[-M, M]$. Then $P_n \circ f \to |f|$ 

        \item if $f,g \in \mathscr{B}$, then $\max(f,g)$ and $\min(f,g) \in \mathscr{B}$

        by:
        \begin{align*}
            \max(f,g) &= \frac{f+g}{2} + \frac{\left| f-g \right|}{2} \\
            \min(f,g) &= - \max(-f, -g)
        \end{align*}

        \item given real continuous function on $K$, a point $x \in K$, and $\epsilon > 0$, 
        there exists a function $g_x \in \mathscr{B}$ such that $g_x(x) = f(x)$ and 

        \[
            g_x(t) \ge f(t) - \epsilon
        \]

        by hypothesis, for any $y \in K$ there exists $\varphi_y \in \mathscr{B}$ and $\varphi_y(x) = f(x),\: \varphi_y(y) = f(y)$

        since $f, \varphi_y$ is continuous at $y$, there exists a open neighbor $U_y$ contains $y$, and $\forall t \in U_y, \varphi_y(t) \ge f(y) - \epsilon$ and
        $f(t) \le f(y) + \epsilon$

        since $K$ is compact, and $K$ has an open cover:

        \[
            K \subseteq \bigcup_{y \in K} U_y
        \]

        and hence has a finite sub cover $U_{y_1}, U_{y_2}, ..., U_{y_n}$. 

        let define $\varphi = \max(\varphi_{y_1}, \varphi_{y_2}, ..., \varphi_{y_n})$. 

        It is easily to verify that $\varphi(x) = f(x)$. And for any $t \in K$, without loss of generality, assume it should belongs to $\varphi_{y_1}$.
        Then $\varphi(t) \ge \varphi_{y_1}(t) \ge f(y_1) - \epsilon \ge f(t) - 2\epsilon$.

        Since $\epsilon$ can be arbitrary small, thus we proved.

        \item given a real function $f$, continuous on $K$. and $\epsilon > 0$, 
        there exists a function $h \in \mathscr{B}$ such that

        \[
            \left| h(x) -f(x) \right| \le \epsilon
        \]

        for every $x \in K$, by (3) we can construct $g_x$ such that $g_x(x) = f(x), \: \forall t \in K,\: g_x(t) \ge f(t) - \epsilon$

        since $g_x, f$ is continuous at $x$, there exists $V_x$ such that $\forall t \in V_x, g_x(t) \le f(x) + \epsilon/2,\: f(t) \ge f(x) -\epsilon/2$. 

        since $K$ is compact, we can apply finite sub cover theorem:

        \begin{align*}
            K &\subseteq \bigcup_{x \in K} V_x \\
        \end{align*}

        thus $K$ has finite sub cover $V_{x_1}, V_{x_2}, ..., V_{x_m}$.

        let's define $h$ as:

        \[
            h = \min(g_{x_1}, g_{x_2}, ..., g_{x_m})
        \]

        Now for any $t \in K$, it is easily to verify that $h(t) \ge f(t) - \epsilon$. 

        Without loss of generality, assume $t \in V_{x_1}$, thus

        \[
            h(t) \le g_{x_1}(t) \le f(x_1) + \epsilon / 2 \le f(t) + \epsilon
        \]
    \end{enumerate}
\end{proof}
