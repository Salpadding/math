\section{Abstract Integration}

\subsection{Measurability}

\begin{definition}
    A collection $\tau$ of subsets of a set $X$ is said to be a topology in $X$ if $\tau$ has the following three properties:

    \begin{enumerate}
        \item If $\tau$ is a topology in $X$, then $X$ is called a topological space, and the members of $\tau$ are called the open sets in $X$.
        \item If $X$ and $Y$ are topological spaces and if $f$ is a mapping of $X$ into $Y$ ,then $f$ is said to be continuous provided that $f ^ { - 1 } ( V )$ is an open set in $X$ for every open set $V$ in $Y$, $f$ is said to be 
        continuous at $x_0$ if for every open set $V$ contains $f(x_0)$, there exists a open set $W$ contains $x_0$  such that $f(W) \subseteq V$.
    \end{enumerate}

   

\end{definition}

\begin{thm}
    $f: X \to Y$, where $X,Y$ are both topological space. Then $f$ is continuous iff it is continuous at every point of $X$.
\end{thm}

\begin{proof}
    for any $V \subseteq Y$, we will prove $f^{-1}(V)$ is open set of $X$. Assume $x \in f^{-1}(V)$, by local continuity, there exists $W_x$
    such that $f(W_x) \subseteq V$. And we have:
    
    \[
        f^{-1}(V) = \bigcup_{x \in f^{-1}(V)}W_x
    \]

    since arbitrary union of open sets is open, so $f^{-1}(V)$ is open.

    Also, if $f$ is continuous. For any $x_0 \in X$, and any $V$ contains $f(x_0)$, we have $f^{-1}(V)$ is open, which 
    contains $x_0$. And consider:

    \[
        f(f^{-1}(V)) \subseteq V
    \]
\end{proof}

\begin{definition}
    A collection $\mathfrak { M }$ of subsets of a set $X$ is said to be a $\sigma$-algebra 
    in $X$ if $\mathfrak { M }$ has the following properties:

\begin{enumerate}
    \item $X \in { \mathfrak { M } }$
    \item If $A \in { \mathfrak { M } }$ ,then $A^C \in \mathfrak { M }$
    \item If $A = \bigcup_{ n = 1 } ^ { \infty } A _ { n }$ and if $A _ { n } \in \mathfrak { M }$ for $n = 1 , 2 , 3 , \ldots ,$ then $A \in { \mathfrak { M } }$   
\end{enumerate}

If $\mathfrak { M }$ is a $\sigma$-algebra in $X$ then $X$ is called a measurable space, and the members of $\mathfrak { M }$ are called the measurable sets in $X$   
If $X$ is a measurable space, $Y$ is a topological space,and $f$ is a mapping of $X$ into $Y$, then $f$ is said to be measurable provided that $f ^ { - 1 } ( V )$ is a measurable set in $X$ for every open set $V$ in $Y$.

\end{definition}

\begin{thm}
If $f$ is a complex measurable function on $X$; 
there is a complex measurable function $ \alpha $ on $X$ 
such that $| \alpha | = 1$ and $f = \alpha \vert f \vert$
\end{thm}

\begin{proof}
Let $E = \{ x \colon f ( x ) = 0 \}$,
let $Y$ be the complex plane with the origin removed, 
define $\varphi ( z ) = z / | z |$ for $z \in  Y$,and put

$$
\alpha ( x ) = \varphi ( f ( x ) + \chi _ { E } ( x ) ) \qquad ( x \in X ) .
$$

If $x \in E$,$\alpha ( x ) = 1$; if $x \notin E$,$\alpha ( x ) = f ( x ) / \mid f ( x ) \mid$.
Since $\varphi$ is continuous on Y and
since $E$ is measurable, 
the measurability of $\alpha$ follows continuous composition.
\end{proof}

\begin{thm}
If $\mathscr { F }$ is any collection of subsets of $X$,
there exists a smallest $\sigma$-algebra ${ \mathfrak { M } } ^ { * }$ 
in $X$ such that ${ \mathscr { F } } \subset { \mathfrak { M } } ^ { * }$

This $\mathfrak { M } ^ { * }$ is 
sometimes called the $\sigma$ -algebra generated by $\mathscr { F }$

\end{thm}

\begin{proof}
     Let $\Omega$ be the family of all $\sigma$-algebras $\mathfrak { M }$ 
     in $X$ which contain $\mathscr { F }$.
     Since the collection of all subsets of $X$ is such a $\sigma$-algebra,
     $\Omega$ is not empty. Let $\mathfrak { M } ^ { \ast }$ be the intersection 
     of all $\mathfrak { M } \in \Omega$ It is clear that ${ \mathscr { F } } \subset { \mathfrak { M } } ^ { * }$ 
     and that $\mathfrak { M } ^ { \ast }$ lies in every $\sigma$-algebra 
     in $X$ which contains $\mathscr { F }$. 
     To complete the proof, we have to show that $\mathfrak { M } ^ { * }$ 
     is itself a $\sigma$-algebra.
\end{proof}

\subsection{Properties Of Measures}

\begin{definition}
    measure is defined as $\mu: \mathfrak{M} \to [0, \infty]$, where $\mathfrak{M}$ is a $\sigma$-algebra.
    \begin{enumerate}
        \item countable additivity

        \[
            \mu\left(\bigcup_{n=1}^{\infty} A_n\right) = \sum_{n=1}^{\infty}\mu(A_n)
        \]

        \item there should exists at least $A \in \mathfrak{M}$ such that

        \[
            \mu(A) < \infty
        \]
    \end{enumerate}
\end{definition}


\begin{thm}
    Let $\mu$ be a positive measure on a $\sigma$-algebra $\mathfrak{M}$. Then

    \begin{enumerate}
        \item $\mu(\emptyset) = 0$

        \item \[
            \mu\left( \bigcup_{k=1}^{n}A_k \right) = \sum_{k=1}^{n} \mu(A_k)
        \]

        if $A_k$ are pairwise disjoint

        \item if $A \subseteq B$ then $\mu(A) \le \mu(B)$

        \item \[
            \lim_{n \to \infty}\mu(A_n) = \mu\left( \lim_{n \to \infty}A_n\right)
        \]

        if $A_1 \subseteq A_2 \subseteq A_3 ...$ and $A_k \in \mathfrak{M}$


        \item \[
            \lim_{n \to \infty}\mu(A_n) = \mu\left( \lim_{n \to \infty}A_n\right)
        \]

        if $A_1 \supseteq A_2 \supseteq A_3 ...$ and $A_k \in \mathfrak{M}$ and $\mu(A_1) < \infty$
    \end{enumerate}
\end{thm}

\begin{proof}
    proofs:

    \begin{enumerate}
        \item $\mu(\emptyset) = 0$

        by

        \[
            \mu\left(\emptyset\right) = \sum_{n=1}^{\infty}\mu\left(\emptyset\right)
        \]

        \item consider

        let $B_1 = A_1, B_2 = A_2,..., B_n =A_n, B_{n+1} = \emptyset, B_{n+2} = \emptyset ...$

        \item consider:

        \begin{align}
            \mu(B) &= \mu(B \setminus A) + \mu(B \cap A) \\
            &= \mu(B \setminus A) + \mu(A) \ge \mu(A)
        \end{align}

    \end{enumerate}
\end{proof}

