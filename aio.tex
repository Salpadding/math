%!LW recipe=latexmk
\documentclass[11pt,a4paper]{article}
% math 
\usepackage{amsmath,amsfonts,amssymb,amsthm,mathrsfs}
% cross reference, use \autoref instead of \ref
\usepackage{aliascnt}
\usepackage[hidelinks]{hyperref}
\usepackage{enumitem}
\usepackage{geometry}


\geometry{left=1.5cm, right=1.5cm, top=2cm, bottom=2cm}

\newtheorem{thm}{Theorem}[subsection]

\newaliascnt{lem}{thm}
\newaliascnt{prop}{thm}
\newaliascnt{definition}{thm}
\newaliascnt{exercise}{thm}
\newaliascnt{corollary}{thm}
\newaliascnt{remark}{thm}

\theoremstyle{definition}
\newtheorem{lem}[lem]{Lemma}
\newtheorem{prop}[prop]{Proposition}
\newtheorem{definition}[definition]{Definition}
\newtheorem{exercise}[exercise]{Exercise}
\newtheorem{corollary}[corollary]{Corollary}
\newtheorem{remark}[remark]{Remark}

\def\lemautorefname{Lemma}
\def\thmautorefname{Theorem}
\def\corollaryautorefname{Corollary}
\def\definitionautorefname{Definition}
\aliascntresetthe{lem}
\aliascntresetthe{prop}
\aliascntresetthe{definition}
\aliascntresetthe{exercise}
\aliascntresetthe{corollary}
\aliascntresetthe{remark}

\newcommand{\R}{\mathbb{R}}
\newcommand{\N}{\mathbb{N}}
\newcommand{\C}{\mathbb{C}}
\newcommand{\vx}{\mathbf{x}}
\newcommand{\vy}{\mathbf{y}}
\newcommand{\e}{\mathrm{e}}
\renewcommand{\d}{\mathrm{d}}
\renewcommand{\i}{\mathrm{i}}



\title{All in One}
\author{Yingjie Zhu}


\begin{document}


\section{Metric Space}

\subsection{Compact Set}

\begin{thm}
      If $E \subseteq \mathbb{R}^n$ is bounded and closed, then $E$ is compact.
\end{thm}

\begin{proof}
Let $\{V_\alpha\}_{\alpha \in I}$ be an open cover of $E$. Define, for each $x \in E$,

\[
  r(x) := \sup\{\, r > 0 : \exists\, \alpha \in I \text{ with } B(x,r) \subseteq V_\alpha \,\}.
\]


We first show that $r$ is lower semicontinuous. Fix $a \in \mathbb{R}$. 
If $r(x) > a$,
then by definition there exist $\alpha$ and $r>a$ with $B(x,r) \subseteq V_\alpha$.
Let $\varepsilon = (r-a)/4 > 0$. For any $p \in B(x,\varepsilon)$ we have
$B(p,r-\varepsilon) \subseteq B(x,r) \subseteq V_\alpha$; since $r-\varepsilon > a+\varepsilon$,
it follows that $r(p) > a$. Hence the set $\{x \in E : r(x) > a\}$ is open, i.e. $r$ is
lower semicontinuous. In particular, for any sequence $x_n \to x$ in $E$,

\[
  \varliminf_{n \to \infty} r(x_n) \ge r(x).
\]

Set $r_0 := \inf\{ r(x) : x \in E \}$.

\textbf{Claim 1}: $r_0 > 0$. Suppose to the contrary that $r_0 = 0$. Then there exists a
sequence $x_n \in E$ with $r(x_n) \to 0$. Since $E$ is bounded, by Bolzano–Weierstrass
there is a subsequence $y_n = x_{f(n)}$ converging to some $y \in \mathbb{R}^n$; because
$E$ is closed, $y \in E$. As $\{V_\alpha\}$ covers $E$ and $y \in E$, there exists some
$\alpha_0$ with $y \in V_{\alpha_0}$; since $V_{\alpha_0}$ is open, there exists
$\delta>0$ with $B(y,\delta) \subseteq V_{\alpha_0}$. Hence $r(y) \ge \delta > 0$.
But lower semicontinuity gives
$
  \liminf_{n \to \infty} r(y_n) \ge r(y) > 0,
$
contradicting $r(y_n)=r(x_{f(n)}) \to 0$. Thus $r_0>0$.

\textbf{Claim 2}: $\{V_\alpha\}$ has a finite subcover. Choose any $x_0 \in E$.
By definition of $r_0$ and the cover, there exists $\alpha_0$ with
$B(x_0,r_0) \subseteq V_{\alpha_0}$. If $E \subseteq V_{\alpha_0}$, we are done.
Otherwise pick $x_1 \in E \setminus V_{\alpha_0}$; then there exists $\alpha_1$
with $B(x_1,r_0) \subseteq V_{\alpha_1}$. Note that $B(x_0,r_0)$ and $B(x_1,r_0)$
are disjoint, since otherwise $x_1 \in B(x_0,r_0) \subseteq V_{\alpha_0}$.

Proceed inductively: having chosen pairwise disjoint balls $B(x_k,r_0) \subseteq V_{\alpha_k}$
for $k=0,\dots,m-1$ whose corresponding $V_{\alpha_k}$ do not yet cover $E$, pick
$x_m \in E \setminus \bigcup_{k=0}^{m-1} V_{\alpha_k}$ and then choose $\alpha_m$
with $B(x_m,r_0) \subseteq V_{\alpha_m}$. Again, $B(x_m,r_0)$ is disjoint from the
previous balls, so the centers satisfy $|x_i - x_j| \ge r_0$ for $i \ne j$.

If this process never stops, we produce infinitely many pairwise disjoint balls of
radius $r_0/2$ inside a fixed bounded set containing $E$. This is impossible in
$\mathbb{R}^n$ (e.g., by a simple volume-packing bound). Therefore the process must
terminate after finitely many steps, yielding a finite subcover of $E$.

Since every open cover of $E$ admits a finite subcover, $E$ is compact.
\end{proof}

\begin{thm}
    In Hausdorff space, compact set is closed.
\end{thm}

\begin{proof}
   Assume $K$ is compact and $p \notin K$, we can construct an open cover of $K$.

   \[
        K \subseteq \bigcup_{q \in K} V_q \quad \text{where } U_q \cap V_p = \emptyset
   \]

   $U_q$ and $V_p$ is open neighborhood of $q,p$ respectively, thus we could pick a finite subcover.
\end{proof}

\end{document}