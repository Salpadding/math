%!LW recipe=latexmk
\documentclass[11pt,a4paper]{article}
\input% math 
\usepackage{amsmath,amsfonts,amssymb,amsthm}
% cross reference, use \autoref instead of \ref
\usepackage{aliascnt}
\usepackage[hidelinks]{hyperref}
\usepackage{enumitem}
\usepackage{geometry}


\geometry{left=1.5cm, right=1.5cm, top=2cm, bottom=2cm}

\newtheorem{thm}{Theorem}[section]

\newaliascnt{lem}{thm}
\newaliascnt{prop}{thm}
\newaliascnt{definition}{thm}
\newaliascnt{exercise}{thm}
\newaliascnt{corollary}{thm}

\theoremstyle{definition}
\newtheorem{lem}[lem]{Lemma}
\newtheorem{prop}[prop]{Proposition}
\newtheorem{definition}[definition]{Definition}
\newtheorem{exercise}[exercise]{Exercise}
\newtheorem{corollary}[corollary]{Corollary}

\def\lemautorefname{Lemma}
\def\thmautorefname{Theorm}
\aliascntresetthe{lem}
\aliascntresetthe{prop}
\aliascntresetthe{definition}
\aliascntresetthe{exercise}
\aliascntresetthe{corollary}


\title{Learning Notes}
\author{Yingjie Zhu}
\date{\today}


\begin{document}
\maketitle

\section{Basic Set Theory}

\begin{lem}
    $A \subseteq B$ iff $A \cap B = A$
\end{lem}

\begin{proof}
    since $x \in A \subseteq A \cap B$, so $x \in B$
\end{proof}

\begin{lem}
    $A \subseteq B$ iff $A \cap B^C = \emptyset$
\end{lem}

\begin{proof}
    \begin{align*}
        A \cap B^C &= \emptyset \\
        A \setminus \left(A \cap B^C\right) &= A \\
        \left( A \cap A^C \right) \cup \left( A \cap B \right) &= A \\
        A \cap B = A 
    \end{align*}

    on the other hand, if $A \subseteq B$, then

    \begin{align*}
        A \cap B  &= A \\
        A \setminus \left(A \cap B \right) &= \emptyset \\
        A \cap B^C &= \emptyset
    \end{align*}
\end{proof}

\begin{lem}
    $A \subsetneq B$ iff $A \cap B^C \ne \emptyset$
\end{lem}

\begin{lem}
    $A \subseteq B$ iff $A \cap B = A$
\end{lem}

\begin{lem}
    $f(A \cup B) = f(A) \cup f(B)$
\end{lem}


\begin{lem}
    $f(A \cap B) \subseteq f(A) \cap f(B)$
\end{lem}


\begin{lem}
    $f(f^{-1}(A)) \subseteq A$, the equal always holds on when $f$ is surjective.
\end{lem}


\begin{lem}
    $A \subseteq f^{-1}(f(A))$, the equal holds on when $f$ is injective.
\end{lem}

\section{Pointwise Topology}

\subsection{Open and Interior}

\begin{definition}[Interior point]
    $x$ is an interior point of $A$ iff exists open set $V_x \subseteq A,\; x \in V_x$ 
\end{definition}


\begin{definition}
    \[
        \text{int}(A) = \{ x \;\text{ is interior point of}\; A \}
    \]
\end{definition}

\begin{lem}
    $\text{int}(A) \subseteq A$
\end{lem}

\begin{proof}
    proved by definition, $x \in V_x,\; V_x \subseteq A$
\end{proof}


\begin{lem}
    $\text{int}(A)$ is always open 
\end{lem}

\begin{proof}
    We can write $A$ as

    \[
        A = \bigcup_{x \in A}V_x
    \]

    since arbitrary union of open sets $V_x$ is also open, so $A$ is open.
\end{proof}

\begin{lem}
    $A$ is open iff $\text{int}(A) = A$
\end{lem}

\begin{proof}
    if $A$ is open, then every point of $A$ is interior point, since we can let $V_x = A$ which is open.
\end{proof}

\begin{corollary}
   $\text{int}(A)$ is largest open set which is sub set of $A$
\end{corollary}

\begin{proof}
    \begin{align*}
        V &\subseteq A \\
        \text{int}(V) &\subseteq \text{int}(A) \\
        V &\subseteq \text{int}(A) \\
    \end{align*}

    so we can denote $\text{int}(A)$ as

    \[
        \text{int}(A) = \bigcup_{V \subseteq A,\; V\;\text{is open}}V
    \]
\end{proof}


\begin{lem}
$\text{int}(A \cap B) = \text{int}(A) \cap \text{int}(B)$

\end{lem}

\begin{proof}
    let $x \in \text{int}(A \cap B)$, so we have open set $V \subseteq A \cap B,\; x \in V$. so 
    $V \subseteq A,\; V \subseteq B$, so $x \in \text{int}(A)$ and $x \in \text{int}(B)$. 

    on the other hand, if $x$ in both $\text{int}(A)$ and $\text{int}(B)$, there exists open set $V_A \subseteq V$
    and open set $V_B \in B$, since finite intersection of open set is open, so we have open set $V_A \cap V_B \subseteq A \cap B$,
    and $x \in V_A,\; x \in V_B$, so $x \in \text{int}(A \cap B)$
\end{proof}


\begin{lem}
$\text{int}(A) \cup \text{int}(B)  \subseteq \text{int}(A \cup B)$
\end{lem}

\begin{proof}
\begin{align*}
    \text{int}(A) &\subseteq \text{int}(A \cup B) \\
    \text{int}(B) &\subseteq \text{int}(A \cup B) \\
\text{int}(A) \cup \text{int}(B) & \subseteq \text{int}(A \cup B)
\end{align*}
\end{proof}

\subsection{Limit Point and Closure}

\begin{definition}[Limit point]
   A limit point of $A$ is that, for every open set $V$ contains $x$, $A \cap \left( V \setminus \{x\} \right) \ne \emptyset$.
   Pay attention that a limit point $x$ of $A$ may not in $A$. 
\end{definition}

\begin{definition}[Derived set]
    Derived set $A'$ of $A$ is all its limit point.
\end{definition}

\begin{definition}[Closure]
    Closure $\overline{A}$ of $A$ is defined as $A \cup A'$
\end{definition}



\begin{lem}
    $\text{int}(A) \cap (A^C)' = \emptyset$
\end{lem}

\begin{proof}
    assume $x \in \text{int}(A)$ and $x \in (A^C)'$, then we have $x \in V_x \subseteq A$, and 
    $(V_x \setminus \{ x\}) \cap A^C = V_x \cap A^C \ne \emptyset$

    however we have

    \[
        V_x \cap A^C \subseteq A \cap A^C = \emptyset
    \]
\end{proof}


\begin{lem}
    $\left( \text{int}(A) \right)^C \subseteq (A^C)'$
\end{lem}

\begin{proof}
    since $x \notin \text{int}(A)$, so for every $V_x$ contains $x$, $V_x \cap A^C \ne \emptyset$ and we have

    \begin{align*}
        \left( V_x \setminus \{ x \} \right) \cap A^C &= V_x \cap \left( A^C \setminus \{ x \} \right)  \\
        &= V_x \cap A^C  \ne \emptyset
    \end{align*}
\end{proof}

\begin{lem}
    $\overline{A^C} = \left( \text{int}(A) \right)^C $
\end{lem}

\begin{proof}
    we first prove $\overline{A^C} \subseteq \left( \text{int}(A) \right)^C$. since $\overline{A^C} = A^C \cup \left(A^C\right)'$. 
    
    as $\text{int}(A) \subseteq A$, so $A^C \subseteq \left(\text{int}(A) \right)^C$
    
    on the other hand, since $\text{int}(A) \cap (A^C)' = \emptyset$, then we have $(A^C)' \subseteq (\text{int}(A))^C$, combine them 
    we have 

    \begin{align*}
        A^C &\subseteq \left(\text{int}(A) \right)^C \\
        (A^C)' & \subseteq \left(\text{int}(A) \right)^C \\
        \overline{A^C} &\subseteq A^C \cup (A^C)' \subseteq \left(\text{int}(A) \right)^C
    \end{align*}

    then we prove $(\text{int}(A))^C \subseteq \overline{A^C}$

    since $\text{int}(A)^C \subseteq (A^C)'$  so we have

    \[
        \text{int}(A)^C \subseteq (A^C)' \cup A^C \subseteq \overline{A^C}
    \]

\end{proof}


\begin{lem}
   $\overline{A}$ is closed 
\end{lem}

\begin{proof}
    $(\overline{A})^C = \text{int}(A^C)$
\end{proof}

\begin{lem}
    $A$ is closed iff $A = \overline{A}$
\end{lem}

\begin{proof}
    \begin{align*}
        A &= \overline{A} \\
        A^C &= (\overline{A})^C = \text{int}(A^C) \\
        A^C & \: \text{is open}
    \end{align*}
\end{proof}

\begin{lem}
    $\overline{A}$ is minimal closed set which is super set of $A$
\end{lem}

\begin{proof}
    let $A \subseteq Y$ and $Y$ is closed. then we have

    \begin{align*}
        A &\subseteq Y \\
        \overline{A} &\subseteq \overline{Y} \subseteq Y
    \end{align*}
\end{proof}

\begin{corollary}
    We can denote closure of $A$ as below

    \[
        \overline{A} = \bigcap_{B \supseteq A,\, B\; \text{is closed}}B
    \]
\end{corollary}

\begin{lem}
    $\overline{A} \cup \overline{B} = \overline{A \cup B}$
\end{lem}

\begin{proof}
    take complement on both side

    \begin{align*}
        \text{int}(A^C) \cap \text{int}(B^C) &= \text{int}(A^C \cap B^C) \\
        \overline{A} \cup \overline{B} &= \overline{A \cup B}
    \end{align*}
\end{proof}


\begin{lem}
    $\overline{A} \cap \overline{B} \subseteq \overline{A \cap B}$
\end{lem}


\begin{proof}
    take complement on both side:
    \begin{align*}
        \text{int}(A^C) \cup \text{int}(B^C) &\subseteq \text{int}(A^C \cup B^C) \\
        \overline{A \cap B} & \subseteq \overline{A} \cap \overline{B} 
    \end{align*}
\end{proof}

\section{Boundary Point, Isolated Point}

\begin{definition}[Boundary Point]
    Boundary point of $A$ is defined as 
    
    \[
    \partial A = \left(\text{int}(A) \right)^C \cap \left(\text{int}(A^C) \right)^C
    \]
\end{definition}

\begin{lem}
    $\partial A = \overline{A} \setminus \text{int}(A)$
\end{lem}

\begin{proof}
   \begin{align*}
    \partial A &= \left(\text{int}(A) \right)^C \cap \left(\text{int}(A^C) \right)^C \\
    &= \left(\text{int}(A) \right)^C \cap \overline{A} \\
    &= \overline{A} \setminus \text{int}(A)
   \end{align*} 
\end{proof}

\begin{lem}
    $\partial A$ is closed
\end{lem}

\begin{proof}
    since $A$ is intersection of two closed set.
\end{proof}


\begin{lem}
    $\partial A$ contains no interior point
\end{lem}

\begin{proof}
    \begin{align*}
        \text{int}(\partial A) &= \text{int}(\overline{A^C} \cap \overline{A}) \\
        & \subseteq \text{int}(\overline{A \cap A^C}) \\
        & \subseteq \text{int}(\overline{\emptyset}) \\
        & \subseteq \emptyset
    \end{align*}
\end{proof}

if $\overline{A}$ has no interior point, $A$ is called no where dense set.

\begin{definition}[Isolated Point]
    $x$ is isolated point of $A$ iff $x \in A \setminus A'$ 
\end{definition}

\begin{lem}
    \begin{align*}
        \overline{A} \setminus A' &= \left( A \cup A'\right) \setminus A' \\
        &=  A \setminus A'
    \end{align*}
\end{lem}

\section{Continuity}

\begin{lem}
    continuous funtion maps compact set to compact set
\end{lem}

\begin{proof}
    let $A \subseteq X$ and $A$ is compact, and we let $f(A)$ covered by arbitrary number of open sets

    \begin{align*}
        f(A) \subseteq \bigcup_{\alpha \in I}V_{\alpha}
    \end{align*}

    then we have

    \[
        A \subseteq f^{-1}(f(A)) \subseteq \bigcup_{\alpha \in I}f^{-1}(V_{\alpha})
    \]

    since $f$ is continus, whose preimage of open set is open, now we have a open cover of $A$

    \[
        A \subseteq \bigcup_{\alpha \in I}f^{-1}(V_{\alpha})
    \]

    so there is a finite sub cover $V_1, V_2, .. V_m$ such that

    \[
        A \subseteq \bigcup_{k=1}^{m}f^{-1}(V_{k})
    \]

    so we have

    \[
        f(A)\subseteq \bigcup_{k=1}^{m}f(f^{-1}(V_{k})) \subseteq \bigcup_{k=1}^{m}V_k
    \]

    notice that

    \[
        f(f^{-1}(A)) \subseteq A
    \]
\end{proof}

\end{document}








