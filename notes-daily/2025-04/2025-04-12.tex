%!LW recipe=latexmk
\documentclass[11pt,a4paper]{article}
\input% math 
\usepackage{amsmath,amsfonts,amssymb,amsthm}
% cross reference, use \autoref instead of \ref
\usepackage{aliascnt}
\usepackage[hidelinks]{hyperref}
\usepackage{enumitem}
\usepackage{geometry}


\geometry{left=1.5cm, right=1.5cm, top=2cm, bottom=2cm}

\newtheorem{thm}{Theorem}[section]

\newaliascnt{lem}{thm}
\newaliascnt{prop}{thm}
\newaliascnt{definition}{thm}
\newaliascnt{exercise}{thm}
\newaliascnt{corollary}{thm}

\theoremstyle{definition}
\newtheorem{lem}[lem]{Lemma}
\newtheorem{prop}[prop]{Proposition}
\newtheorem{definition}[definition]{Definition}
\newtheorem{exercise}[exercise]{Exercise}
\newtheorem{corollary}[corollary]{Corollary}

\def\lemautorefname{Lemma}
\def\thmautorefname{Theorm}
\aliascntresetthe{lem}
\aliascntresetthe{prop}
\aliascntresetthe{definition}
\aliascntresetthe{exercise}
\aliascntresetthe{corollary}


\title{Learning Notes}
\author{Yingjie Zhu}
\date{\today}


\begin{document}
\maketitle

\section{Theory of Real Variable Function}

\subsection{Exercise}

\begin{lem}
    $E$ is a perfect set iff it is closed and contains no isolated points.
\end{lem}

\begin{proof}
    let $E$ be a perfect set such that $E = E'$, then we have

    \[
        \overline{E} = E \cup E' = E
    \]

    so $E$ is closed, and isolated points have

    \[
        E \setminus E' = E \setminus E = \emptyset
    \]

    On the other hand, if $E$ is closed and has no isolated points, then we have

    \begin{align*}
        E \setminus E' &= \emptyset \\
        E & \subseteq E' \\ 
        E \cup E' &= E \\
        E' & \subseteq E
    \end{align*}
\end{proof}


\begin{lem}
    let $x$ be a limit point of $E$, $E \subseteq \mathbb{R}^n$, and $y \in E$, then $x$ is also a limit point of $E \setminus y$ 
\end{lem}

\begin{proof}
    since $x$ is a limit point of $E$, there should exists a distinct sequence $x_n$ and

    \[
        \lim_{n \to \infty}x_n = x
    \]

    then we can construct a sequence of $E \setminus \{ y \}$ by remove $y$ of $x_n$. Since $x_n$ 
    is distinct, so $y$ occurs at most once. So $x$ is a limit point of $E \setminus \{ y \}$
\end{proof}

\begin{exercise}
    Perfect Set on $\mathbb{R}^n$ is not countable. 
\end{exercise}

\begin{proof}
    let $E$ be a non empty perfect set and assume it is countable so that

    \[
        E = \{ x_1, x_2, .. \}
    \]

    since every finite set under $\mathbb{R}^n$ has isolated points, so $E$ cannot be finite.

    we first pick $y_1 \in E \setminus \{ x_1 \}$, since $y_1$ is a limit point of $E \setminus \{ x_1 \}$. 
    there should exists an open qube $Q_1$ conatins $y_1$, and $x_1 \notin \overline{Q_1}$

    let's consider $E \setminus \{ x_2 \}$, since $y_1$ is limit point of $E \setminus \{ x_2 \}$, and $Q_1$ is a openset contains $y_1$ so
    we have

    \[
       \text{int}(Q_1) \cap \left( E \setminus \{ x_2 \} \right) \ne \emptyset
    \]

    now we can select $y_2 \in \text{int}(Q_1) \cap E \setminus \{ x_2 \}$, and we can create open cube $Q_2$ contains 
    $y_2$, so that $Q_2 \subseteq Q_1$, and $\{ x_1, x_2 \} \cap \overline{Q_2} = \emptyset$

    let's consider $E \setminus \{ x_3 \}$, since $y_2$ is a limit point of $E \setminus \{ x_3 \}$, and $Q_2$ is an
    open set contains $y_2$, so we have

    \[
       \text{int}(Q_2) \cap \left( E \setminus \{ x_3 \} \right) \ne \emptyset
    \]

    now we can select $y_3 \in \text{int}(Q_2) \cap \left( E \setminus \{ x_3 \} \right)$
    as $Q_2$ is open, and $\{ x_1, x_2 \} \cap Q_2 = \emptyset$. so we can create 
    a smaller qube $Q_3$ contains $y_3$ so that $\{x_1 ,x_2, x_3 \} \cap \overline{Q_3} = \emptyset$

    so we can create a sequence $y_n$ and open qube $Q_n$ contains $y_n$, and also a 
    sequence of closed cube $\overline{Q_n}$

    let's consider $\overline{Q_n}$, since $\overline{Q_n}$ is closed and non empty, by cantor's theorm,
    we have

    \[
        \bigcap_{n=1}^{\infty}\overline{Q_n} \ne \emptyset
    \]

    howevery, for any $x_n \in E$, the qube $\overline{Q_{n}}$ not contains $x_n$,
    which is contradict.
\end{proof}

\begin{definition}[Distance]
    in a metric space $(X,d)$, the distance between point $x$ and set $E$ is defined as

    \[
        d(x, E) = \inf \{ d(x,y) \vert y \in E \}
    \]
\end{definition}

\begin{definition}[Distance]
    in a metric space $(X, d)$, the distance between two set $A$ and $B$ is defined as 

    \[
        d(A, B) = \inf \{ d(x,y) \vert x \in A,\, y \in B \}
    \]
\end{definition}

\begin{lem}
    we have $d(A,B) = \inf \{ d(x,B) \vert x \in A \}$
\end{lem}

\begin{proof}
    let $\alpha = \inf \{ d(x,B) \vert x \in A \}$, and $\beta = d(A,B)$ since inferior is a limit point, where exists $x_n \in A$ such that

    \[
        \alpha = \lim_{n \to \infty}d(x_n, B)
    \]

    for every $x_n$, we have a sequence $y_m \in B$

    \[
        d(x_n, B) = \lim_{m \to \infty}d(x_n, y_m)
    \]

    since $d(x_n, y_m) \ge \beta$, by law of limit $d(x_n, B) \ge \beta$, take $n \to \infty$, we have $d(x, B) \ge \beta$

    on the other hand, $\beta$ could be expressed as a limit 

    \[
        \beta = \lim_{n \to \infty}d(x_n, y_n)
    \]

    by definition of inferior, we have

    \[
        d(x_n, y_n) \ge d(x_n, B) \ge \alpha
    \]
    
    take $n \to \infty$, we have 

    \[
        \lim_{n \to \infty}d(x_n, y_n ) \ge \lim_{n \to \infty}d(x_n, B) \ge \alpha
    \]

    so as

    \[
        \beta \ge \alpha
    \]

\end{proof}

\begin{lem}
    let $E$ be a set under metric space $(X,d)$, and $x \in X$, and there exists a closed ball $\overline{B}(x, r)$
    such that $\overline{B}(x,r) \cap E \ne \emptyset$, then we have

    \[
        d(x, E) = d(x, E \cap \overline{B}(x,r))
    \]
\end{lem}

\begin{proof}
    we first prove that $d(x, E) \le d(x, E \cap \overline{B}(x,r))$, since $E \cap \overline{B}(x,r) \subseteq E$, and
    inferior of subset should not less than its super set. so we have

    \[
        d(x, E) \le d(x, E \cap \overline{B}(x,r))
    \]

    then we prove $d(x, E) \ge d(x, E \cap \overline{B}(x,r))$, we will discuss on elements of $E$. For any point $y \in E$,
    if $y \in \overline{B}(x,r)$, then $x \in E \cap \overline{B}(x,r)$, so by definition of inferior. we have

    \[
        d(x, y) \ge d(x, E \cap \overline{B}(x,r))
    \]

    on the other hand, if $y \notin \overline{B}(x,r)$, by definition of ball, we have $d(x,y) \ge r$. since
    $E \cap \overline{B}(x,r) \ne \emptyset$. we can find $y' \in E \cap \overline{B}(x,r) $, and $d(x,y') \le r$.
    so we have $d(x,y) \ge r \ge d(x,y') \ge d(x, E \cap \overline{B}(x,r))$

    after discussion, for every point $y \in E$, we have $d(x,y) \ge d(x, E \cap \overline{B}(x,r))$, after take inferior on $y$, we have

    \[
        d(x,E) \ge d(x, E \cap \overline{B}(x,r))
    \]

    after all, $d(x,E) = d(x, E \cap \overline{B}(x,r))$
\end{proof}

\begin{lem}
    let $E$ be a closed and non empty set under metric space $(X,d)$. and $x \in X$, then exists $y \in E$ such that 
    $d(x,y) = d(x, E)$
\end{lem}

\begin{proof}
   we select a element $y_0$ of $E$, and inspect $d(x,y_0)$. if $d(x,y_0) = 0$, then $y_0$ is the $y$ what we want. because
   $d(x,y_0) = d(x, E) = 0$. 
   
   if $d(x, y_0) > 0$, we set $r_0 = d(x,y_0)$, and create a closed ball $\overline{B}(x,r_0)$. since
   $y_0 \in E \cap \overline{B}(x, r_0)$, then $E \cap \overline{B}(x, r_0) \ne \emptyset$. by our proved lemma, we have

   \[
    d(x, E) = d(x, E \cap \overline{B}(x, r_0))
   \]

   we create a function $g: E \cap \overline{B}(x, r_0) \to \mathbb{R}$, and

   \[
    g(y) = d(x, y)
   \]

   since $g$ is a continuous function on a compact set $E \cap \overline{B}(x, r_0)$. so there
   should exists $y_1$ such that

   \[
    g(y_1) = \inf \{ d(x,y) \vert y \in E \cap \overline{B}(x, r_0)\}
   \]

   and $y_1$ is what we want
\end{proof}

\begin{lem}
    let $(X,d)$ be a metric space and $E \subseteq X$, then function $f: X \to \mathbb{R},\, f(x) = d(x, E)$
    is uniformly continuous.
\end{lem}

\begin{proof}
    consider two point $x,y \in X$.  then for any point $z \in E$, we have 

    \begin{align*}
        d(x,z) &\le d(x,y) + d(y,z) \\
        d(x, E) - d(y,z) &\le d(x,z) -d(y,z) \le d(x,y) \\
    \end{align*}

    since we could select arbitrary $z$, so we can select a sequence $z_n$ and such that

    \begin{align*}
        \lim_{n \to \infty}d(y,z_n) &= d(y,E) 
    \end{align*}

    then we have

    \[
        d(x, E) - d(y,z_n)  \le d(x,y) \\
    \]

    take $n \to \infty$, then we got

    \[
        d(x,E) - d(y,E) \le d(x,y)
    \]

    also we will got

    \[
        d(y,E) - d(x,E) \le d(x,y)
    \]

    if use below triangle inequality

    \[
        d(y,z) \le d(y,x) + d(x,z)
    \]

    so we have

    \[
        \lvert d(y,E) - d(x,E) \rvert \le d(x,y)
    \]

\end{proof}

\end{document}









