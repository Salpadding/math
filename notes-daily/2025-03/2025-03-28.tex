%!LW recipe=latexmk (xelatex)
% a4 页面 字体大小12像素
\documentclass[12pt,a4paper]{ctexart}
\input% math 
\usepackage{amsmath,amsfonts,amssymb,amsthm}
% cross reference, use \autoref instead of \ref
\usepackage{aliascnt}
\usepackage[hidelinks]{hyperref}
\usepackage{enumitem}
\usepackage{geometry}


\geometry{left=1.5cm, right=1.5cm, top=2cm, bottom=2cm}

\newtheorem{thm}{Theorem}[section]

\newaliascnt{lem}{thm}
\newaliascnt{prop}{thm}
\newaliascnt{definition}{thm}
\newaliascnt{exercise}{thm}
\newaliascnt{corollary}{thm}

\theoremstyle{definition}
\newtheorem{lem}[lem]{Lemma}
\newtheorem{prop}[prop]{Proposition}
\newtheorem{definition}[definition]{Definition}
\newtheorem{exercise}[exercise]{Exercise}
\newtheorem{corollary}[corollary]{Corollary}

\def\lemautorefname{Lemma}
\def\thmautorefname{Theorm}
\aliascntresetthe{lem}
\aliascntresetthe{prop}
\aliascntresetthe{definition}
\aliascntresetthe{exercise}
\aliascntresetthe{corollary}


\title{学习记录}
\author{朱英杰}
\date{2025-03-28}

\begin{document}
\zihao{-4}
\maketitle
\tableofcontents

\section{行列式计算}

\subsection{包含三角函数的行列式}

\begin{align*}
    A_n = \begin{vmatrix}
        \cos x & 1 & 0 & 0 & .. & 0 & 0 & 0 \\
        1 & 2\cos x & 1 & 0 & .. & 0 & 0 & 0 \\
        0 & 1 & 2 \cos x & 1 &  .. & 0 & 0 & 0\\
        .. & .. &.. &.. &.. &.. &.. & .. \\
        0 & 0 & 0  & 0 &  .. & 1 & 2 \cos x & 1\\
        0 & 0 & 0  & 0 &  .. & 0 & 1 & 2 \cos x\\
    \end{vmatrix} =  D_n - \cos x D_{n-1}
\end{align*}

其中对 $D_n$ 有 $D_n = 2 \cos x D_{n-1} - D_{n-2}$

用求根公式可以求解

\begin{align*}
\alpha + \beta &= 2 \cos x \\
-\alpha \beta &= -1 \\
y^2 - 2 \cos x y + 1 & = 0 
\end{align*}

得到当 $\alpha \ne \beta$ 时

\begin{align*}
\alpha &= \frac{2 \cos x + \sqrt{4 \cos^2 x - 4}}{2} = \cos x + i \sin x \\
\beta &= \frac{2 \cos x - \sqrt{4 \cos^2 x - 4}}{2} = \cos x - i \sin x \\
\end{align*}

$D_n$ 可以计算出来

\begin{align*}
    D_n &= \frac{ (\cos x + i \sin x)^{n+1} - (\cos x - i \sin x)^{n+1} }{2 i \sin x} \\
    &= \frac{e^{i(n+1)x}  - e^{-i(n+1)x}}{2 i \sin x} = \frac{\sin ((n+1)x)}{ \sin x}
\end{align*}

于是有

\begin{align*}
A_n = D_n - \cos x D_{n-1} &= \frac{\sin ((n+1)x)}{\sin x} - \cos x \frac{\sin (nx)}{\sin x} \\
&= \frac{\cos (nx) \sin x}{\sin x} = \cos (nx)
\end{align*}

当 $\alpha = \beta = \cos x$ 时,还要满足 $\alpha \beta = 1$,所以此时必然有 $x = k \pi$

\[
D_n = (n+1)(\cos x)^n = (n+1)(-1)^{kn}
\]

而

\begin{align*}
    A_n &= D_n - \cos x D_{n-1} = (n+1)(-1)^{kn} - n(-1)^{k(n-1)}(-1)^k \\
    &= (-1)^{kn} \\
    \cos (nx) &= \cos (n k \pi) = (-1)^{kn}
\end{align*}

所以有

\[
A_n = \cos (nx)
\]

\subsection{递推}

\begin{align*}
    D_n &= \begin{vmatrix}
        x_1 & y & y & .. & y & y \\
        z & x_2 & y & .. & y & y \\
        z & z & x_3 & .. & y & y \\
        .. & .. &.. &.. &.. & .. \\
        z & z & z &.. & x_{n-1} & y \\
        z & z & z &.. & z & x_n \\
    \end{vmatrix} = \begin{vmatrix}
        x_1 & y & y & .. & y & 0 \\
        z & x_2 & y & .. & y & 0 \\
        z & z & x_3 & .. & y & 0 \\
        .. & .. &.. &.. &.. & .. \\
        z & z & z &.. & x_{n-1} & 0 \\
        z & z & z &.. & z & x_n - y\\
    \end{vmatrix} + \begin{vmatrix}
        x_1 & y & y & .. & y & y \\
        z & x_2 & y & .. & y & y \\
        z & z & x_3 & .. & y & y \\
        .. & .. &.. &.. &.. & .. \\
        z & z & z &.. & x_{n-1} & y \\
        z & z & z &.. & z & y \\
    \end{vmatrix} \\
    &= (x_n - y)D_{n-1} + y\prod_{k=1}^{n-1}(x_k - z)
\end{align*}

同理也有

\[
= (x_n - z) D_{n-1} + z\prod_{k=1}^{n-1}(x_k - y)
\]

若 $y \ne z$ 可以解得,克莱姆法则

\begin{align*}
D_n &= \frac{(x_n -y)z\prod_{k=1}^{n-1}(x_k -y) - (x_n -z)y \prod_{k=1}^{n-1}(x_k-z)}{z-y} \\
&= \frac{1}{z-y}\left[z\prod_{k=1}^{n}(x_k -y) - y \prod_{k=1}^{n}(x_k-z)\right]
\end{align*}

\subsection{计算下列行列式的值}

\begin{align*}
    D_n &= \begin{vmatrix}
        1-a_1 & a_2 & 0 & 0 & .. & 0 & 0 \\
        -1 & 1-a_2 & a_3 & 0 & .. & 0 & 0 \\
        0 & -1 & 1-a_3 & a_4 & .. & 0 & 0 \\
        .. & .. &.. &.. &.. & .. & ..\\
        0 & 0 & 0 & 0 &.. & 1-a_{n-1} & a_n\\
        0 & 0 & 0 & 0 &.. & -1 & 1-a_n\\
    \end{vmatrix} =     \begin{vmatrix}
        -a_1 & 0 & 0 & 0 & .. & 0 & 1 \\
        -1 & 1-a_2 & a_3 & 0 & .. & 0 & 0 \\
        0 & -1 & 1-a_3 & a_4 & .. & 0 & 0 \\
        .. & .. &.. &.. &.. & .. & ..\\
        0 & 0 & 0 & 0 &.. & 1-a_{n-1} & a_n\\
        0 & 0 & 0 & 0 &.. & -1 & 1-a_n\\
    \end{vmatrix} \\
    &= (-a_1)D_{n-1} + (-1)^{n+1}(-1)^{n-1} = -a_1D_{n-1} + 1
\end{align*}

\subsection{数学归纳法}

\begin{align*}
    \begin{vmatrix}
        a_0 + a_1 & a_1 & 0 & 0 & .. & 0 & 0 \\
        a_1 & a_1 + a_2 & a_2 & 0 & .. & 0 & 0 \\
        0 & a_2 & a_2 + a_3 & a_3 & .. & 0 & 0 \\
        ..& ..&..&..&..& .. & .. \\
        0 & 0 & 0 & 0 & .. & a_{n-1} & a_{n-1} + a_n \\
    \end{vmatrix}
\end{align*}

对 $a_i$ 作轮换,令 $b_i = a_{n-i}$,得到

\begin{align*}
    \begin{vmatrix}
        b_n + b_{n-1} & b_{n-1} & 0 & 0 & .. & 0 & 0 \\
        b_{n-1} & b_{n-1} + b_{n-2} & b_{n-2} & 0 & .. & 0 & 0 \\
        0 & b_{n-2} & b_{n-2} + b_{n-3} & b_{n-3} & .. & 0 & 0 \\
        ..& ..&..&..&..& .. & .. \\
        0 & 0 & 0 & 0 & .. & b_1 & b_1 + b_0 \\
    \end{vmatrix}
\end{align*}

然后得到递推式

\[
D_n = (b_n + b_{n-1})D_{n-1} - b_{n-1}^{2}D_{n-2}
\]

我们要证明 

\[
D_n = \prod_{i=0}^{n}b_i\left( \sum_{i=0}^{n}\frac{1}{b_i} \right)
\]

用数学归纳法

\begin{align*}
D_n &= (b_n + b_{n-1})D_{n-1} - b_{n-1}^{2}D_{n-2} \\
    &= (b_n + b_{n-1})\prod_{i=0}^{n-1}b_i\left( \sum_{i=0}^{n-1}\frac{1}{b_i} \right) - b_{n-1}^2 \prod_{i=0}^{n-2}b_i\left( \sum_{i=0}^{n-2}\frac{1}{b_i} \right) \\
    &= (b_n + b_{n-1})\prod_{i=0}^{n-1}b_i\left( \sum_{i=0}^{n-2}\frac{1}{b_i} + \frac{1}{b_{n-1}} \right) - b_{n-1} \prod_{i=0}^{n-1}b_i\left( \sum_{i=0}^{n-2}\frac{1}{b_i} \right) \\
    &= \prod_{i=0}^{n}b_i\left( \sum_{i=0}^{n-1}\frac{1}{b_i} \right) + b_{n-1}\prod_{i=0}^{n-1}b_i \frac{1}{b_{n-1}} \\
    &= \prod_{i=0}^{n}b_i\left( \sum_{i=0}^{n}\frac{1}{b_i} \right)
\end{align*}

\subsection{行列式函数的微分}

令

\begin{align*}
    F(t) = \begin{vmatrix}
        f_{11}(t) & f_{12}(t) & .. & f_{1n}(t)\\
        f_{21}(t) & f_{22}(t) & .. & f_{2n}(t)\\
        .. & .. & .. & .. \\
        f_{n1}(t) & f_{n2}(t) & .. & f_{nn}(t)\\
    \end{vmatrix}
\end{align*}

证明 

\[
\frac{\text{d} F}{\text{d}t} = \sum_{j=1}^{n}F_j(t)
\]

其中 

\[
F_j(t) = \begin{vmatrix}
f_{11}(t) & f_{12}(t) & .. & f'_{1j}(t) & .. & f_{1n}(t)\\
f_{21}(t) & f_{22}(t) & .. & f'_{2j}(t) & .. & f_{2n}(t)\\
.. & .. & .. & .. & .. & .. \\
f_{n1}(t) & f_{n2}(t) & .. & f'_{nj}(t) & .. & f_{nn}(t)\\
\end{vmatrix}
\]

同样可以用数学归纳法,对 $F(t)$ 的第一列展开得到

\begin{align*}
F(t) = \sum_{i=1}^{n}f_{i1}(t)A_{i1}(t)
\end{align*}

两边同时微分得到

\begin{align*}
F'(t) = \sum_{i=1}^{n}f'_{i1}(t)A_{i1}(t) + f_{i1}(t)(-1)^{i+1}M'_{i1}(t)
\end{align*}

注意到有

\[
F_1(t) = \sum_{i=1}^{n}f'_{i1}(t)A_{i1}(t)
\]

而且有

\begin{align*}
\sum_{i=1}^{n}f_{i1}(t)(-1)^{i+1}M'_{i1}(t) &= \sum_{i=1}^{n}f_{i1}(t)(-1)^{i+1} \sum_{j=2}L_{ij} \\ 
&= \sum_{j=2}^{n}\sum_{i=1}^{n}(-1)^{i+1}f_{i1}(t)L_{ij}
\end{align*}

这里用 $L_{ij}$ 表示 $M_{ij}$ 的 $F_j$,注意到当 $j$ 固定时有,思考下按第一列展开

\[
F_j = \sum_{i=1}^{n}(-1)^{i+1}f_{i1}(t)L_{ij}
\]

\end{document}

