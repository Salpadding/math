\subsubsection{Topology}

\begin{exercise}
    let $V, W \subseteq \mathbb{R}^n$ and $V \cap \overline{W} = \emptyset,\: \overline{V} \cap W = \emptyset$, prove:

    for any $x \in V,\: y \in W$ and $\gamma: [0,1] \to \mathbb{R}^n$ is a continuous function, and  $\gamma(0) = x,\: \gamma(1) = y$

    then $\exists 0 < t < 1, \gamma(t) \notin V \cup W$

\end{exercise}

\begin{proof}
    we define: 
    
    \begin{align*}
        t_0 &= \sup_{0 \le t \le 1} \{ t: \gamma(t) \in V \} \\
           z &= \gamma(t_0)
    \end{align*}

    let's define $t_n \to t_0, \gamma(t_n) \in V$, since $\overline{V}$ is closed and $\gamma$ is continuous, we have

    \[
        \lim_{n \to \infty}\gamma(t_n) = \gamma(t_0) \in \overline{V}
    \]

    since $z \in \overline{V}$ then we must have $z \notin W$ and $t_0 < 1$ let's discuss on if $z \in V$

    \begin{enumerate}
        \item $z \notin V$

        then $t_0$ is what we want

        \item $z \in V$

        then we have $z \notin (\overline{W})^C$, let's define $F = (\overline{W})^C$, since $F$ is open, we have an open set $V_z \subseteq F$,
        since $\gamma$ is continuous, so $\gamma^{-1}(V_z)$ is also an open set, and we have $t_0 \in \gamma^{-1}(V_z)$

        let's pick $0 < \epsilon < 1 - t_0$ be small enough, so that $t_0 + \epsilon \in \gamma^{-1}(V_z)$

        because if $z' \in V$, then $t_0 + \epsilon$ is contradict with that $t_0$ is sup.

        so $t_0 + \epsilon$ is what we want
    \end{enumerate}
\end{proof}


\begin{exercise}
    if $A$ is a set under metric space, then below statements are identical

    \begin{enumerate}
        \item $x$ is a limit point of $A$

        \item exists a sequence $a_n \in A \setminus \{ x \}$ and $a_n \to x$

        \item exists a distinct sequence $a_n$ in $A$ and $a_n \to x$

    \end{enumerate}
\end{exercise}

\begin{proof}
    steps:

    \begin{enumerate}
        \item $1 \to 2$

        we construct a sequence like below:

        \begin{align*}
            a_1 & \in A \setminus \{x\} \quad \text {and} \quad d(a_1, x) < 1 \\
            a_2 & \in A \setminus \{x\} \quad \text {and} \quad d(a_2, x) < \min(d(a_1, x), \frac{1}{2}) \\
            .. \\
            a_n & \in A \setminus \{x\} \quad \text {and} \quad d(a_n, x) < \min(d(a_{n-1}, x), \frac{1}{n}) \\
        \end{align*}

        by definition of limit point, $a_n$ is well defined and $a_n \to x$

        \item $2 \to 3$

        assume $a_n \to x$ and $\forall n, a_n \ne x$, we then pick a subsequence of $a_n$:

        \begin{align*}
            b_1 &= a_{n_1} \quad \text{and} \quad d(b_1, x) < 1 \\
            b_2 &= a_{n_2} \quad \text{and} \quad n_2 > n_1,\: d(b_2, x) < \min(d(b_1, x), \frac{1}{2}) \\
            .. \\
            b_k &= a_{n_k} \quad \text{and} \quad n_k > n_{k-1},\: d(b_k, x) < \min(d(b_{k-1}, x), \frac{1}{k}) \\
        \end{align*}

        \item $3 \to 1$

        assume $a_n$ is distinct, so $x$ occurs at most once in $a_n$, so we can remove this item and get a subsequence $b_n \in A \setminus \{ x \}$
        and $b_n \to x$

        by definition of metric space, for any open ball $B(x, r), r > 0$ we have $N$ and $\forall n \ge N,\: b_n \in B(x,r)$

        since $b_N \in A \setminus \{ x \}$, so $x$ is a limit point of $A$
    \end{enumerate}
\end{proof}


\begin{exercise}
    let $\{ a_n \} $ contains points under metric space, and we define

    \[
        E(x, r) = \{ n : d(a_n, x) < r \}
    \]

    and

    \[
        F = \{ x: \forall r > 0, E(x,r) \text{\: is infinite }\}
    \]

    prove: $F$ is closed

\end{exercise}

\begin{proof}
    We will prove $F' \subseteq F$. assume $x \in F'$ and there exists a sequence $x_n,\: x_n \in F \setminus \{x\}$ and $x_n \to x$

    we will prove $x \in F$. consider any $r > 0$, since $x_n \to x$, we can pick $N$ be large enough so that
    $d(x_N, x) < r/2$, and since $x_N \in F$, we have $E(x_N, r/2)$ be infinite. let's prove $E(x_N, r/2) \subseteq E(x,r)$:

    \[
        d(a_n, x) \le d(a_n, x_N) + d(x_N, x) < r
    \]

    so $E(x,r)$ is infinite and $x \in F$
\end{proof}