\subsubsection{Analysis}

\begin{exercise}
    let $f_n: \mathbb{R} \to [0, \infty)$ meets:

    \begin{enumerate}
        \item $f_n$ is continuous and supported on $[-1,1]$

        which means $\forall |x| > 1,\: f_n(x)= 0$
        
        \item integral of $f_n(x)$:

    \[
        \int_{\mathbb{R}} f_n(x) \mathrm{d}x = 1
    \]

    \end{enumerate}


    and $\forall \epsilon > 0$ and $ 0 < \delta \le 1$, exists $N$, and for all $n \ge N$:

$\forall \delta \le |x| \le 1,\: f_n(x) \le \epsilon$

let $g: \mathbb{R} \to \mathbb{R}$ be continuous and supported on $[0,1]$, prove:

\[
    f_n \ast g \to g \quad \mathrm{uniformly}
\]
\end{exercise}

\begin{proof}
    since $g$ is continuous on a closed interval, and hence bounded and uniformly continuous. we pick any $\epsilon > 0$ 
    and let $|g(x)| \le M$, since $g$ is uniformly continuous, we pick $\delta > 0$ so that 
    for all $x,y \in [0,1],\: |x-y| \le \delta $  we have $|g(x) -g(y)| \le \epsilon$

    by definition of $f_n$, we pick $N$, so that for all $n \ge N$, we have:

 $\forall  \delta \le |x| \le 1,\:  f(x) \le \epsilon$, then we have

    \begin{align*}
        \left| (f_n \ast g)(x) - g(x) \right| &= \left| \int_{\mathbb{R}}f_n(y)g(x-y) \mathrm{d}y  - g(x)\right| = \left| \int_{[-1,1]}f_n(y)g(x-y) \mathrm{d}y  - g(x)\right| \\ 
        & \le \left| \int_{[-1,-\delta]}f_n(y)g(x-y) \mathrm{d}y \right| + \left| \int_{[\delta,1]}f_n(y)g(x-y) \mathrm{d}y \right| + \left| \int_{[-\delta,\delta]}f_n(y)g(x-y) \mathrm{d}y - g(x)\right| \\
        & \le \int_{[-1,-\delta]}f_n(y)\left| g(x-y) \right| \mathrm{d}y + \int_{[\delta,1]}f_n(y)\left| g(x-y) \right| \mathrm{d}y +  \left| \int_{[-\delta,\delta]}f_n(y)g(x-y) \mathrm{d}y - g(x)\right| \\
        & \le  2M\epsilon(1-\delta) + \left| \int_{[-\delta,\delta]}f_n(y)g(x-y) \mathrm{d}y - g(x)\right|
    \end{align*}

    let's consider:

    \[
    \int_{[-\delta,\delta]}f_n(y)g(x-y) \mathrm{d}y
    \]

    since $f_n(y) \ge 0$ on $[-\delta,\delta]$ and $g(x-y)$ of $y$ is continuous on $[-\delta, \delta]$, because we can assume $g(0)=g(1) =0$, 
    and it will not affect result of convolution. so there exists $\xi \in [-\delta, \delta]$ meets:

    \[
    \int_{[-\delta,\delta]}f_n(y)g(x-y) \mathrm{d}y = g(x-\xi)\int_{[-\delta,\delta]}f_n(y)\mathrm{d}y
    \]

    since $|x-\xi -x| \le \delta$, so we have $g(x-\xi) -g(x) = c_1,\: |c_1| \le \epsilon$, and


    \[
        1-\int_{[-\delta,\delta]}f_n(y) \mathrm{d}y = c_2  =  \int_{[-1, -\delta]}f_n(y) \mathrm{d}y + \int_{[\delta, 1]}f_n(y) \mathrm{d}y
    \]

    and $|c_2| \le 2\epsilon(1-\delta)$

    so we got:

    \begin{align*}
        & \left| \int_{[-\delta,\delta]}f_n(y)g(x-y) \mathrm{d}y - g(x)\right| \\
        &= \left| g(x-\xi)\int_{[-\delta, \delta]}f_n(y)\mathrm{d}y - g(x) \right| \\
        &= \left| (g(x) + c_1)(1-c_2) - g(x) \right| \\
        &= \left| c_1 - c_1c_2 - c_2g(x) \right| \\
        & \le \epsilon + 2\epsilon^2(1-\delta) + 2M\epsilon(1-\delta)
    \end{align*}


    after all, we got

    \[
\left| (f_n \ast g)(x) - g(x) \right| \le 4M\epsilon(1-\delta) + \epsilon + 2\epsilon^2(1-\delta)
    \]

\end{proof}

\begin{exercise}
    prove: exists $c_n$ meets:

    \begin{enumerate}
        \item

        \[
            \int_{[-1,1]}c_n(1-x^2)^n \mathrm{d}x = 1
        \]

        \item  
        
        \[
            \forall 0 < |x| \le 1, \: c_n(1-x^2)^n \to 0
        \]

    \end{enumerate}
\end{exercise}

\begin{proof}
    consider:

    \begin{align*}
        \int_{[-1,1]}(1-x^2)^n\mathrm{d}x &\ge \int_{[-1,1]}(1-|x|)^n\mathrm{d}x \\
        & \ge 2 \frac{1}{n+1}
    \end{align*}

    if we assign:


    \[
        c_n = \left(\int_{[-1,1]}(1-x^2)^n\mathrm{d}x \right)^{-1}
    \]

    then we got $c_n \le \frac{n+1}{2}$, consider any $0 < |x| \le 1$, then we got $0 \le 1-x^2 < 1$ and
    $c_n(1-x^2)^n \le \frac{n+1}{2}(1-x^2)^n$

    let $\epsilon = 1-x^2$ and $\epsilon < 1$, then we got

    \[
        \lim_{n \to \infty}\frac{n+1}{2}\epsilon ^n = 0
    \]

    because

    \[
        \varlimsup_{n \to \infty}\frac{n+2}{n+1}\epsilon = \epsilon < 1
    \]
\end{proof}

\subsubsection{Algebra}

\begin{exercise}
    let $G$ be a semi group, ant it contains a left identity $1_l$ and a right identity $1_r$,
    prove $1_l = 1_r$
\end{exercise}

\begin{proof}
    by definition of identity:

    \[
        1_l 1_r = 1_r = 1_l
    \]
\end{proof}


\begin{exercise}
    let $G$ be a monoid, and every element $a$ has a left inverse $a^{-1}l$, 
    prove: every element also has at least one right inverse, and left inverse is right inverse
\end{exercise}

\begin{proof}
    let $ba = 1$ and $cb = 1$, then we have:

    \[
        cba=(cb)a = a = c(ba) = c
    \]

    so we got $ab = cb= 1 $, so $b$ is also a right inverse of $a$
\end{proof}

\begin{exercise}
   let $G$ be a monoid, and every element $a$ has a left inverse $a^{-1}l$, then left inverse is unique
\end{exercise}

\begin{proof}
    let $ba = 1$ and $ca= 1$, since $a$ has at least a right inverse $d$, we got:
    
    \[
        ba = ca,\: b(ad) = c(ad) = b  = c
    \]
\end{proof}