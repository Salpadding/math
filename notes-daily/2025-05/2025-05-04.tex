%!LW recipe=latexmk
\documentclass[11pt,a4paper]{article}
\input% math 
\usepackage{amsmath,amsfonts,amssymb,amsthm}
% cross reference, use \autoref instead of \ref
\usepackage{aliascnt}
\usepackage[hidelinks]{hyperref}
\usepackage{enumitem}
\usepackage{geometry}


\geometry{left=1.5cm, right=1.5cm, top=2cm, bottom=2cm}

\newtheorem{thm}{Theorem}[section]

\newaliascnt{lem}{thm}
\newaliascnt{prop}{thm}
\newaliascnt{definition}{thm}
\newaliascnt{exercise}{thm}
\newaliascnt{corollary}{thm}

\theoremstyle{definition}
\newtheorem{lem}[lem]{Lemma}
\newtheorem{prop}[prop]{Proposition}
\newtheorem{definition}[definition]{Definition}
\newtheorem{exercise}[exercise]{Exercise}
\newtheorem{corollary}[corollary]{Corollary}

\def\lemautorefname{Lemma}
\def\thmautorefname{Theorm}
\aliascntresetthe{lem}
\aliascntresetthe{prop}
\aliascntresetthe{definition}
\aliascntresetthe{exercise}
\aliascntresetthe{corollary}


\title{Learning Notes}
\author{Yingjie Zhu}
\date{\today}


\begin{document}
\maketitle

\section{Theory of Real Variable Function}

\begin{exercise}
    if $f_n: E \to \mathbb{R}$ is measurable, and $\forall \delta > 0, \exists E_{\delta} \subseteq E$ such that
    $m(E_{\delta}) < \delta$ and $f_n: E \setminus E_{\delta} \to \mathbb{R}$ converges to $f$ uniformly.

    Then $f_n \to f$ converges in measure.
\end{exercise}

\begin{proof}
    we fix $\epsilon > 0$ at first, by definition of uniformly convergence. There exists $N$ and $\forall n \ge N$, we have

    \[
        E \setminus E_{\delta} \subseteq \{ x : \lvert f_n(x) - f(x) \rvert < \epsilon \}
    \]

    so we have


    \begin{align*}
        \{ x : \lvert f_n(x) - f(x) \rvert \ge \epsilon \} & \subseteq E_{\delta} \\
        m(\{ x : \lvert f_n(x) - f(x) \rvert \ge \epsilon \}) & \le \delta
    \end{align*}

    let 
    \[
        a_n = m(\{ x : \lvert f_n(x) - f(x) \rvert \ge \epsilon \})
    \]

    then take limsup and liminf we have

    \[
       0 \le \varliminf_{n \to \infty}a_n \le \varlimsup_{n \to \infty}a_n \le \delta
    \]

    take $\delta \to 0$, then we have

    \[
        \lim_{n \to \infty a_n} = 0
    \]

    so we got

    \[
        \lim_{n \to \infty} m(\{ x : \lvert f_n(x) - f(x) \rvert \ge \epsilon \}) = 0
    \]
\end{proof}

\begin{exercise}
    $f_n$ is Cauchy in measure, prove: $f_n$ converges to measurable function $f$ in measure
\end{exercise}

\begin{proof}
    since $f_n$ is Cauchy in measure,take $i=1,2,3,..$ we have

    \[
        \lim_{n,m \to \infty} m(\{ \lvert f_n(x) - f_m(x)\rvert \ge \frac{1}{2^i} \}) = 0
    \]

    for any $i$ exists $N_i$,for all $n,m \ge N_i$ have

    \[
        m(\{ \lvert f_n(x) - f_m(x)\rvert \ge \frac{1}{2^i} \}) < 2^{-i}
    \]

    take $k_1 = N_1, k_2 = \max(k_1 + 1, N_2), ..$ we got $f_{k_i}$

    define $E_i$ as:

    \[
        E_i = \{ \lvert f_{k_i}(x) - f_{k_{i+1}}(x)\rvert \ge \frac{1}{2^i} \}
    \]

    Since

    \[
        \sum_{i=1}^{\infty}m(E_i) < \infty
    \]

    we have

    \[
        m(S) = m(\bigcap_{j=1}^{\infty}\bigcup_{i=j}^{\infty}E_i) = 0
    \]

    we will prove: $f_{k_i}$ converges point wise on $E \setminus S$ 
    
    for any $x \notin S$ we got $i_0$ so that $\forall i \ge i_0$

    \[
        \lvert f_{k_{i}}(x) - f_{k_{i+1}}(x) \rvert < 2^{-i}
    \]

    so
    \[
    f_{k_{i_0 + t}}(x) = f_{k_{i_0}}(x) + \sum_{j=1}^{t}\left( f_{k_{i_0+j}}(x) - f_{k_{i_0+j-1}}(x) \right)
    \]

    converges absolutely, take $t \to \infty$ we got

    \[
        \lim_{i \to \infty}f_{k_i}(x) = f(x)
    \]

    let

    \[
        S_j = \bigcup_{i=j}^{\infty}E_i
    \]

    next we will show $f_{k_i}$ converges on $E \setminus S_j$ uniformly.
    
    take $x \in E \setminus S_j$ we got

    $\forall i \ge j$

    \[
        \lvert f_{k_i}(x) - f_{k_{i+1}}(x) \rvert < 2^{-i}
    \]

    so $j$ is not related with $x$ , because $S \subseteq S_j$ so $ E \setminus S_j \subseteq E \setminus S $ 

    so $f_{k_i}$ converges to $f$ on $E \setminus S_j$ uniformly

    we got that

    \[
        \lim_{j \to \infty}m(S_j) = 0
    \]

    by our previous proved lemma, $f_{k_i}$ converges to $f$ on $E$ in measure

    by inequality

    \[
        \lvert f_k(x) - f(x)\rvert \le \lvert f_k(x) - f_{k_i}(x) + f_{k_i}(x) - f(x)\rvert
    \]

    we got

    \begin{align*}
        \{x: \lvert f_k(x) - f(x)\rvert \ge \epsilon \} & \subseteq  \bigcap_{i=1}^{\infty} \{x: \lvert f_k(x) - f_{k_{i}}(x)\rvert + \lvert f_{k_{i}}(x) - f(x)\rvert \ge \epsilon \} \\
        & \subseteq \bigcap_{i=1}^{\infty} \{x: \lvert f_k(x) - f_{k_{i}}(x) \rvert \ge \frac{\epsilon}{2}  \} \cup \{x: \lvert f_{k_i}(x) - f(x) \rvert \ge \frac{\epsilon}{2}  \}
    \end{align*}

    so $\forall i$

    \[
        m\left( \{x: \lvert f_k(x) - f(x)\rvert \ge \epsilon \} \right) \le m\left(\{x: \lvert f_{k}(x) - f_{k_i}(x) \rvert \ge \frac{\epsilon}{2} \} \right) + m\left(\{x: \lvert f_{k_i}(x) - f(x) \rvert \ge \frac{\epsilon}{2} \right)
    \]

    because $f_n$ is Cauchy in measure, and $f_{k_i}$ converges to $f$ in measure. 
    
    let $k \ge N,\, i \ge N$  we take $N \to \infty$ and got

    \[
        \lim_{k \to \infty} m\left( \{x: \lvert f_k(x) - f(x)\rvert \ge \epsilon \} \right) \le  0
    \]

\end{proof}

\end{document}














