\subsubsection{Mathematical Analysis}

\begin{exercise}
    prove: if $A \subseteq X$ meets: every sequence $x_n \in A$ has at least one convergent sub sequence, where $(X,d)$ is a complete metric space, then $A$ has finite sub cover 
    if $A$ has a open cover.
\end{exercise}

\begin{proof}
    let $A$ have a group of open cover
    \[
        A \subseteq \bigcup_{\alpha \in I}V_{\alpha}
    \]

    and define $r(x): A \to \mathbb{R}^*$

    \[
        r(x) = \sup_{r \ge 0} \{ r: B(x,r) \subseteq V_{\alpha},\: \alpha \in I \}
    \]

    and define $r_0$ as

    \[
        r_0 = \inf_{x \in A} r(x)
    \]

    then we discuss on $r_0$:

    \begin{enumerate}
        \item we prove a lemma at first: for any convergent $x_n \in A$, we have

        \[
            \lim_{n \to \infty}r(x_n) \ge r(\lim_{n \to \infty}x_n)
        \]

        consider $x \in A$ and

        \[
            \lim_{n \to \infty}x_n = x
        \]

        we prove that for any $n$

        \[
            r(x_n) \ge r(x) - d(x_n, x)
        \]

        because for any $y \in A$, we have

        \begin{align*}
             d(y, x) &\le d(y,x_n) + d(x_n, x) \\
        \end{align*}

        so if $d(y, x_n) < r(x) - d(x_n, x)$

        we have

        \[
            d(y,x) < r(x)
        \]

        thus we have

        \[
            B(x_n, r(x)- d(x_n, x)) \subseteq B(x, r(x))
        \]

        by definition of $r(x_n)$ we have

        \[
            r(x_n) \ge r(x) - d(x_n, x)
        \]

        take $n \to \infty$ we have

        \[
            \lim_{n \to \infty}r(x_n) \ge r(x)
        \]


        \item consider if $r_0 = 0$

        by definition of $\inf$, there exists a sequence $x_n \in A$, such that

        \[
            \lim_{n \to \infty}r(x_n) = 0
        \]

        since $x_n \in A$, by condition there exists a sub sequence $x_{n_k}$ such that

        \[
            \lim_{k \to \infty}x_{n_{k}} = x_0 
        \]

        by our previous lemma we have

        \[
            r(x_0) \le \lim_{k \to \infty}r(x_{n_{k}}) \le 0
        \]

        however since $x_0$ belongs to at lease one open set $V_{\alpha}$, so $r(x_0) > 0$, 
        so $r_0 \ne 0 $

        \item consider $0 < r_0 \le \infty$

        we  take $0 < r_1 < \infty$ and $r_1 < r_0$, then for every $x \in A$, there exists
        a ball $B(x, r_x),\, r_x \ge r_1$ and $B(x, r_x) \subseteq V_{\alpha}$ for some $\alpha \in I$

        then we can construct a sequence $x_n$:

        \begin{align*}
            & x_1 \in A\: & B(x_1, r_1) \subseteq V_{1} \\
            & x_2 \in A \setminus (V_1)\: & B(x_2, r_2) \subseteq V_2 \\
            & x_3 \in A \setminus (V_1 \cup V_2) & \: B(x_3, r_3) \subseteq V_3 \\
            & ...
        \end{align*}

        since $A$ cannot covered by finite open sets $V_1, V_2, .. V_n$, so we can construct infinite $x_n$.
        it is obviously that for any $j > i$ we have $d(x_i, x_j) \ge r_1$

        so $x_n$ cannot contains any convergent sub sequence which is contradict.



    \end{enumerate}
\end{proof}

\begin{exercise}
    let $I \subseteq \mathbb{R}^n$ and $a_1 \le b_1, a_2 \le b_2, .., a_n \le b_n$

    \[
        I = \{ (x_1,x_2, .. ,x_n) : a_i \le x_i \le b_i \}
    \]

    prove that $I$ is compact
\end{exercise}

\begin{proof}
    \begin{enumerate}
        \item Lemma 1: let $J_1 = [a_1, b_1], J_2 = [a_2, b_2], ..$ and $J_1 \supseteq J_2 \supseteq J_3 .. $ then:

        \[
            \bigcap_{n=1}^{\infty}J_n \ne \emptyset
        \]

        since we have $a_1 \ge a_2 \ge a_3$ and for $a_n$ has upper bound $b_1,b_2, .. $, so we take

        \[
            x = \sup_{n \ge 1}a_n
        \]

        then we have for any $n$, $a_n \le x$. since for any $n,\: b_n$ is also an upper bound so we have

        \[
            a_n \le x \le b_n
        \]
        so we have

        \[
            x \in \bigcap_{n=1}^{\infty}J_n
        \]

        \item Lemma 2: let 

        \[
            J_m = \{ (x_1,x_2, .. ,x_n) : a_i^{(m)} \le x_i \le b_i^{(m)} \}
        \]

        and $J_1 \supseteq J_2 \supseteq J_3 .. $ then we have

        \[
            \bigcap_{n=1}^{\infty}J_n \ne \emptyset
        \]

        we define a function $f: \mathbb{R}^n \to \mathbb{R}$:

        \[
            f((x_1, x_2, .. ,x_n)) = x_1
        \]

        then we have

        \[
            f(J_m) = [a_1^{(m)}, b_1^{(m)}] 
        \]

        and

        \[
            f^{-1}([a_1^{(m)}, b_1^{(m)}]) = J_m
        \]

        by $J_1 \supseteq J_2$, we have

        \[
            f(J_1) \supseteq f(J_2):\: [a_1^{(1)}, b_1^{(1)}] \supseteq [a_1^{(2)}, b_1^{(2)}]
        \]

        so there exists $c_1$ such that

        \[
            c_1 \in \bigcap_{m=1}^{\infty}f(J_m) = \bigcap_{m=1}^{\infty}[a_1^{(m)}, b_1^{(m)}]
        \]

        replace $f$ with

        \[
            g((x_1, x_2, .. ,x_n)) = x_2
        \]

        we can obtain $c_2$ such that

        \[
            c_2 \in \bigcap_{m=1}^{\infty}[a_2^{(m)}, b_2^{(m)}]
        \]

        so we construct $c_1, c_2, .. c_n$ and we have

        \[
            (c_1, c_2, .., c_n) \in \bigcap_{m=1}^{\infty}J_m
        \]

        \item $J$ has finite sub cover

        assume $J$ has a open cover, and this open cover contains none finite sub cover.
        let $J$ as

        \[
            J = \{ (x_1,x_2, .., x_n): a_i \le x_i \le b_i\} \quad a_i \le b_i
        \]

        and

        \[
            J \subseteq \bigcup_{\alpha \in I}V_{\alpha}
        \]

        and define

        \[
            \| J \|_{\infty} = \max_{1 \le i \le n}b_i - a_i
        \]

        we will divide $J$ by $n$ times, and get $2^n$ sub sets

        \[
            J = \bigcup_{k=0}^{2^n-1} \{ (x_1, x_2, .., x_n): l_i^{(k)} \le x_i \le r_i^{(k)} \}
        \]

        and

        \begin{align*}
            l_i^{(k)} = \begin{cases}
                a_i & k < 2^{i-1} \\
                \frac{a_i + b_i}{2} & k \ge 2^{i-1} \\
            \end{cases} \\
            r_i^{(k)} = \begin{cases}
                \frac{a_i + b_i}{2} & k < 2^{i-1} \\
                b_i & k \ge 2^{i-1} \\
            \end{cases}
        \end{align*}

        since $J$ has none finite sub cover, there should exists $0 \le k \le 2^{n}-1$ such that

        \[
            \{ (x_1, x_2, .., x_n): l_i^{(k)} \le x_i \le r_i^{(k)} \}
        \]

        also has none finite sub cover, we pick this parts $J_1$ and we have

        \[
            \| J_1 \|_{\infty} = \frac{1}{2^n} \| J \|_{\infty}
        \]

        and we can divide $J_1$ then got $J_2$, consider we pick $m$ times and we have

        \[
            \| J_m \|_{\infty} = \frac{1}{2^{m}} \| J \|_{\infty}
        \]

        since  $J \supseteq J_1 \supseteq J_2 .. $ so we have

        \[
            \bigcap_{m=1}^{\infty}J_m \ne \emptyset
        \]

        pick $x$

        \[
            x \in \bigcap_{m=1}^{\infty}J_m 
        \]

        since $x \in V_{\alpha}$ for some $\alpha \in I$, and $V_{\alpha}$ is open, there should exists $r$ such that

        \[
            B(x, r) \subseteq V_{\alpha}
        \]

        we take $m$ be large enough such that

        \[
 \| J_m \|_{\infty}  = \frac{1}{2^{m}} \| J \|_{\infty} < \frac{r}{n}
        \]

        for any $y \in J_m$ we have

        \begin{align*}
            d_{l_2}(y,x) & \le d_{l_1}(y,x) \\
            & \le n d_{l_\infty}(y,x) \le n  \| J_m \|_{\infty} \\
            & < r
        \end{align*}

        so we have

        \[
            J_m \subseteq B(x, r) \subseteq V_{\alpha}
        \]

        so $J_m$ has a finite sub cover, which is contradict.

    \end{enumerate} 
\end{proof}

\begin{exercise}
    On topological space $(X, \mathcal{F})$, if every singe point set is closed, then every derive set is closed 
\end{exercise}

\begin{proof}
   let $E \subseteq X$ and 
   
   \[
    E' = \{ x: \forall V \in \mathcal{F},\, x \in V,\, V \cap (E \setminus \{ x\}) \ne \emptyset \}
   \]

   then we have

   \[
    (E')^C = \{ x: \exists V \in \mathcal{F},\, x \in V,\, V \cap (E \setminus \{ x\}) = \emptyset \}
   \]

   consider that any $y \in V,\, y \ne x$: we have a open set $V \setminus (\{ x\})$:

   \begin{align*}
    (V \setminus \{x\} ) \cap (E \setminus \{y \}) &= V \cap E \cap (\{ x, y\})^C \\
    & \subseteq \{ x \} \cap (\{x, y\})^C = \emptyset
   \end{align*}


   so for any $x \in (E')^C$ we have a open set $V_x$ which contains $x$ and

   \begin{align*}
        V_x \subseteq (E')^C
   \end{align*}

   
   so $(E')^C$ can be written as union of open sets:

   \[
    (E')^C = \bigcup_{x \in (E')^C} V_x
   \]

   and $E'$ is closed

\end{proof}

\begin{exercise}
    let $I_1, I_2, .., I_n$ be disjoint and non-empty closed interval. and $J$ is a bounded and nonempty open interval and

    \[
        J \subseteq \bigcup_{k=1}^{n}I_k
    \]

    prove: exists $1\le p \le n$, $J \subseteq I_p$
\end{exercise}

\begin{proof}
    pick $x_0 \in J$, assume $x_0 \in I_{q_0}$ for some $1\le q_0 \le n$, we prove that $J \subseteq I_{q_0}$.

    assume $\exists x_1 \in J, x_1 \ne x_0, x_1 \in I_{q_1}, q_1 \ne q_0$
    assume $x_0 < x_1$ there should exists $x_2, x_0 < x_2 < x_1$, and
    $x_2 \notin I_{q_0}, x_2 \notin I_{q_1}$. otherwise we have $(x_0, x_1) \subseteq I_{q_0} \cup I_{q_1}$,
    which is contradict with they are disjoint bounded and closed interval.

    since $x_2 \in J$, there should exists $I_{q_2}$ such that $x_2 \in I_{q_2}$, and $q_2, q_1, p$ is distinct.
    because $I_{q_0}$ and $I_{q_2}$ is disjoint, there should exists $x_3, x_0 < x_3 < x_2$
    and $x_3 \notin I_{q_0}, x_3 \notin I_{q_2}$

\end{proof}

















